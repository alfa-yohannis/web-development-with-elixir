\documentclass[aspectratio=169, table]{beamer}

\usepackage[utf8]{inputenc}
\usepackage{listings}

\usetheme{Pradita}

\subtitle{IF140303-Web Application Development}

\title{Session-10:\\ \LARGE{OAuth using GitHub and\\Phoenix Framework}
	\vspace{5pt}}
\date[Serial]{\scriptsize {PRU/SPMI/FR-BM-18/0222}}
\author[Pradita]{\small{\textbf{Alfa Yohannis}}}

\lstdefinelanguage{Elixir} {
	keywords={def, defmodule, do, end, for, if, else, true, false},
	basicstyle=\ttfamily\small,
	keywordstyle=\color{blue}\bfseries,
	ndkeywords={@moduledoc, iex, Enum, @doc},
	ndkeywordstyle=\color{purple}\bfseries,
	sensitive=true,
	commentstyle=\color{gray},
	stringstyle=\color{red},
	numbers=left,
	numberstyle=\tiny\color{gray},
	breaklines=true,
	frame=lines,
	backgroundcolor=\color{lightgray!10},
	tabsize=2,
	comment=[l]{\#},
	morecomment=[s]{/*}{*/},
	commentstyle=\color{gray}\ttfamily,
	stringstyle=\color{purple}\ttfamily,
	showstringspaces=false
}

\begin{document}
	
	\frame{\titlepage}
	
	\begin{frame}
		\frametitle{Saving Data in Phoenix}
		\begin{itemize}
			\item This session covers the process of saving user input to the database using Phoenix and Elixir.
			\item We will explore the workflow from user input to data persistence.
			\item We’ll also see how to handle errors and enhance the user experience with CSS.
		\end{itemize}
	\end{frame}
	
	\begin{frame}
		\frametitle{\shortstack{Understanding the \\ Save Workflow}}
		\begin{itemize}
			\item The process begins with user input parameters submitted via a form.
			\item The `changeset` function validates this input.
			\item If validation succeeds, data is saved to the database.
			\item The user is then redirected to the topic list with a success message.
			\item If validation fails, the form is re-rendered with error messages.
		\end{itemize}
	\end{frame}
	
	\begin{frame}
		\frametitle{\shortstack{Understanding \\ Ecto}}
		\begin{itemize}
			\item **Ecto** is a toolkit for data mapping and language integrated queries.
			\item It provides tools to define schemas, validate data, and interact with databases.
			\item The `changeset` function in Ecto allows data validation and casting.
			\item **Repo** is an Ecto module responsible for persisting data to the database.
		\end{itemize}
	\end{frame}
	
	\begin{frame}[fragile]
		\frametitle{Updating the Create Function}
		\begin{lstlisting}[language=Elixir]
			defmodule Discuss.TopicController do
			use Discuss.Web, :controller
			
			alias Discuss.Topic
			
			def new(conn, _params) do
			changeset = Topic.changeset(%Topic{}, %{})
			render conn, "new.html", changset: changeset
			end
		\end{lstlisting}
	\end{frame}
	
	\begin{frame}[fragile]
		\frametitle{Updating the Create Function}
		\begin{lstlisting}[language=Elixir]
			def create(conn, %{"topic" => topic}) do
			changeset = Topic.changeset(%Topic{}, topic)
			case Repo.insert(changset) do
			{:ok, _topic} -> IO.inspect(post)
			{:error, changeset} -> 
			render conn, "new.html", changeset: changeset
			end
			end
			end
		\end{lstlisting}
	\end{frame}
	
	\begin{frame}
		\frametitle{Explaining the Create Function}
		\begin{itemize}
			\item **new/2**: Initializes a new changeset for an empty `Topic`.
			\item **create/2**: Handles form submission and attempts to save the new topic.
			\item The `changeset` function validates the input data and prepares it for saving.
			\item **Repo.insert/2**: Attempts to insert the validated data into the database.
			\item If successful, the topic is saved and a success message is shown.
			\item If it fails, the form is re-rendered with error messages.
		\end{itemize}
	\end{frame}
	
	\begin{frame}[fragile]
		\frametitle{Error Handling with error\_tag}
		\begin{lstlisting}[language=Elixir]
			<%= form_for @changeset, topic_path(@conn, :create), fn f -> %>
			<div class="form-group">
			<%= text_input f, :title, placeholder: "Title", class: "form-control" %>
			<%= error_tag f, :title %>
			</div>
			<%= submit "Save Topic", class: "btn btn-primary" %>
			<% end % >
		\end{lstlisting}
	\end{frame}
	
	\begin{frame}
		\frametitle{Understanding error\_tag}
		\begin{itemize}
			\item `error\_tag` helps in displaying error messages next to form fields.
			\item It checks if the `changeset` contains errors for a specific field.
			\item If an error exists, it generates an HTML element showing the error message.
			\item This improves user experience by providing immediate feedback.
		\end{itemize}
	\end{frame}
	
	\begin{frame}[fragile]
		\frametitle{\shortstack{Adding CSS to the Project}}
		\begin{itemize}
			\item Add the following CSS to `web > static > css > app.css`:
		\end{itemize}
		\begin{lstlisting}[language=Elixir]
			.help-block{
				color: red;
				text-transform: capitalize;
				position: absolute;
				top: 37px;
			}
			
			.form-group{
				position: relative;
				margin-bottom: 25px;
			}
		\end{lstlisting}
	\end{frame}
	
	\begin{frame}
		\frametitle{\shortstack{Impact of Adding \\ CSS}}
		\begin{itemize}
			\item Adding CSS to `app.css` impacts the entire project.
			\item The `.help-block` class styles error messages, making them more noticeable.
			\item The `.form-group` class provides structure and spacing to form elements.
			\item These styles enhance the overall user interface and user experience.
		\end{itemize}
	\end{frame}
	
	\begin{frame}
		\frametitle{Summary}
		\begin{itemize}
			\item We explored the workflow of saving data in Phoenix using Ecto.
			\item We updated the create function and handled errors with `error\_tag`.
			\item Finally, we enhanced the user interface with custom CSS.
		\end{itemize}
	\end{frame}
	
	\begin{frame}[fragile]
		\frametitle{Updating the Router}
		\begin{itemize}
			\item The router is updated to change the root path to show a list of topics.
			\item This change breaks from the RESTful convention by routing "/" to the `TopicController`.
			\item Remove unused `PageController` components.
		\end{itemize}
		\begin{lstlisting}[language=Elixir]
			scope "/", Discuss do
			pipe_through :browser
			get "/", TopicController, :index
			get "/topics/new", TopicController, :new
			post "/topics", TopicController, :create
		\end{lstlisting}
	\end{frame}
	
	\begin{frame}
		\frametitle{Explanation of the Router Update}
		\begin{itemize}
			\item The root path ("/") now directs users to the `TopicController`'s `index` action.
			\item The `index` action will display a list of topics.
			\item Removed `PageController` since it’s no longer needed.
			\item The `scope` block specifies the routing paths for the Discuss application.
		\end{itemize}
	\end{frame}
	
	\begin{frame}[fragile]
		\frametitle{\shortstack{Adding the Index Function \\to TopicController}}
		\begin{lstlisting}[language=Elixir]
			defmodule Discuss.TopicController do
			use Discuss.Web, :controller
			
			alias Discuss.Topic
			
			def index(conn, _params) do
			topics = Repo.all(Topic)
			render conn, "index.html", topics: topics
			end
		\end{lstlisting}
	\end{frame}
	
	\begin{frame}
		\frametitle{Explanation of the Index Function}
		\begin{itemize}
			\item The `index` function retrieves all topics from the database using `Repo.all/1`.
			\item These topics are passed to the "index.html" template for rendering.
			\item The `topics` list is available in the template as `@topics`.
		\end{itemize}
	\end{frame}
	
	\begin{frame}[fragile]
		\frametitle{Creating the Index Template}
		\begin{lstlisting}[language=elixir]
			<h5>Topics</h5>
			
			<ul class="collection">
			<%= for topic <- @topics do %>
			<li class="collection-item">
			<%= topic.title %>
			</li>
			<% end %>
			</ul>
		\end{lstlisting}
	\end{frame}
	
	\begin{frame}
		\frametitle{Explanation of the Index Template}
		\begin{itemize}
			\item The template displays a list of topics.
			\item The `<ul>` element represents the unordered list, styled with the `collection` class.
			\item The `for` loop iterates over each topic and renders its title inside a `<li>` element.
		\end{itemize}
	\end{frame}
	
	\begin{frame}[fragile]
		\frametitle{Updating the Create Function}
		\begin{lstlisting}[language=Elixir]
			def create(conn, %{"topic" => topic}) do
			changeset = Topic.changeset(%Topic{}, topic)
			case Repo.insert(changeset) do
			{:ok, _topic} -> 
			conn
			|> put_flash(:info, "Topic Created")
			|> redirect(to: topic_path(conn, :index))
			{:error, changeset} -> 
			render conn, "new.html", changeset: changeset
			end
			end
		\end{lstlisting}
	\end{frame}
	
	\begin{frame}
		\frametitle{\shortstack{Explanation of the Updated \\Create Function}}
		\begin{itemize}
			\item If `Repo.insert/1` is successful, a success message is flashed.
			\item The user is then redirected to the `index` action, displaying the list of topics.
			\item If the insert fails, the form is re-rendered with errors for correction.
		\end{itemize}
	\end{frame}
	
	\begin{frame}[fragile]
		\frametitle{\shortstack{Adding a Button to the \\Index Template}}
		\begin{lstlisting}[language=Elixir]
			<div class="fixed-action-btn">
			<%= link to: topic_path(@conn, :new), class: "btn-floating btn-large waves-effect waves-light red" do %>
			<i class="material-icons">add</i>
			<% end %>
			</div>
		\end{lstlisting}
	\end{frame}
	
	\begin{frame}
		\frametitle{Explanation of the Button Addition}
		\begin{itemize}
			\item The button is created using the `link/2` helper, directing users to the `new` action.
			\item `topic\_path(@conn, :new)` generates the appropriate URL for the `new` topic form.
			\item The button uses Materialize CSS for styling, with a floating effect and an add icon.
		\end{itemize}
	\end{frame}
	
	\begin{frame}[fragile]
		\frametitle{Adding Material Icons}
		\begin{lstlisting}[language=elixir]
			<link rel="stylesheet" href="https://fonts.googleapis.com/icon?family=Material+Icons">
		\end{lstlisting}
	\end{frame}
	
	\begin{frame}
		\frametitle{Explanation of Material Icons}
		\begin{itemize}
			\item The `<link>` tag includes Material Icons from Google Fonts into the project.
			\item This allows the use of icons like the "add" icon in buttons or other UI elements.
		\end{itemize}
	\end{frame}
	
	\begin{frame}[fragile]
		\frametitle{Adding Edit and Update Routes}
		\begin{lstlisting}[language=Elixir]
			scope "/", Discuss do
			pipe_through :browser
			
			get "/", TopicController, :index
			get "/topics/new", TopicController, :new
			post "/topics", TopicController, :create
			get "/topics/:id/edit", TopicController, :edit
			end
		\end{lstlisting}
	\end{frame}
	
	\begin{frame}
		\frametitle{Explanation of the Edit Route}
		\begin{itemize}
			\item The `edit` route allows users to load a form to edit an existing topic.
			\item The `:id` parameter in the path corresponds to the topic being edited.
			\item The `edit` action will retrieve the topic and display it for editing.
		\end{itemize}
	\end{frame}
	
	\begin{frame}[fragile]
		\frametitle{\shortstack{Adding the Edit Function\\to TopicController}}
		\begin{lstlisting}[language=Elixir]
			def edit(conn, %{"id" => topic_id}) do
			topic = Repo.get(Topic, topic_id)
			changeset = Topic.changeset(topic)
			
			render conn, "edit.html", changeset: changeset, topic: topic
			end
		\end{lstlisting}
	\end{frame}
	
	\begin{frame}
		\frametitle{Explanation of the Edit Function}
		\begin{itemize}
			\item The `edit` function retrieves the topic to be edited using `Repo.get/2`.
			\item A `changeset` is created for the topic, allowing for validation and form rendering.
			\item The `edit.html` template is rendered with the `changeset` and `topic` data.
		\end{itemize}
	\end{frame}
	
	\section{Editing and Deleting Topics}
	
	\begin{frame}{\shortstack{Adding \texttt{edit.html.eex} to \\ \texttt{templates > topic}}}
		\begin{itemize}
			\item \texttt{
				<\%= form\_for @changeset, topic\_path(@conn, :update, @topic), fn f -> \%> \\
				<div class="form-group"> \\
				<\%= text\_input f, :title, placeholder: "Title", class: "form-control" \%> \\
				<\%= error\_tag f, :title \%> \\
				</div> \\
				<\%= submit "Save Topic", class: "btn btn-primary" \%> \\
				<\% end \%>
			}
		\end{itemize}
	\end{frame}
	
	\begin{frame}{\shortstack{Adding \texttt{edit.html.eex} to \\ \texttt{templates > topic}}} 
		\begin{itemize}
			\item \texttt{form\_for @changeset}: Generates an HTML form bound to the \texttt{update} action.
			\item \texttt{topic\_path(@conn, :update, @topic)}: The URL for submitting form data to \texttt{update}.
			\item \texttt{text\_input f, :title}: Creates an input field for \texttt{title}, with validation.
			\item \texttt{error\_tag f, :title}: Displays error message if \texttt{title} is invalid.
			\item \texttt{submit "Save Topic"}: Submit button to save changes.
		\end{itemize}
	\end{frame}
	
	\begin{frame}
		\frametitle{\shortstack{Updating Router to Include\\\texttt{put "/topics/:id"}}}
		\begin{itemize}
			\item \texttt{
				scope "/", Discuss do \\
				pipe\_through :browser  \\
				get "/", PageController, :index \\
				get "/topics/new", TopicController, :new \\
				post "/topics", TopicController, :create \\
				get "/topics/:id/edit", TopicController, :edit \\
				put "/topics/:id", TopicController, :update \\
				end
			}
		\end{itemize}
	\end{frame}
	
	\begin{frame}
		\frametitle{\shortstack{Updating Router to Include\\\texttt{put "/topics/:id"}}}
		\begin{itemize}
			\item \texttt{put "/topics/:id"}: Defines the route for updating an existing topic.
			\item The route maps to the \texttt{update} action in the \texttt{TopicController}.
			\item \texttt{:id} is a placeholder for the topic’s ID.
		\end{itemize}
	\end{frame}
	
	\begin{frame}
		\frametitle{\shortstack{Updating \texttt{TopicController} with\\\texttt{update/2} Function}}
		\begin{itemize}
			\item \texttt{
				def update(conn, \%{"id" => topic\_id, "topic" => topic}) do \\
				old\_topic = Repo.get(Topic, topic\_id) \\
				changeset = Topic.changeset(old\_topic, topic) \\
				case Repo.update(changset) do \\
				{:ok, \_topic} -> \\
				conn \\
				|> put\_flash(:info, "Topic Updated") \\
				|> redirect(to: topic\_path(conn, :index)) \\
			}
		\end{itemize}
	\end{frame}
	
	\begin{frame}
		\frametitle{\shortstack{Updating \texttt{TopicController} with\\\texttt{update/2} Function}}
		\begin{itemize}
			\item \texttt{{:error, changset} -> \\
				render conn, "edit.html", changeset: changeset, topic: old\_topic \\
				end \\
				end \\}
			\item Retrieves the existing topic from the database using \texttt{Repo.get}.
			\item Creates a changeset to validate and apply updates.
			\item If update succeeds, flashes success message and redirects to \texttt{index}.
			\item If update fails, re-renders the edit page with the errors.
		\end{itemize}
	\end{frame}
	
	\begin{frame}
		\frametitle{\shortstack{Linking \texttt{index.html.eex}\\to Edit Pages}}
		\texttt{
			<h5>Topics</h5> \\
			<ul class = "collection"> \\
			<\%= for topic <- @topics do \%> \\
			<li class="collection-item"> \\
			<\%= topic.title \%> \\
			<div class="right"> \\
			<\%= link "Edit", to: topic\_path(@conn, :edit, topic) \%> \\
			</div> \\
			</li> \\
			<\% end \%> \\
			</ul>
		}
	\end{frame}
	
	\begin{frame}
		\frametitle{\shortstack{Linking \texttt{index.html.eex}\\to Edit Pages}}
		\begin{itemize}
			\item Lists all topics with an "Edit" link next to each one.
			\item \texttt{link "Edit"} generates a link to the edit page for each topic.
			\item The link is tied to the specific topic’s ID.
		\end{itemize}
	\end{frame}
	
	\begin{frame}{\shortstack{Updating Router with \\Resource Handler}}
		\texttt{
			scope "/", Discuss do\\
			resources "/", TopicController\\
			end\\
		}
		
		\begin{itemize}
			\item Simplifies routing by using a \texttt{resources} macro.
			\item Automatically generates RESTful routes for \texttt{TopicController}.
			\item Eliminates the need to manually define individual routes.
		\end{itemize}
	\end{frame}
	
	\begin{frame}
		\frametitle{\shortstack{Adding \texttt{delete/2} Function to\\\texttt{TopicController}}}
		\texttt{
			def delete(conn, \%{"id" => topic\_id}) do
			Repo.get!(Topic, topic\_id) |> Repo.delete!
			conn
			|> put\_flash(:info, "Topic Deleted")
			|> redirect(to: topic\_path(conn, :index)) 
			end
		}
		
		\begin{itemize}
			\item Retrieves the topic to delete using \texttt{Repo.get!}.
			\item Deletes the topic from the database using \texttt{Repo.delete!}.
			\item Flashes a success message and redirects to the \texttt{index} page.
		\end{itemize}
	\end{frame}
	
	\begin{frame}
		\frametitle{Linking \texttt{index.html.eex} to Delete Link}
		\texttt{
			<h5>Topics</h5>\\
			<ul class = "collection">\\
			<\%= for topic <- @topics do \%>\\
			<li class="collection-item"> \\
			<\%= topic.title \%>\\
			<div class="right">\\
			<\%= link "Edit", to: topic\_path(@conn, :edit, topic) \%>\\
			<\%= link "Delete", to: topic\_path(@conn, :delete , topic), method: :delete \%>\\
			</div>
			</li>
			<\% end \%> \\
			</ul>
		}
		
		\begin{itemize}
			\item Adds a "Delete" link next to each topic.
			\item The \texttt{method: :delete} ensures the correct HTTP method is used.
			\item The link is tied to the specific topic’s ID.
		\end{itemize}
	\end{frame}
	
\end{document}
