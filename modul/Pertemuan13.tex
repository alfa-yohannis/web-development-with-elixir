\chapter{Phoenix LiveView: Aplikasi Web Real-Time (bagian 2)}

\section{Pendahuluan}

Bab sebelumnya sudah membahas bagaimana aplikasi manajemen antrian yang dibangun mengelola nomor antrian. Akan tetapi, masih terdapat kekurangan di mana antrian satu masih bercampur dengan antrian yang lain. Misalnya, antrian A dinaikan satu nomor, maka antrian B juga akan naik. Bab ini berisi perbaikan dari masalah tersebut tidak ada pencampuran antara antrian dengan memberikan \texttt{topic} tersediri untuk masing-masing antrian.

\section{Penggunaan \texttt{Topic} Tersediri untuk Tiap Antrian pada \texttt{GenServer}}

Modul praktik ini menunjukkan penggunaan \texttt{GenServer} di Elixir untuk mengelola nomor antrian. Setiap antrian memiliki identitas unik berupa \texttt{queue\_id}. Modul ini menyediakan API untuk membaca dan memperbarui nomor antrian, serta menyiarkan pembaruan menggunakan \texttt{Phoenix.PubSub}. Desain ini memungkinkan pengelolaan status yang dinamis dan pembaruan secara real-time dalam sistem terdistribusi.

\subsection{Ikhtisar Modul}

Modul \texttt{Hello.QueueCounter} memiliki fungsi berikut:
\begin{enumerate}
	\item Mengelola status dalam bentuk \texttt{map}, di mana setiap kunci adalah \texttt{queue\_id} dan nilai adalah nomor saat ini untuk antrian tersebut.
	\item Menyediakan fungsi untuk membaca dan memperbarui nomor antrian.
	\item Menyiarkan pembaruan ke topik tertentu menggunakan \texttt{Phoenix.PubSub}.
\end{enumerate}

\subsection{Implementasi Kode}

Berikut adalah implementasi lengkapnya:

\begin{lstlisting}[language=Elixir, caption={lib/hello/queue\_counter.ex}]
	defmodule Hello.QueueCounter do
	use GenServer
	
	alias Phoenix.PubSub
	
	# Inisialisasi dengan map kosong untuk menyimpan nomor antrian
	def init(initial_map_state) do
	IO.puts("## initial_map_state")
	IO.inspect(initial_map_state)
	{:ok, initial_map_state}
	end
	
	# Memulai GenServer dengan map kosong
	def start_link(_) do
	start_link_initial_map_state = %{}
	IO.puts("## start_link_initial_map_state")
	IO.inspect(start_link_initial_map_state)
	GenServer.start_link(__MODULE__, start_link_initial_map_state, name: __MODULE__)
	end
	
	# API publik untuk mendapatkan nomor saat ini untuk queue_id tertentu
	def get_current_number(queue_id) do
	GenServer.call(__MODULE__, {:get_current_number, queue_id})
	end
	
	# API publik untuk memperbarui nomor saat ini untuk queue_id tertentu
	def set_current_number(queue_id, new_number) do
	GenServer.cast(__MODULE__, {:set_current_number, queue_id, new_number})
	end
	
	# Menangani pengambilan nomor saat ini untuk queue_id tertentu
	def handle_call({:get_current_number, queue_id}, _from, current_map_state) do
	IO.puts("## current_map_state")
	IO.inspect(current_map_state)
	
	# Default ke 0 jika queue_id tidak ditemukan
	current_number = Map.get(current_map_state, queue_id, 0)
	{:reply, current_number, current_map_state}
	end
	
	# Menangani pembaruan nomor untuk queue_id tertentu dan menyiarkan pembaruan
	def handle_cast({:set_current_number, queue_id, new_number}, current_map_state) do
	updated_map_state = Map.put(current_map_state, queue_id, new_number)
	
	PubSub.broadcast(
	Hello.PubSub,
	"queue_counter:updates:#{queue_id}",
	{:number_update, queue_id, new_number}
	)
	
	{:noreply, updated_map_state}
	end
	end
\end{lstlisting}

\subsection{Penjelasan Detail}

\paragraph{Inisialisasi Status}
Status \texttt{GenServer} diinisialisasi dalam fungsi \texttt{init/1} sebagai \texttt{map} kosong:
\begin{lstlisting}[language=Elixir]
	def init(initial_map_state) do
	{:ok, initial_map_state}
	end
\end{lstlisting}
Setiap kunci dalam \texttt{map} mewakili \texttt{queue\_id}, dan nilainya adalah nomor saat ini untuk antrian tersebut.

\paragraph{Memulai GenServer}
Fungsi \texttt{start\_link/1} memulai GenServer dengan status awal berupa \texttt{map} kosong:
\begin{lstlisting}[language=Elixir]
	def start_link(_) do
	GenServer.start_link(__MODULE__, %{}, name: __MODULE__)
	end
\end{lstlisting}

\paragraph{Membaca Nomor Antrian}
Fungsi \texttt{get\_current\_number/1} mengambil nomor saat ini untuk \texttt{queue\_id} tertentu:
\begin{lstlisting}[language=Elixir]
	def handle_call({:get_current_number, queue_id}, _from, current_map_state) do
	current_number = Map.get(current_map_state, queue_id, 0)
	{:reply, current_number, current_map_state}
	end
\end{lstlisting}
Jika \texttt{queue\_id} tidak ditemukan, nilai default adalah \texttt{0}.

\paragraph{Memperbarui Nomor Antrian}
Fungsi \texttt{set\_current\_number/2} memperbarui nomor untuk \texttt{queue\_id} tertentu dan menyiarkan pembaruan:
\begin{lstlisting}[language=Elixir]
	def handle_cast({:set_current_number, queue_id, new_number}, current_map_state) do
	updated_map_state = Map.put(current_map_state, queue_id, new_number)
	PubSub.broadcast("queue_counter:updates:#{queue_id}", {:number_update, queue_id, new_number})
	{:noreply, updated_map_state}
	end
\end{lstlisting}

\paragraph{Penyiaran Perubahan}
Pembaruan disiarkan ke topik \texttt{"queue\_counter:updates:\{queue\_id\}"}, sehingga komponen lain dapat merespons pembaruan secara real-time.

\subsection{Contoh Penggunaan}

Berikut adalah contoh penggunaan modul:
\begin{enumerate}
	\item Memulai \texttt{QueueCounter}:
	\begin{lstlisting}[language=Elixir]
		{:ok, _pid} = Hello.QueueCounter.start_link([])
	\end{lstlisting}
	
	\item Memperbarui dan membaca nomor antrian:
	\begin{lstlisting}[language=Elixir]
		Hello.QueueCounter.set_current_number(:queue_1, 5)
		IO.inspect(Hello.QueueCounter.get_current_number(:queue_1)) # Output: 5
	\end{lstlisting}
	
	\item Membaca nomor untuk antrian yang tidak ada:
	\begin{lstlisting}[language=Elixir]
		IO.inspect(Hello.QueueCounter.get_current_number(:queue_2)) # Output: 0
	\end{lstlisting}
	
	\item Mendengarkan pembaruan menggunakan PubSub:
	\begin{lstlisting}[language=Elixir]
		Hello.QueueCounter.set_current_number(:queue_1, 10)
		# Subscriber pada "queue_counter:updates:queue_1" akan menerima {:number_update, :queue_1, 10}
	\end{lstlisting}
\end{enumerate}

\section{Mengapa \texttt{queue\_counter:updates:\{queue\_id\}}?}

Pemilihan format topik \texttt{"queue\_counter:updates:\{queue\_id\}"} dalam fungsi \texttt{handle\_cast/2} dilakukan untuk mendukung skenario yang lebih fleksibel dan efisien dibandingkan hanya menggunakan topik umum seperti \texttt{"queue\_counter:updates"}. Berikut adalah alasannya:

\begin{enumerate}
	\item \textbf{Isolasi Notifikasi untuk Tiap Queue.} Format \texttt{"queue\_counter:updates:\{queue\_id\}"} membuat pembaruan nomor antrian dapat difokuskan hanya pada \texttt{queue\_id} tertentu. Subscriber (pihak yang mendengarkan pembaruan) hanya akan menerima notifikasi untuk antrian yang relevan. Sebagai contoh:
	\begin{itemize}
		\item Jika pembaruan dilakukan pada \texttt{queue\_id = :queue\_1}, maka hanya subscriber pada topik \texttt{"queue\_counter:updates:queue\_1"} yang akan menerima pemberitahuan.
		\item Hal ini mencegah subscriber lain menerima notifikasi yang tidak relevan.
	\end{itemize}
	Jika hanya menggunakan \texttt{"queue\_counter:updates"} sebagai topik global, maka semua subscriber akan menerima semua pembaruan, yang dapat menyebabkan:
	\begin{itemize}
		\item Beban komunikasi yang tidak perlu.
		\item Kebingungan karena subscriber harus memfilter data berdasarkan \texttt{queue\_id} secara manual.
	\end{itemize}
	
	\item \textbf{Efisiensi Kinerja.} Format \texttt{"queue\_counter:updates:\{queue\_id\}"} membuat topik menjadi spesifik sehingga pembaruan disiarkan hanya kepada subscriber yang benar-benar membutuhkan informasi tersebut. Ini mengurangi:
	\begin{itemize}
		\item Konsumsi bandwidth, karena data tidak dikirim ke subscriber yang tidak memerlukannya.
		\item Beban pemrosesan pada subscriber, karena mereka tidak perlu memeriksa apakah pembaruan relevan untuk \texttt{queue\_id} tertentu.
	\end{itemize}
	Sebaliknya, jika menggunakan topik umum, sistem akan mengirimkan pembaruan ke semua subscriber, yang dapat mengakibatkan bottleneck ketika jumlah subscriber atau frekuensi pembaruan meningkat.
	
	\item \textbf{Skalabilitas dan Ekspansi.} Pendekatan ini mendukung desain sistem yang skalabel:
	\begin{itemize}
		\item Dengan topik spesifik, mudah untuk menambah atau menghapus antrian tanpa memengaruhi subscriber lain.
		\item Jika aplikasi berkembang untuk mendukung ribuan \texttt{queue\_id}, desain ini tetap efisien karena setiap antrian dapat diisolasi secara logis.
	\end{itemize}
	Sebagai perbandingan, topik umum seperti \texttt{"queue\_counter:updates"} akan menyebabkan subscriber harus memproses semua pembaruan, yang tidak efisien jika jumlah antrian meningkat.
	
	\item \textbf{Pengelompokan yang Jelas.} 
	Topik yang spesifik memberikan struktur hierarki yang lebih jelas. Dalam format \texttt{"queue\_counter:updates:\{queue\_id\}"}, setiap antrian memiliki ruang pembaruan yang terpisah, sehingga mudah untuk:
	\begin{itemize}
		\item Melacak pembaruan berdasarkan \texttt{queue\_id}.
		\item Melakukan debugging jika terjadi masalah pada antrian tertentu.
	\end{itemize}
	
	\item \textbf{Contoh Implementasi.} Berikut adalah contoh kasus penggunaan dengan topik spesifik:
	\begin{enumerate}
		\item Subscriber hanya perlu mendengarkan topik \texttt{"queue\_counter:updates:queue\_1"} untuk \texttt{:queue\_1}:
		\begin{lstlisting}[language=Elixir]
			Phoenix.PubSub.subscribe(Hello.PubSub, "queue_counter:updates:queue_1")
		\end{lstlisting}
		
		\item Jika ada pembaruan untuk \texttt{:queue\_1}, hanya subscriber ini yang akan menerima pesan:
		\begin{lstlisting}[language=Elixir]
			Hello.QueueCounter.set_current_number(:queue_1, 10)
			# Subscriber akan menerima {:number_update, :queue_1, 10}
		\end{lstlisting}
	\end{enumerate}
\end{enumerate}


\section{DisplayOnlyLive: Komponen LiveView untuk Menampilkan Nomor Antrian}

Modul \texttt{HelloWeb.DisplayOnlyLive} adalah komponen LiveView yang bertanggung jawab untuk menampilkan nomor antrian secara real-time. Modul ini memanfaatkan PubSub untuk menerima pembaruan dari \texttt{QueueCounter} dan secara dinamis memperbarui tampilan pengguna.

\subsection{Kode Implementasi}

Berikut adalah implementasi lengkap dari \texttt{DisplayOnlyLive}:

\begin{lstlisting}[language=Elixir, caption={lib/hello\_web/live/display\_only\_live.ex}]
	defmodule HelloWeb.DisplayOnlyLive do
	use HelloWeb, :live_view
	alias Hello.QueueCounter
	alias Hello.Queue
	alias Hello.Repo
	
	def mount(params, _session, socket) do
	%{"id" => queue_id} = params
	
	# Subscribe to updates from the QueueCounter
	:ok = Phoenix.PubSub.subscribe(Hello.PubSub, "queue_counter:updates:#{queue_id}")
	
	# Get the initial current number of cases handled from QueueCounter
	current_number = QueueCounter.get_current_number(queue_id)
	
	queue = Repo.get!(Queue, queue_id)
	
	socket =
	socket
	|> assign(:current_number, current_number)
	|> assign(:queue_name, queue.name)
	|> assign(:queue_prefix, queue.prefix)
	|> assign(:queue_id, queue_id)
	
	{:ok, socket, layout: false}
	end
	
	def handle_info({:number_update, queue_id, new_number}, socket) do
	# Update the socket state when a new number is broadcasted
	socket =
	socket
	|> assign(:current_number, new_number)
	|> assign(:queue_id, queue_id)
	{:noreply,socket}
	end
	end
\end{lstlisting}

\subsection{Penjelasan Detail}

\textbf{1. Tujuan Modul.} Modul ini dirancang untuk menampilkan informasi nomor antrian yang diperbarui secara real-time kepada pengguna. Komunikasi real-time dicapai melalui integrasi dengan Phoenix PubSub.

\textbf{2. Fungsi \texttt{mount/3}.} Fungsi ini dijalankan ketika LiveView dimuat pertama kali. Berikut langkah-langkah yang dilakukan:
\begin{enumerate}
	\item \textbf{Mengambil Parameter.} Parameter \texttt{params} digunakan untuk mendapatkan \texttt{queue\_id} dari URL.
	\item \textbf{Berlangganan ke Topik PubSub.} LiveView ini berlangganan ke topik \texttt{"queue\_counter:updates:\{queue\_id\}"}, sehingga akan menerima pesan setiap kali nomor antrian diperbarui.
	\item \textbf{Mengambil Data Awal.} Nilai awal \texttt{current\_number} diambil dari modul \texttt{QueueCounter} menggunakan fungsi \texttt{get\_current\_number/1}.
	\item \textbf{Mengambil Informasi Antrian.} Informasi tambahan tentang antrian, seperti \texttt{queue\_name} dan \texttt{queue\_prefix}, diambil dari database menggunakan \texttt{Repo.get!/2}.
	\item \textbf{Inisialisasi Socket.} Data yang telah diambil diassign ke \texttt{socket}, sehingga dapat digunakan di tampilan.
\end{enumerate}

\textbf{3. Fungsi \texttt{handle\_info/2}.} Fungsi ini menangani pesan broadcast dari PubSub. Ketika nomor antrian diperbarui, fungsi ini:
\begin{enumerate}
	\item \textbf{Menerima Pesan Broadcast.} Pesan dalam format \texttt{\{:number\_update, queue\_id, new\_number\}} diterima dari PubSub.
	\item \textbf{Memperbarui Socket.} Nilai \texttt{current\_number} di \texttt{socket} diperbarui dengan nilai baru (\texttt{new\_number}), sehingga perubahan akan otomatis terlihat di antarmuka pengguna.
\end{enumerate}

\subsection{Poin Penting}

\begin{enumerate}
	\item \textbf{Integrasi PubSub.} Modul ini memanfaatkan \texttt{Phoenix.PubSub} untuk menerima pembaruan secara real-time, menjadikannya efisien untuk skenario di mana banyak pengguna mengakses data yang sama.
	\item \textbf{Pengelolaan Data.} Informasi yang relevan untuk tampilan, seperti \texttt{current\_number}, \texttt{queue\_name}, dan \texttt{queue\_prefix}, diambil dari modul \texttt{QueueCounter} dan database.
	\item \textbf{Desain yang Responsif.} Perubahan data pada backend langsung dipantulkan ke frontend tanpa perlu muat ulang halaman.
\end{enumerate}

\subsection{Contoh Penggunaan}

Berikut adalah contoh alur kerja LiveView ini:
\begin{enumerate}
	\item Ketika halaman LiveView dimuat, LiveView berlangganan ke topik PubSub untuk \texttt{queue\_id} tertentu dan menampilkan nomor antrian saat ini.
	\item Jika ada pembaruan nomor antrian (misalnya melalui \texttt{QueueCounter.set\_current\_number/2}), pesan broadcast dikirim ke topik \texttt{"queue\_counter:updates:\{queue\_id\}"}, dan fungsi \texttt{handle\_info/2} memperbarui tampilan pengguna.
\end{enumerate}

\section{ModifiableLive: Komponen LiveView untuk Memodifikasi Nomor Antrian}

Modul \texttt{HelloWeb.ModifiableLive} adalah komponen LiveView yang bertugas menampilkan dan memodifikasi nomor antrian secara real-time. Modul ini memungkinkan pengguna dengan otentikasi untuk meningkatkan nomor antrian, serta menerima pembaruan dari \texttt{QueueCounter} melalui PubSub.

\subsection{Kode Implementasi}

Berikut adalah implementasi lengkap dari \texttt{ModifiableLive}:

\begin{lstlisting}[language=Elixir, caption={lib/hello\_web/live/modifiable\_live.ex}]
	defmodule HelloWeb.ModifiableLive do
	use HelloWeb, :live_view
	alias Hello.QueueCounter
	alias Hello.Queue
	alias Hello.Repo
	
	def mount(params, session, socket) do
	%{"id" => queue_id} = params
	
	case session do
	%{"user_id" => user_id} ->
	# Subscribe to updates from the QueueCounter
	:ok = Phoenix.PubSub.subscribe(Hello.PubSub, "queue_counter:updates:#{queue_id}")
	
	# Get the current number of cases handled from QueueCounter
	current_number = QueueCounter.get_current_number(queue_id)
	
	queue = Repo.get!(Queue, queue_id)
	
	socket =
	socket
	|> assign(:current_number, current_number)
	|> assign(:queue_id, queue_id)
	|> assign(:user_id, user_id)
	|> assign(:queue, queue)
	
	{:ok, socket, layout: {HelloWeb.Layouts, :live}}
	
	_ ->
	socket =
	socket
	|> put_flash(:error, "You haven't signed in.")
	
	{:ok, push_navigate(socket, to: "/")}
	end
	end
	
	def handle_event("increment_number", _params, socket) do
	queue_id = socket.assigns.queue_id
	old_queue = Repo.get!(Queue, queue_id)
	new_state = %{"current_number" => old_queue.current_number + 1}
	changeset = Queue.changeset(old_queue, new_state)
	Repo.update(changeset)
	
	current_queue = Repo.get!(Queue, queue_id)
	
	# Increment the current number
	new_number = current_queue.current_number
	
	# Update QueueCounter with the new number
	QueueCounter.set_current_number(queue_id, new_number)
	
	# Update the socket state with the new number
	socket =
	socket
	|> assign(:current_number, new_number)
	|> assign(:queue_id, queue_id)
	{:noreply,socket}
	end
	
	def handle_info({:number_update, queue_id, new_number}, socket) do
	# Update the socket state when a new number is broadcasted
	socket =
	socket
	|> assign(:current_number, new_number)
	|> assign(:queue_id, queue_id)
	{:noreply,socket}
	end
	end
\end{lstlisting}

\subsection{Penjelasan Detail}

\textbf{1. Tujuan Modul.} Modul ini memungkinkan pengguna untuk melihat dan memodifikasi nomor antrian secara real-time. Modul ini menggunakan PubSub untuk menerima pembaruan nomor antrian dan menyediakan event untuk mengubah nilai antrian.

\textbf{2. Fungsi \texttt{mount/3}.} Fungsi ini dijalankan ketika LiveView dimuat pertama kali. Berikut adalah langkah-langkah utamanya:
\begin{enumerate}
	\item \textbf{Mengambil Parameter dan Otentikasi.} Parameter \texttt{params} digunakan untuk mendapatkan \texttt{queue\_id}, dan \texttt{session} digunakan untuk memverifikasi keberadaan \texttt{user\_id}.
	\item \textbf{Berlangganan ke PubSub.} Jika pengguna sudah login, LiveView berlangganan ke topik \texttt{"queue\_counter:updates:\{queue\_id\}"} untuk menerima pembaruan nomor antrian.
	\item \textbf{Mengambil Data Awal.} Data awal, termasuk nomor antrian dan informasi tambahan (\texttt{queue\_name} dan \texttt{queue\_prefix}), diambil dari database menggunakan modul \texttt{Repo}.
	\item \textbf{Otentikasi Gagal.} Jika pengguna belum login, pesan error ditampilkan menggunakan \texttt{put\_flash}, dan pengguna diarahkan ke halaman login.
\end{enumerate}

\textbf{3. Fungsi \texttt{handle\_event/3}.} Fungsi ini menangani event \texttt{"increment\_number"}, yang dipicu ketika pengguna meningkatkan nomor antrian. Berikut langkah-langkahnya:
\begin{enumerate}
	\item \textbf{Mengambil Data Saat Ini.} Data antrian diambil dari database, dan nomor antrian saat ini ditingkatkan sebesar satu.
	\item \textbf{Memperbarui Database.} Perubahan nomor antrian diperbarui ke database menggunakan \texttt{Repo.update/1}.
	\item \textbf{Memperbarui QueueCounter.} Nomor antrian yang baru dikirim ke \texttt{QueueCounter} untuk disiarkan ke subscriber.
	\item \textbf{Memperbarui State Socket.} Nilai \texttt{current\_number} pada \texttt{socket} diperbarui, sehingga perubahan langsung terlihat di tampilan pengguna.
\end{enumerate}

\textbf{4. Fungsi \texttt{handle\_info/2}.} Fungsi ini menangani pesan broadcast dari PubSub. Ketika nomor antrian diperbarui oleh komponen lain, nilai \texttt{current\_number} di \texttt{socket} diperbarui untuk mencerminkan perubahan tersebut.

\subsection{Poin Penting}

\begin{enumerate}
	\item \textbf{Otentikasi Pengguna.} Modul ini memastikan bahwa hanya pengguna yang telah login dapat memodifikasi nomor antrian.
	\item \textbf{Integrasi PubSub.} Sistem PubSub memastikan bahwa semua subscriber menerima pembaruan nomor antrian secara real-time.
	\item \textbf{Modifikasi Data.} Fungsi \texttt{handle\_event/3} memungkinkan pengguna untuk memodifikasi data secara langsung dan aman.
	\item \textbf{Responsivitas Tinggi.} Perubahan data langsung dipantulkan ke antarmuka pengguna tanpa memerlukan refresh halaman.
\end{enumerate}

\subsection{Contoh Penggunaan}

Berikut adalah contoh alur kerja LiveView ini:
\begin{enumerate}
	\item Ketika halaman LiveView dimuat, pengguna harus login untuk mengakses fitur modifikasi antrian.
	\item Jika pengguna meningkatkan nomor antrian melalui tombol di antarmuka, event \texttt{"increment\_number"} dipicu, dan nomor antrian diperbarui.
	\item Semua subscriber yang berlangganan ke topik terkait menerima pembaruan secara real-time.
\end{enumerate}

