\chapter{Dokumentasi dan Unit Test pada Elixir}


\section{Dokumentasi di Elixir}

Elixir mendukung dokumentasi yang mudah untuk modul dan fungsi. Dokumentasi ini dapat ditambahkan langsung dalam kode menggunakan komentar khusus yang disebut dengan \texttt{@moduledoc} untuk modul dan \texttt{@doc} untuk fungsi. Selain itu, Elixir juga menyediakan cara mudah untuk menggenerate dokumentasi dalam bentuk HTML menggunakan perintah \texttt{mix docs}.


\subsection{Dokumentasi pada Level Modul}

Untuk mendokumentasikan sebuah modul, digunakan anotasi \texttt{@moduledoc}. Ini adalah tempat yang tepat untuk memberikan deskripsi tentang tujuan modul, bagaimana modul tersebut digunakan, serta contoh-contoh jika diperlukan.

\begin{lstlisting}[language=Elixir]
	defmodule ExampleModule do
	@moduledoc """
	This module is an example of how to document a module in Elixir.
	
	It serves to demonstrate how module documentation can be written
	and used as a reference when generating HTML documentation.
	"""
	
	# Other functions and logic can be written here
	end
\end{lstlisting}

\subsection{Dokumentasi pada Level Fungsi}

Selain mendokumentasikan modul, Anda juga bisa mendokumentasikan setiap fungsi dengan menggunakan anotasi \texttt{@doc}. Dokumentasi fungsi biasanya memberikan penjelasan tentang apa yang dilakukan fungsi tersebut, parameter yang dibutuhkan, dan hasil yang dikembalikan.

\begin{lstlisting}[language=Elixir]
	defmodule ExampleModule do
	@moduledoc """
	This module is an example of how to document a module in Elixir.
	"""
	
	@doc """
	This function adds two numbers.
	
	## Parameters
	- a: the first number.
	- b: the second number.
	
	## Example
	
	iex> ExampleModule.add(2, 3)
	5
	"""
	def add(a, b) do
	a + b
	end
	end
\end{lstlisting}

\section{Menambahkan \texttt{ex\_doc} untuk Dokumentasi}

Untuk menghasilkan dokumentasi dalam proyek Elixir, Anda perlu menambahkan library \texttt{ex\_doc}. Library ini memungkinkan Anda untuk menghasilkan dokumentasi dalam berbagai format, termasuk HTML. Berikut adalah langkah-langkah untuk menambahkan \texttt{ex\_doc} ke dalam proyek Elixir Anda.

\subsection{Langkah-langkah Menambahkan \texttt{ex\_doc}}

\begin{enumerate}
	\item Buka file \texttt{mix.exs} di proyek Anda, dan tambahkan \texttt{ex\_doc} ke dalam daftar dependensi di fungsi \texttt{deps}. Pastikan Anda menambahkan dependensi ini hanya untuk lingkungan pengembangan (\texttt{dev}) dan tidak di runtime.
	
	\begin{lstlisting}[language=Elixir]
		defp deps do
		[
			{:ex_doc, "~> 0.34"}
		]
		end
	\end{lstlisting}
	
	\item Setelah menambahkan \texttt{ex\_doc}, jalankan perintah berikut di terminal untuk mengunduh dan menginstal dependensi:
	
	\begin{lstlisting}[language=Bash]
		mix deps.get
	\end{lstlisting}
	
	\item Setelah dependensi berhasil diinstal, Anda dapat menghasilkan dokumentasi untuk proyek Anda dengan perintah berikut:
	
	\begin{lstlisting}[language=Bash]
		mix docs
	\end{lstlisting}
	
	\item Dokumentasi yang dihasilkan akan disimpan dalam folder \texttt{doc/} di dalam direktori proyek Anda. Anda dapat membuka file \texttt{index.html} di browser untuk melihat dokumentasi dalam format HTML.
\end{enumerate}

\section{Mengenerate Dokumentasi HTML}

Untuk menggenerate dokumentasi dalam format HTML, Elixir menyediakan perintah sederhana, yaitu \texttt{mix docs}. Perintah ini akan mencari dokumentasi di dalam modul dan fungsi yang ada, lalu mengubahnya menjadi halaman HTML yang mudah dibaca. Langkah-langkahnya adalah sebagai berikut:

\begin{enumerate}
	\item Pastikan Anda berada di dalam direktori proyek Elixir.
	\item Jalankan perintah berikut pada \textit{command prompt}:
	\begin{lstlisting}[language=Bash]
mix docs
	\end{lstlisting}
	
	\item Dokumentasi HTML akan dihasilkan di dalam folder \texttt{doc}.
\end{enumerate}

Dengan mendokumentasikan kode dengan baik dan mengenerate dokumentasi HTML, Anda dapat mempermudah orang lain (atau diri sendiri di masa depan) untuk memahami kode yang Anda tulis.



\section{Unit Test di Elixir}

Unit testing di Elixir dapat dilakukan menggunakan modul bawaan bernama \texttt{ExUnit}. Elixir menyediakan fitur untuk membuat test dalam file khusus serta memungkinkan test ditulis dalam dokumentasi fungsi menggunakan anotasi \texttt{doctest}. Unit test sangat penting untuk memastikan bahwa fungsi-fungsi dalam kode bekerja sesuai harapan.



\subsection{Kesimpulan}

Dengan menambahkan \texttt{ex\_doc}, Anda dapat dengan mudah menghasilkan dokumentasi untuk proyek Elixir Anda. Dokumentasi ini membantu memudahkan pemahaman tentang kode yang ditulis dan berfungsi sebagai panduan bagi pengguna lain yang ingin menggunakan atau mengembangkan proyek tersebut.


\subsection{Unit Test dalam File Khusus}

Untuk membuat unit test di Elixir, biasanya digunakan file khusus yang ditempatkan dalam folder \texttt{test}. File test ini harus memiliki akhiran \texttt{\_test.exs}. Di dalam file tersebut, Anda mendefinisikan modul yang menggunakan \texttt{ExUnit.Case} sebagai template.

\begin{lstlisting}[language=Elixir]
	defmodule ExampleModuleTest do
	use ExUnit.Case
	
	test "adds two numbers" do
	assert ExampleModule.add(2, 3) == 5
	end
	
	test "subtracts two numbers" do
	assert ExampleModule.subtract(5, 2) == 3
	end
	end
\end{lstlisting}

\subsection{Unit Test dalam Dokumentasi Fungsi (Doctest)}

Selain menulis test di file khusus, Elixir juga memungkinkan Anda untuk menulis unit test langsung dalam dokumentasi fungsi menggunakan anotasi \texttt{doctest}. Test ini secara otomatis dijalankan ketika Anda menjalankan \texttt{ExUnit}. Contoh doctest dapat ditulis dalam bagian \texttt{@doc} dari suatu fungsi.

\begin{lstlisting}[language=Elixir]
	defmodule ExampleModule do
	@moduledoc """
	This module demonstrates how to document and test functions.
	"""
	
	@doc """
	This function adds two numbers.
	
	## Parameters
	- a: the first number.
	- b: the second number.
	
	## Example
	
	iex> ExampleModule.add(2, 3)
	5
	"""
	def add(a, b) do
	a + b
	end
	
	@doc """
	This function subtracts two numbers.
	
	## Parameters
	- a: the first number.
	- b: the second number.
	
	## Example
	
	iex> ExampleModule.subtract(5, 2)
	3
	"""
	def subtract(a, b) do
	a - b
	end
	end
\end{lstlisting}

Untuk mengaktifkan \texttt{doctest}, Anda perlu menambahkannya dalam modul test:

\begin{lstlisting}[language=Elixir]
	defmodule ExampleModuleTest do
	use ExUnit.Case
	doctest ExampleModule
	end
\end{lstlisting}

\subsection{Menjalankan Test dari Command Prompt}

Elixir memudahkan Anda untuk menjalankan test langsung dari command prompt. Ada beberapa opsi untuk menjalankan test tergantung kebutuhan Anda.

\subsubsection{Menjalankan Semua Test}

Untuk menjalankan semua test yang ada di proyek, cukup gunakan perintah berikut:

\begin{lstlisting}[language=Bash]
	mix test
\end{lstlisting}

Perintah ini akan mengeksekusi semua test yang ada di folder \texttt{test} dan menampilkan hasilnya di terminal.

\subsubsection{Menjalankan Test Spesifik untuk Modul Tertentu}

Jika Anda hanya ingin menjalankan test untuk modul tertentu, Anda dapat menyebutkan nama file test yang bersangkutan. Misalnya, untuk menjalankan test pada \texttt{ExampleModuleTest}, jalankan perintah berikut:

\begin{lstlisting}[language=Bash]
	mix test test/example_module_test.exs
\end{lstlisting}

\subsubsection{Menjalankan Test Spesifik untuk Fungsi Tertentu}

Untuk menjalankan test yang spesifik untuk suatu fungsi dalam modul, Anda dapat menggunakan tag line number. Misalnya, jika test untuk fungsi \texttt{add} ada pada baris 4 dalam file test, Anda bisa menjalankan perintah berikut:

\begin{lstlisting}[language=Bash]
	mix test test/example_module_test.exs:4
\end{lstlisting}

Jika Anda ingin menjalankan test untuk fungsi lain seperti \texttt{subtract} yang ada di baris 8, Anda cukup menjalankan:

\begin{lstlisting}[language=Bash]
	mix test test/example_module_test.exs:8
\end{lstlisting}

Dengan cara ini, hanya test yang ada pada baris tersebut yang akan dijalankan, sehingga memudahkan dalam proses debugging dan pengujian parsial.

Dengan menggunakan \texttt{ExUnit}, Elixir memberikan cara yang mudah dan efisien untuk melakukan unit testing. Anda bisa menulis test baik dalam file khusus maupun langsung dalam dokumentasi fungsi menggunakan \texttt{doctest}. Menjalankan test dari command prompt juga fleksibel, memungkinkan Anda untuk menjalankan seluruh test, test untuk modul tertentu, atau test untuk fungsi spesifik.

\section{Latihan}

Bagian ini berisi beberapa latihan yang dirancang untuk membantu Anda mempraktikkan dokumentasi dan unit testing di Elixir. Setiap latihan akan mencakup pembuatan modul, menulis dokumentasi, dan membuat unit test. 

\subsection{Latihan 1: Menulis Dokumentasi Modul}

Buatlah sebuah modul Elixir bernama \texttt{Calculator} yang memiliki dua fungsi: \texttt{multiply/2} dan \texttt{divide/2}. Tulislah dokumentasi untuk setiap fungsi menggunakan anotasi \texttt{@doc}.

\begin{lstlisting}[language=Elixir]
	defmodule Calculator do
	@moduledoc """
	A module for basic arithmetic operations.
	"""
	
	@doc """
	Multiplies two numbers.
	
	## Parameters
	- x: the first number.
	- y: the second number.
	
	## Example
	
	iex> Calculator.multiply(4, 5)
	20
	"""
	def multiply(x, y) do
	x * y
	end
	
	@doc """
	Divides two numbers.
	
	## Parameters
	- x: the numerator.
	- y: the denominator.
	
	## Example
	
	iex> Calculator.divide(10, 2)
	5
	"""
	def divide(x, y) do
	x / y
	end
	end
\end{lstlisting}

\subsection{Latihan 2: Menulis Unit Test untuk Modul}

Buatlah file test bernama \texttt{calculator\_test.exs} di dalam folder \texttt{test} untuk mengetes fungsi \texttt{multiply/2} dan \texttt{divide/2} dari modul \texttt{Calculator}. Sertakan beberapa skenario pengujian.

\begin{lstlisting}[language=Elixir]
	defmodule CalculatorTest do
	use ExUnit.Case
	doctest Calculator
	
	test "multiplies two numbers" do
	assert Calculator.multiply(4, 5) == 20
	end
	
	test "divides two numbers" do
	assert Calculator.divide(10, 2) == 5
	end
	
	test "divides by zero raises an error" do
	assert_raise ArithmeticError, fn -> Calculator.divide(10, 0) end
	end
	end
\end{lstlisting}

\subsection{Latihan 3: Menjalankan Test untuk Modul}

Jalankan unit test yang telah Anda tulis untuk modul \texttt{Calculator} menggunakan perintah berikut:

\begin{lstlisting}[language=Bash]
	mix test test/calculator_test.exs
\end{lstlisting}

\subsection{Latihan 4: Menjalankan Test untuk Fungsi Tertentu}

Tambahkan sebuah test baru dalam \texttt{calculator\_test.exs} untuk fungsi \texttt{multiply/2} dan jalankan test tersebut menggunakan nomor baris.

\begin{lstlisting}[language=Elixir]
	defmodule CalculatorTest do
	use ExUnit.Case
	doctest Calculator
	
	test "multiplies two numbers" do
	assert Calculator.multiply(4, 5) == 20
	end
	
	# New test added for specific function
	test "checks multiply with negative numbers" do
	assert Calculator.multiply(-4, 5) == -20
	end
	end
\end{lstlisting}

Jika test baru ditambahkan pada baris 6, jalankan perintah berikut:

\begin{lstlisting}[language=Bash]
	mix test test/calculator_test.exs:6
\end{lstlisting}


\section{Soal}

Bagian ini berisi beberapa soal latihan tambahan yang dirancang untuk mempraktikkan dokumentasi dan unit testing di Elixir. Setiap soal merupakan varian dari contoh-contoh latihan sebelumnya.

\subsection{Soal 1: Dokumentasi dan Unit Test untuk Modul \texttt{Statistics}}

Buatlah modul Elixir bernama \texttt{Statistics} yang memiliki dua fungsi: \texttt{mean/1} dan \texttt{median/1}. 

\begin{enumerate}
	\item Dokumentasikan setiap fungsi dengan anotasi \texttt{@doc}. Fungsi \texttt{mean/1} harus menghitung rata-rata dari sebuah list angka, sementara \texttt{median/1} harus menghitung median dari list angka yang sudah diurutkan.
	\item Buatlah file test bernama \texttt{statistics\_test.exs} di dalam folder \texttt{test} untuk mengetes fungsi \texttt{mean/1} dan \texttt{median/1}. Sertakan beberapa skenario pengujian, seperti menghitung rata-rata dan median untuk list dengan berbagai panjang dan nilai.
\end{enumerate}

\subsection{Soal 2: Dokumentasi dan Unit Test untuk Modul \texttt{StringUtils}}

Buatlah modul Elixir bernama \texttt{StringUtils} yang memiliki dua fungsi: \texttt{reverse/1} dan \texttt{capitalize/1}. 

\begin{enumerate}
	\item Dokumentasikan setiap fungsi dengan anotasi \texttt{@doc}. Fungsi \texttt{reverse/1} harus membalikkan string, sementara \texttt{capitalize/1} harus mengubah huruf pertama dari string menjadi huruf kapital.
	\item Buatlah file test bernama \texttt{string\_utils\_test.exs} di dalam folder \texttt{test} untuk mengetes fungsi \texttt{reverse/1} dan \texttt{capitalize/1}. Sertakan beberapa skenario pengujian, seperti membalikkan string dengan berbagai karakter dan mengubah kapitalisasi string dengan berbagai kasus huruf.
\end{enumerate}


