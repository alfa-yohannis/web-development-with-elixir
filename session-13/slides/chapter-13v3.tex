\documentclass[aspectratio=169, table]{beamer}

\usepackage[utf8]{inputenc}
\usepackage{listings} 

\usetheme{Pradita}

\subtitle{IF140303-Web Application Development}

\title{Session-13:\\\LARGE{WebSocket \& PubSubs using\\
		Phoenix's LiveView (part 2)}}
\date[Serial]{\scriptsize {PRU/SPMI/FR-BM-18/0222}}
\author[Pradita]{\small{\textbf{Alfa Yohannis}}}

\lstdefinelanguage{Elixir} {
	keywords={def, defmodule, do, end, for, if, else, true, false},
	basicstyle=\ttfamily\small,
	keywordstyle=\color{blue}\bfseries,
	ndkeywords={@moduledoc, iex, Enum, @doc},
	ndkeywordstyle=\color{purple}\bfseries,
	sensitive=true,
	commentstyle=\color{gray},
	stringstyle=\color{red},
	numbers=left,
	numberstyle=\tiny\color{gray},
	breaklines=true,
	frame=lines,
	backgroundcolor=\color{lightgray!10},
	tabsize=2,
	comment=[l]{\#},
	morecomment=[s]{/*}{*/},
	commentstyle=\color{gray}\ttfamily,
	stringstyle=\color{purple}\ttfamily,
	showstringspaces=false
}

\begin{document}
	
	\frame{\titlepage}
	
	\begin{frame}
		\frametitle{\shortstack{Introduction to\\ Live Comments Section}}
		\begin{itemize}
			\item In this session, we will implement a live comments section for topics.
			\item Users will be able to add comments that instantly appear on all other users' screens.
			\item This will function similarly to a live chat.
			\item We will utilize WebSockets to achieve this functionality.
		\end{itemize}
	\end{frame}
	
	\begin{frame}
		\frametitle{\shortstack{What is a WebSocket?}}
		\begin{itemize}
			\item WebSockets provide a persistent connection between the client and server.
			\item This allows for real-time, two-way communication.
			\item Unlike HTTP, WebSockets maintain an open connection, allowing data to be sent and received without repeatedly establishing connections.
			\item Ideal for applications requiring real-time updates, such as live chats or live notifications.
		\end{itemize}
	\end{frame}
	
	\begin{frame}
		\frametitle{\shortstack{Flow of the\\ Comments Section}}
		\begin{itemize}
			\item **User fills out comment form.**
			\item **Clicks submit button.**
			\item **WebSocket emits event to the server.**
			\item **Server catches the event.**
			\item **Server creates a new comment in the database.**
			\item **Server emits an event with the updated list of comments.**
			\item **All connected clients receive the update and display the new comment.**
		\end{itemize}
	\end{frame}
	
	\begin{frame}
		\frametitle{\shortstack{Updating the TopicController}}
		\begin{itemize}
			\item We need to update the `TopicController` to include a `show` action.
			\item The `show` action retrieves a specific topic by its ID.
			\item It then renders the `show.html` template, passing the retrieved topic to the view.
			\item No need to update the router since we are using `resources "/"` which already includes a route for `show`.
		\end{itemize}
	\end{frame}
	
	\begin{frame}[fragile]
		\frametitle{\shortstack{Code for the Updated\\ TopicController}}
		\begin{lstlisting}[language=Elixir]
			defmodule Discuss.TopicController do
			use Discuss.Web, :controller
			
			alias Discuss.Topic
			
			plug Discuss.Plugs.RequireAuth when action in 
			[:new, :create, :edit, :update, :delete]
		\end{lstlisting}
	\end{frame}
	
	\begin{frame}[fragile]
		\frametitle{\shortstack{Code for the Updated\\ TopicController}}
		\begin{lstlisting}[language=Elixir]
			plug :check_topic_owner when action in 
			[:update, :edit, :delete]
			
			def show(conn, %{"id" => topic_id}) do
			topic = Repo.get!(Topic, topic_id)
			render conn, "show.html", topic: topic
			end
		\end{lstlisting}
	\end{frame}
	
	\begin{frame}[fragile]
		\frametitle{show.html.eex Template}
		\begin{lstlisting}[language=Elixir]
			<%= topic.title %>
		\end{lstlisting}
		\begin{itemize}
			\item This simple template will display the title of the topic.
			\item It will serve as the foundation for our live comments section.
		\end{itemize}
	\end{frame}
	
	\begin{frame}[fragile]
		\frametitle{\shortstack{Making Topics Clickable\\ in index.html.eex}}
		\begin{lstlisting}[language=Elixir]
			<h5>Topics</h5>
			<ul class = "collection">
			<%= for topic <- @topics do%>
			<li class="collection-item"> 
			<%= link topic.title, to: topic_path(@conn, :show, topic) %>
			<%= if @conn.assigns.user.id == topic.user_id do %>
		\end{lstlisting}
	\end{frame}
	
	\begin{frame}[fragile]
		\frametitle{\shortstack{Making Topics Clickable\\ in index.html.eex}}
		\begin{lstlisting}[language=Elixir]
			<div class="right">
			<%= link "Edit", to: topic_path(@conn, :edit, topic) %>
			<%= link "Delete", to: topic_path(@conn, :delete , topic), method: :delete %>
			</div>
			<% end %>
			</li>
			<%end%>
			</ul>
		\end{lstlisting}
		\begin{itemize}
			\item This makes each topic in the list clickable, linking to its `show` page.
		\end{itemize}
	\end{frame}
	
	\begin{frame}[fragile]
		\frametitle{Add Comments Migration}
		\begin{lstlisting}[language=Elixir]
			defmodule Discuss.Repo.Migrations.AddComments do
			use Ecto.Migration
			
			def change do
			create table(:comments) do
			add :content, :string
			add :user_id, references(:users)
			add :topic_id, references(:topics)
			timestamps()
			end
			...
		\end{lstlisting}
		\begin{itemize}
			\item Creates a `comments` table with fields for `content`, `user\_id`, and `topic\_id`.
			\item Establishes relationships between comments , users, and topics.
		\end{itemize}
	\end{frame}
	
	
	\begin{frame}[fragile]
		\frametitle{\shortstack{Creating the Comment Model\\ in comment.ex}}
		\begin{lstlisting}[language=Elixir]
			defmodule Discuss.Comment do
			use Discuss.Web, :model
			
			schema "comments" do
			field :content, :string
			belongs_to :user, Discuss.User
			belongs_to :topic, Discuss.Topic
			timestamps()
			end
		\end{lstlisting}
	\end{frame}
	
	\begin{frame}[fragile]
		\frametitle{\shortstack{Creating the Comment Model\\ in comment.ex}}
		\begin{lstlisting}[language=Elixir]
			def changeset(struct, params \\ %{}) do
			struct
			|> cast(params, [:content])
			|> validate_required([:content])
			end
			end
		\end{lstlisting}
		\begin{itemize}
			\item This model represents the `comments` table.
			\item Includes fields for content, user, and topic relationships.
			\item The `changeset` function ensures that the content is present when creating or updating a comment.
		\end{itemize}
	\end{frame}
	
	\begin{frame}[fragile]
		\frametitle{Updating topic.ex and user.ex Models}
		\begin{lstlisting}[language=Elixir]
			# Add this to both topic.ex and user.ex models
			has_many :comments, Discuss.Comment
		\end{lstlisting}
		\begin{itemize}
			\item Establishes a one-to-many relationship between topics and comments.
			\item Similarly, it connects users to their respective comments.
		\end{itemize}
	\end{frame}
	
	\begin{frame}
		\frametitle{\shortstack{Working with WebSocket:\\ Server and Client Side}}
		\begin{itemize}
			\item We will work on both server-side and client-side to implement WebSockets.
			\item WebSockets in Phoenix are handled using channels.
			\item Channels are an abstraction on top of WebSockets, making it easier to work with real-time features.
			\item They allow for joining, leaving, and broadcasting messages to groups of users.
		\end{itemize}
	\end{frame}
	
	\begin{frame}
		\frametitle{How WebSocket Works in Phoenix}
		\begin{itemize}
			\item **Server-side:** Defines channels that handle different types of messages.
			\item **Client-side:** JavaScript code connects to the server via WebSockets, joins channels, and sends/receives messages.
			\item This setup ensures real-time communication between all connected clients and the server.
		\end{itemize}
		\end{frame}
	
	\begin{frame}
	\frametitle{\shortstack{Creating a Comment Channel}}
	\begin{itemize}
		\item We'll create a new channel to handle comments.
		\item This will manage events like adding new comments and broadcasting them to all connected clients.
		\item The channel will be defined in `comment\_channel.ex`.
	\end{itemize}
	\end{frame}
	
	\begin{frame}[fragile]
		\frametitle{comment\_channel.ex}
		\begin{lstlisting}[language=Elixir]
			defmodule Discuss.CommentChannel do
			use Discuss.Web, :channel
			alias Discuss.{Repo, Comment}
			
			def join("comments:" <> topic_id, _params, socket) do
			topic_id = String.to_integer(topic_id)
			comments = Repo.all(
			from c in Comment,
			where: c.topic_id == ^topic_id,
			order_by: [desc: c.inserted_at],
			limit: 100,
			preload: [:user]
			)
		\end{lstlisting}
	\end{frame}
	
	\begin{frame}[fragile]
		\frametitle{comment\_channel.ex}
		\begin{lstlisting}[language=Elixir]
			{:ok, %{comments: comments}, assign(socket, :topic_id, topic_id)}
			end
			
			def handle_in("new_comment", %{"content" => content}, socket) do
			topic_id = socket.assigns.topic_id
			user_id = socket.assigns.user_id
			changeset = Comment.changeset(%Comment{topic_id: topic_id, user_id: user_id}, %{"content" => content})
		\end{lstlisting}
	\end{frame}
	
	\begin{frame}[fragile]
		\frametitle{comment\_channel.ex}
		\begin{lstlisting}[language=Elixir]
			case Repo.insert(changeset) do
			{:ok, comment} ->
			broadcast!(socket, "new_comment", %{comment: comment})
			{:reply, :ok, socket}
			{:error, _reason} ->
			{:reply, :error, socket}
			end
			end
			end
		\end{lstlisting}
		\begin{itemize}
			\item Defines a `join/3` function for users to connect to a specific topic's comments channel.
			\item The `handle\_in/3` function handles the `"new\_comment"` event, adding the comment to the database and broadcasting it to all clients.
		\end{itemize}
	\end{frame}
	
	\begin{frame}[fragile]
	\frametitle{\shortstack{Updating the Socket to\\ Include the Channel}}
	\begin{lstlisting}[language=Elixir]
		defmodule Discuss.UserSocket do
		use Phoenix.Socket
		channel "comments:*", Discuss.CommentChannel
		transport :websocket, Phoenix.Transports.WebSocket
		
		def connect(_params, socket) do
		{:ok, socket}
		end
	\end{lstlisting}
	\end{frame}
	
	\begin{frame}[fragile]
		\frametitle{\shortstack{Updating the Socket to\\ Include the Channel}}
		\begin{lstlisting}[language=Elixir]
			def id(_socket), do: nil
			end
		\end{lstlisting}
		\begin{itemize}
			\item Adds the `comments:*` channel to the socket.
			\item This ensures that any user connecting to the socket can join the comments channel.
		\end{itemize}
	\end{frame}
	
	\begin{frame}[fragile]
	\frametitle{\shortstack{Client-side: Connecting to\\ the WebSocket}}
	\begin{lstlisting}[language=Elixir]
		import {Socket} from "phoenix"
		
		let socket = new Socket("/socket", {params: {userToken: window.userToken}})
		
		socket.connect()
		
		let channel = socket.channel("comments:" + topicId, {})
		channel.join()
		.receive("ok", resp => { console.log("Joined successfully", resp) })
	\end{lstlisting}
	\end{frame}
	
	\begin{frame}[fragile]
		\frametitle{\shortstack{Client-side: Connecting to\\ the WebSocket}}
		\begin{lstlisting}[language=Elixir]
			.receive("error", resp => { console.log("Unable to join", resp) })
			
			channel.on("new_comment", (payload) => {
				let commentContainer = document.querySelector("#comments")
				commentContainer.innerHTML += `<p>${payload.comment.content}</p>`
			})
		\end{lstlisting}
		\begin{itemize}
			\item Establishes a connection to the WebSocket and joins the comments channel.
			\item Listens for the `"new\_comment"` event and updates the DOM with the new comment.
		\end{itemize}
	\end{frame}
	
	\begin{frame}[fragile]
	\frametitle{\shortstack{Adding a Comment Form\\ in show.html.eex}}
	\begin{lstlisting}[language=Elixir]
		<div id="comments">
		<%= for comment <- @comments do %>
		<p><%= comment.content %></p>
		<% end %>
		</div>
		
		<form id="comment-form">
		<input type="text" id="comment-input" placeholder="Add a comment..."/>
	\end{lstlisting}
	\end{frame}
	
	\begin{frame}[fragile]
		\frametitle{\shortstack{Adding a Comment Form\\ in show.html.eex}}
		\begin{lstlisting}[language=Elixir]
			<button type="submit">Submit</button>
			</form>
		\end{lstlisting}
		\begin{itemize}
			\item Renders existing comments and a form for submitting new ones.
			\item The form submits comments to the WebSocket channel.
		\end{itemize}
	\end{frame}
	
	\begin{frame}[fragile]
	\frametitle{\shortstack{Handling Form Submission\\ in the Client}}
	\begin{lstlisting}[language=Elixir]
		let form = document.querySelector("#comment-form")
		let commentInput = document.querySelector("#comment-input")
		
		form.addEventListener("submit", (e) => {
			e.preventDefault()
			
			let payload = {content: commentInput.value}
			channel.push("new_comment", payload)
			.receive("ok", (resp) => { commentInput.value = "" })
		\end{lstlisting}
	\end{frame}
	
	\begin{frame}[fragile]
		\frametitle{\shortstack{Handling Form Submission\\ in the Client}}
		\begin{lstlisting}[language=Elixir]
				.receive("error", (resp) => { console.log("Failed to send message", resp) })
			})
		\end{lstlisting}
		\begin{itemize}
			\item Handles the form submission event.
			\item Sends the comment content to the WebSocket channel.
			\item Clears the input field on success or logs an error on failure.
		\end{itemize}
	\end{frame}
	
	\begin{frame}[fragile]
		\frametitle{\shortstack{Socket.js: Managing WebSocket Connections}}
		\begin{lstlisting}[language=Elixir]
			import {Socket} from "phoenix"
			let socket = new Socket("/socket", {params: {token: window.userToken}})
			
			socket.connect()
			
			let channel = socket.channel("comments:1", {})
			channel.join()
			.receive("ok", resp => {console.log("Joined successfully", resp)})
		\end{lstlisting}
	\end{frame}
	
	\begin{frame}[fragile]
		\frametitle{\shortstack{Socket.js: Managing WebSocket Connections}}
		\begin{lstlisting}[language=Elixir]
			.receive("error", resp => {console.log("Unable to join", resp)})
			
			export default socket
		\end{lstlisting}
		\begin{itemize}
			\item \texttt{Socket.js} sets up the WebSocket connection using Phoenix's JavaScript library.
			\item The \texttt{Socket} instance is created with the user's token for authentication.
			\item The \texttt{channel.join()} method attempts to join a channel (e.g., \texttt{"comments:1"}).
			\item Success or error messages are logged to the console, helping with debugging.
		\end{itemize}
	\end{frame}
	
	\begin{frame}[fragile]
		\frametitle{\shortstack{UserSocket: Handling Channel Connections}}
		\begin{lstlisting}[language=Elixir]
			defmodule Discuss.UserSocket do
			use Phoenix.Socket
			
			channel "comments:*", Discuss.CommentsChannel
			
			transport :websocket, Phoenix.Transport.WebSocket
			
			def connect(_params, socket) do
			{:ok, socket}
			end
		\end{lstlisting}
	\end{frame}
	
	\begin{frame}[fragile]
		\frametitle{\shortstack{UserSocket: Handling Channel Connections}}
		\begin{lstlisting}[language=Elixir]
			def id(_socket), do: nil
			end
		\end{lstlisting}
		\begin{itemize}
			\item The \texttt{UserSocket} module defines how the server handles incoming WebSocket connections.
			\item Channels matching the pattern \texttt{"comments:*"} are routed to \texttt{Discuss.CommentsChannel}.
			\item The \texttt{connect/2} function authenticates and establishes the socket connection.
			\item \texttt{id/1} returns \texttt{nil}, meaning this socket won't be identifiable for targeted broadcasts.
		\end{itemize}
	\end{frame}
	
	\begin{frame}[fragile]
		\frametitle{CommentsChannel: Basic Setup}
		\begin{lstlisting}[language=Elixir]
			defmodule Discuss.CommentsChannel do 
			use Discuss.Web, :channel
			
			def join(name, _params, socket) do
			{:ok, %{hey: "there"}, socket}
			end
		\end{lstlisting}
	\end{frame}
	
	\begin{frame}[fragile]
		\frametitle{CommentsChannel: Basic Setup}
		\begin{lstlisting}[language=Elixir]
			def handle_in() do
			end
			end
		\end{lstlisting}
		\begin{itemize}
			\item \texttt{CommentsChannel} handles messages and events for the comments section.
			\item The \texttt{join/3} function confirms the connection and can send initial data to the client.
			\item The \texttt{handle\_in/3} function will process incoming events from the client.
		\end{itemize}
	\end{frame}
	
	\begin{frame}[fragile]
		\frametitle{Updating \texttt{app.js}}
		\begin{lstlisting}[language=Elixir]
			import "phoenix_html"
			import "./socket"
		\end{lstlisting}
		\begin{itemize}
			\item The \texttt{app.js} file imports necessary dependencies for the front-end, including Phoenix HTML helpers and the custom WebSocket logic from \texttt{socket.js}.
		\end{itemize}
	\end{frame}
	
	\begin{frame}
		\frametitle{WebSocket Join Flow in Phoenix}
		\begin{itemize}
			\item Browser initiates connection to the \texttt{"comments:1"} channel.
			\item Server routes the connection through \texttt{UserSocket} to \texttt{CommentsChannel}.
			\item \texttt{CommentsChannel}'s \texttt{join} function is called, returning a response.
			\item Browser receives the response, triggering success or failure handlers.
		\end{itemize}
	\end{frame}
	
	\begin{frame}[fragile]
		\frametitle{Handling Incoming Events}
		\begin{lstlisting}[language=Elixir]
			defmodule Discuss.CommentsChannel do 
			use Discuss.Web, :channel
			
			def join(name, _params, socket) do
			{:ok, %{hey: "there"}, socket}
			end
		\end{lstlisting}
	\end{frame}
	
	\begin{frame}[fragile]
		\frametitle{Handling Incoming Events}
		\begin{lstlisting}[language=Elixir]
			def handle_in(name, message, socket) do
			{:reply, :ok, socket}
			end
			end
		\end{lstlisting}
		\begin{itemize}
			\item \texttt{handle\_in/3} handles incoming events, such as a user submitting a comment.
			\item It processes the event and sends a reply back to the client, indicating success or failure.
		\end{itemize}
	\end{frame}
	
	\begin{frame}[fragile]
		\frametitle{Integrating WebSocket in Views}
		\begin{lstlisting}[language=Elixir]
			<%= @topic.title %>
			<script>
			document.addEventListener("DOMContentLoaded", function() {
				window.createSocket(<%= @topic.id %>)
			}); 
			</script>
		\end{lstlisting}
		\begin{itemize}
			\item This script in \texttt{show.html.eex} triggers the WebSocket connection when the page loads.
			\item The \texttt{createSocket} function is called with the topic ID, establishing the channel connection.
		\end{itemize}
	\end{frame}
	
	\begin{frame}[fragile]
		\frametitle{Refining \texttt{socket.js}}
		\begin{lstlisting}[language=Elixir]
			import {Socket} from "phoenix"
			let socket = new Socket("/socket", {params: {token: window.userToken}})
			
			socket.connect()
			
			const createSocket = (topicId) => {
				let channel = socket.channel(`comments:${topicId}`, {})
				channel.join()
				.receive("ok", resp => {console.log("Joined successfully", resp)})
				.receive("error", resp => {console.log("Unable to join", resp)})
			}
		\end{lstlisting}
	\end{frame}
	
	\begin{frame}[fragile]
		\frametitle{Refining \texttt{socket.js}}
		\begin{lstlisting}[language=Elixir]
			window.createSocket = createSocket;
		\end{lstlisting}
		\begin{itemize}
			\item The \texttt{createSocket} function is updated to accept a \texttt{topicId}, dynamically connecting to the correct channel.
			\item This makes the WebSocket connection context-aware, depending on the topic being viewed.
		\end{itemize}
	\end{frame}
	
	\begin{frame}[fragile]
		\frametitle{\shortstack{Updating CommentsChannel with \\Topic Handling}}
		\begin{lstlisting}[language=Elixir]
			defmodule Discuss.CommentsChannel do 
			use Discuss.Web, :channel
			alias Discuss.Topic
			
			def join("comments:" <> topic_id, _params, socket) do
			topic_id = String.to_integer(topic_id)
			topic = Repo.get(Topic, topic_id)
			
			{:ok, %{}, socket}
			end
		\end{lstlisting}
	\end{frame}
	
	\begin{frame}[fragile]
		\frametitle{\shortstack{Updating CommentsChannel with \\Topic Handling}}
		\begin{lstlisting}[language=Elixir]
			def handle_in(name, %{"content" => content}, socket) do
			{:reply, :ok, socket}
			end
			end
		\end{lstlisting}
		\begin{itemize}
			\item The \texttt{join/3} function now retrieves the topic from the database based on the ID.
			\item This allows for topic-specific operations within the channel.
			\item \texttt{handle\_in/3} processes incoming messages, such as adding a new comment.
		\end{itemize}
	\end{frame}
	
	\begin{frame}[fragile]
		\frametitle{Enhancing \texttt{show.html.eex}}
		\begin{lstlisting}[language=Elixir]
			<h5> <%= @topic.title %> </h5>
			
			<div class="input-field">
			<textarea class="materialize-textarea"></textarea>
			<button class="btn">Add Comment</button>
			<script>
			document.addEventListener("DOMContentLoaded", function() {
				window.createSocket(<%= @topic.id %>)
			}); 
			</script>
		\end{lstlisting}
	\end{frame}
	
	\begin{frame}[fragile]
		\frametitle{Enhancing \texttt{show.html.eex}}
		\begin{itemize}
			\item The updated HTML structure in \texttt{show.html.eex} includes a form for adding comments.
			\item The \texttt{createSocket} function is triggered when the page loads, setting up the WebSocket connection.
		\end{itemize}
	\end{frame}
	
	\begin{frame}[fragile]
		\frametitle{\shortstack{Updating \texttt{socket.js} for \\Comment Submission}}
		\begin{lstlisting}[language=Elixir]
			import {Socket} from "phoenix"
			let socket = new Socket("/socket", {params: {token: window.userToken}})
			
			socket.connect()
			
			const createSocket = () => {
				let channel = socket.channel(`comments:${topicId}`, {})
				channel.join()
				.receive("ok", resp => {console.log("Joined successfully", resp);})
		\end{lstlisting}
	\end{frame}
	
	\begin{frame}[fragile]
		\frametitle{\shortstack{Updating \texttt{socket.js} for \\Comment Submission}}
		\begin{lstlisting}[language=Elixir]
				.receive("error", resp => {console.log("Unable to join", resp);});
				
				document.querySelector('button').addEventListener('click', () => {
					const content = document.querySelector('textarea').value;
					channel.push('comment:add', {content: content});
				});
			}
			
			window.createSocket = createSocket;
		\end{lstlisting}
	\end{frame}
	
	\begin{frame}[fragile]
		\frametitle{\shortstack{Updating \texttt{socket.js} for \\Comment Submission}}
	\begin{itemize}
		\item The \texttt{createSocket} function is enhanced to handle user interaction.
		\item When the button is clicked, the content of the textarea is pushed to the channel.
		\item This push event triggers a message to be sent to the server, where it will be processed.
	\end{itemize}
	\end{frame}
	
	\begin{frame}[fragile]
		\frametitle{\shortstack{Updating \texttt{CommentsChannel} to\\ Handle Comment Submission}}
		\begin{lstlisting}[language=Elixir]
			defmodule Discuss.CommentsChannel do 
			use Discuss.Web, :channel
			alias Discuss.{Topic, Comment}
			
			def join("comments:" <> topic_id, _params, socket) do
			topic_id = String.to_integer(topic_id)
			topic = Repo.get(Topic, topic)
			
			{:ok, %{}, assign(socket, :topic, topic)}
			end
		\end{lstlisting}
	\end{frame}
	
	\begin{frame}[fragile]
		\frametitle{\shortstack{Updating \texttt{CommentsChannel} to\\ Handle Comment Submission}}
		\begin{lstlisting}[language=Elixir]
			def handle_in(name, %{"content" => content}, socket) do
			topic = socket.assigns.topic
			changeset = topic
			|> build_assoc(:comments)
			|> Comment.changeset(%{content: content})
			
			case Repo.insert(changeset) do
			{:ok, comment} ->
			{:reply, :ok, socket}
		\end{lstlisting}
	\end{frame}
	
	\begin{frame}[fragile]
		\frametitle{\shortstack{Updating \texttt{CommentsChannel} to\\ Handle Comment Submission}}
		\begin{lstlisting}[language=Elixir]
			{:error, _reason} ->
			{:reply, {:error, %{errors: changeset}}, socket}
			end
			end
			end
		\end{lstlisting}
		\begin{itemize}
			\item The \texttt{join/3} function assigns the topic to the socket for future use.
			\item \texttt{handle\_in/3} processes the \texttt{comment:add} event, creating a new comment in the database.
			\item The changeset validates the comment before it is inserted. If successful, a confirmation is sent back to the client.
		\end{itemize}
	\end{frame}
	
	\begin{frame}
		\frametitle{\shortstack{Example Flow: Submitting\\ a Comment}}
		\begin{itemize}
			\item **Browser**: The client app starts with \texttt{topic\_id = 1} and joins the channel \texttt{"comments:1"}.
			\item **Server**: The socket is forwarded to \texttt{CommentsChannel}, which sends back the current list of comments.
			\item **Browser**: The JavaScript app renders the list of comments, displaying them to the user.
		\end{itemize}
	\end{frame}
	
	\begin{frame}[fragile]
		\frametitle{\shortstack{Updating \texttt{CommentsChannel}\\ with Preloading}}
		\begin{lstlisting}[language=Elixir]
			defmodule Discuss.CommentsChannel do 
			use Discuss.Web, :channel
			alias Discuss.{Topic, Comment}
			
			def join("comments:" <> topic_id, _params, socket) do
			topic_id = String.to_integer(topic_id)
			topic = Topic 
			|> Repo.get(topic_id)
			|> Repo.preload(:comments)
		\end{lstlisting}
	\end{frame}
	
	\begin{frame}[fragile]
		\frametitle{\shortstack{Updating \texttt{CommentsChannel}\\ with Preloading}}
		\begin{lstlisting}[language=Elixir]
			{:ok, %{comments: topic.comments}, assign(socket, :topic, topic)}
			end
			
			def handle_in(name, %{"content" => content}, socket) do
			topic = socket.assigns.topic
			changeset = topic
			|> build_assoc(:comments)
			|> Comment.changeset(%{content: content})
		\end{lstlisting}
	\end{frame}
	
	\begin{frame}[fragile]
		\frametitle{\shortstack{Updating \texttt{CommentsChannel}\\ with Preloading}}
		\begin{lstlisting}[language=Elixir]
			case Repo.insert(changeset) do
			{:ok, comment} ->
			{:reply, :ok, socket}
			{:error, _reason} ->
			{:reply, {:error, %{errors: changeset}}, socket}
			end
			end
			end
		\end{lstlisting}
		\begin{itemize}
			\item The \texttt{join/3} function now preloads the comments associated with the topic.
			\item This allows the server to send all existing comments to the client when the channel is joined.
		\end{itemize}
	\end{frame}
	
	\begin{frame}[fragile]
		\frametitle{\shortstack{Updating \texttt{Comment} Model: \\JSON Serialization}}
		\begin{lstlisting}[language=Elixir]
			defmodule Discuss.Comment do
			use Discuss.Web, :model
			
			@derive {Poison.Encoder, only: [:content]}
			
			schema "comments" do
			field :content, :string
			belongs_to :user, Discuss.User
			belongs_to :topic, Discuss.Topic
			timestamps()
			end
		\end{lstlisting}
	\end{frame}
	
	\begin{frame}[fragile]
		\frametitle{\shortstack{Updating \texttt{Comment} Model: \\JSON Serialization}}
		\begin{lstlisting}[language=Elixir]
			def changeset(struct, params \\ %{}) do
			struct
			|> cast(params, [:content])
			|> validate_required([:content])
			end
			end 
		\end{lstlisting}
		\begin{itemize}
			\item The \texttt{Comment} model is updated to automatically encode its data to JSON using \texttt{Poison}.
			\item Only the \texttt{content} field will be included in the JSON output.
			\item This is useful for sending comment data back to the client in a format that can be easily consumed.
		\end{itemize}
	\end{frame}
	
	\begin{frame}[fragile]
		\frametitle{Rendering Comments in \texttt{socket.js}}
		\begin{lstlisting}[language=Elixir]
			import {Socket} from "phoenix"
			let socket = new Socket("/socket", {params: {token: window.userToken}})
			
			socket.connect()
			
			const createSocket = () => {
				let channel = socket.channel(`comments:${topicId}`, {})
				channel.join()
				.receive("ok", resp => {
					renderComments(resp.comments);
				})
		\end{lstlisting}
	\end{frame}
	
	\begin{frame}[fragile]
		\frametitle{Rendering Comments in \texttt{socket.js}}
		\begin{lstlisting}[language=Elixir]
				.receive("error", resp => {console.log("Unable to join", resp);});
				
				document.querySelector('button').addEventListener('click', () => {
					const content = document.querySelector('textarea').value;
					channel.push('comment:add', {content: content});
				});
			};
		\end{lstlisting}
	\end{frame}
	
	\begin{frame}[fragile]
		\frametitle{Rendering Comments in \texttt{socket.js}}
		\begin{lstlisting}[language=Elixir]
			function renderComments(comments){
				const renderedComments = comments.map(comment => {
					return `
					<li class="collection-item">
					${comment.content}
					</li>
					`;
				});
				
				document.querySelector('.collection').innerHTML = renderedComments.join('');
			}
		\end{lstlisting}
	\end{frame}
	
	\begin{frame}[fragile]
		\frametitle{Rendering Comments in \texttt{socket.js}}
		\begin{lstlisting}[language=Elixir]
			window.createSocket = createSocket;
		\end{lstlisting}
		\begin{itemize}
			\item The \texttt{renderComments} function dynamically generates HTML for each comment and inserts it into the DOM.
			\item This allows the client to display comments immediately after joining the channel.
		\end{itemize}
	\end{frame}
	
	\begin{frame}[fragile]
		\frametitle{\shortstack{Updating \texttt{show.html.eex} for\\ Comment Display}}
		\begin{lstlisting}[language=Elixir]
			<h5> <%= @topic.title %> </h5>
			
			<div class="input-field">
			<textarea class="materialize-textarea"></textarea>
			<button class="btn">Add Comment</button>
			<ul class="collection">
			</ul>
			<script>
			document.addEventListener("DOMContentLoaded", function() {
				window.createSocket(<%= @topic.id %>)
			}); 
			</script>
		\end{lstlisting}
	\end{frame}
	
	\begin{frame}[fragile]
		\frametitle{\shortstack{Updating \texttt{show.html.eex} for\\ Comment Display}}
		\begin{itemize}
			\item The HTML template includes a list element to display the comments.
			\item When the page loads, the \texttt{createSocket} function is called, which fetches and renders the comments.
		\end{itemize}
	\end{frame}
	
	\begin{frame}[fragile]
		\frametitle{\shortstack{Broadcasting New Comments\\ in \texttt{CommentsChannel}}}
		\begin{lstlisting}[language=Elixir]
			defmodule Discuss.CommentsChannel do 
			use Discuss.Web, :channel
			alias Discuss.{Topic, Comment}
			
			def join("comments:" <> topic_id, _params, socket) do
			topic_id = String.to_integer(topic_id)
			topic = Topic 
			|> Repo.get(topic_id)
			|> Repo.preload(:comments)
			
			{:ok, %{comments: topic.comments}, assign(socket, :topic, topic)}
			end
		\end{lstlisting}
	\end{frame}
	
	\begin{frame}[fragile]
		\frametitle{\shortstack{Broadcasting New Comments\\ in \texttt{CommentsChannel}}}
		\begin{lstlisting}[language=Elixir]
			def handle_in(name, %{"content" => content}, socket) do
			topic = socket.assigns.topic
			changeset = topic
			|> build_assoc(:comments)
			|> Comment.changeset(%{content: content})
			
			case Repo.insert(changeset) do
			{:ok, comment} ->
			broadcast!(socket, "comments:#{socket.assigns.topic.id}:new", %{comment: comment})
		\end{lstlisting}
	\end{frame}
	
	\begin{frame}[fragile]
		\frametitle{\shortstack{Broadcasting New Comments\\ in \texttt{CommentsChannel}}}
		\begin{lstlisting}[language=Elixir]
			{:reply, :ok, socket} 
			{:error, _reason} ->
			{:reply, {:error, %{errors: changeset}}, socket}
			end
			end
			end
		\end{lstlisting}
		\begin{itemize}
			\item The \texttt{broadcast!/3} function sends the new comment to all clients subscribed to the topic.
			\item This ensures real-time updates of comments across all clients.
		\end{itemize}
	\end{frame}
	
	\begin{frame}[fragile]{\shortstack{Update socket.js - \\Adding Event Listener}}
		\begin{itemize}
			\item We update `socket.js` to include an event listener for new comments.
			\item After joining the channel, we listen for `"comments:\${topicId}:new"` events.
			\item The new comment is rendered on the client side.
		\end{itemize}
		
		\begin{lstlisting}[language=Elixir]
			import {Socket} from "phoenix"
			let socket = new Socket("/socket", {params: {token: windows.userTokern}})
			
			socket.connect()
		\end{lstlisting}
	\end{frame}
	
	\begin{frame}[fragile]{\shortstack{Update socket.js - \\Adding Event Listener}}
		\begin{lstlisting}[language=Elixir]
			const createSocket = () => {
				let channel = socket.channel(`comments:${topicId}`, {})
				channel.join()
				.receive("ok", resp => {
					renderComments(resp.comments);
				})
				.receive("error", resp => {console.log("Unable to join", resp);
				});
				
				channel.on(`comments:${topicId}:new`, renderComment);
		\end{lstlisting}
	\end{frame}
	
	\begin{frame}[fragile]{\shortstack{Update socket.js - \\Adding Event Listener}}
		\begin{lstlisting}[language=Elixir]
			document.querySelector('button').addEventListener('click',() => {\
				const content = document.querySelector('textarea').value;
				channel.push('comment:add', {content: content});
			});
		};
		
		function renderComments(comments){
			const renderedComments = comments.map(comment => {
				return commentTemplate(comment)
			});
	\end{lstlisting}
	\end{frame}
	
	\begin{frame}[fragile]{\shortstack{Update socket.js - \\Adding Event Listener}}
		\begin{lstlisting}[language=Elixir]
			document.querySelector('.collection').innerHTML = renderedComments.join('');
		}
		
		function renderComment(event){
			const renderedComment = commentTemplate(event.comment)
			
			document.querySelector('.collection').innerHTML += renderedComment;
		}
		
		window.createSocket = createSocket;
	\end{lstlisting}
	\end{frame}
	
	\begin{frame}{Authentication with Sockets}
		\begin{itemize}
			\item Overview of the authentication flow:
			\item \textbf{Server}: Generate a unique token and add it to the layout.
			\item \textbf{Browser}: Receives the HTML file and boots up the socket, sending the user token.
			\item \textbf{Server}: Verifies the token and assigns the user to the socket.
		\end{itemize}
	\end{frame}
	
	\begin{frame}[fragile]{\shortstack{Update app.html.eex -\\ Adding Unique Token}}
		\begin{itemize}
			\item We add a script to include the user token if the user is logged in.
			\item The token is generated using `Phoenix.Token.sign`.
		\end{itemize}
		
		\begin{lstlisting}[language=Elixir]
			<head>
			<title>Hello Discuss!</title>
			...
			<link rel="stylesheet" href="https://fonts.googleapis.com/icon?family=Material+Icons">
		\end{lstlisting}
	\end{frame}
	
	\begin{frame}[fragile]{\shortstack{Update app.html.eex -\\ Adding Unique Token}}
		\begin{lstlisting}[language=Elixir]
			<script>
			<%= if @conn.assigns.user do %>
			window.userToken = "<%= Phoenix.Token.sign(Discuss.Endpoint, "key", @conn.assigns.user.id) %>"
			<% end %>
			</script>
			...
			</head>
		\end{lstlisting}
	\end{frame}
	
	\begin{frame}[fragile]{\shortstack{Update user\_socket.ex \\- Using Token for Authentication}}
		\begin{itemize}
			\item The `connect` function verifies the token and assigns the user ID to the socket.
			\item This allows us to authenticate users and maintain user-specific data in the channel.
		\end{itemize}
		
		\begin{lstlisting}[language=Elixir]
			defmodule Discuss.UserSocket do
			use Phoenix.Socket
			
			channel "comments:*", Discuss.CommentsChannel
			
			transport :websocket, Phoenix.Transport.WebSocket
		\end{lstlisting}
	\end{frame}
	
	\begin{frame}[fragile]{\shortstack{Update user\_socket.ex \\- Using Token for Authentication}}
		\begin{lstlisting}[language=Elixir]
			def connect(%{"token" => token}, socket) do
			case Phoenix.Token.verify(socket, "key", token) do
			{:ok, user_id} -> 
			{:ok, assign(socket, :user_id, user_id)}
			{:error, _error} ->
			:error
			end
			end
			
			def id(_socket), do: nil
			end
		\end{lstlisting}
	\end{frame}
	
	\begin{frame}[fragile]{Explanation of build\_assoc}
		\begin{itemize}
			\item `build\_assoc` is used to create associations between records.
			\item Limitation: It can only build one relationship at a time and cannot be called twice.
			\item Example: Assigning a topic to a user by updating `comments\_channel.ex`.
		\end{itemize}
		
		\begin{lstlisting}[language=Elixir]
			defmodule Discuss.CommentsChannel do 
			use Discuss.Web, :channel
			alias Discuss.{Topic, Comment}
			
			def join("comments:" <> topic_id, _params, socket) do
			topic_id = String.to_integer(topic_id)
			topic = Topic 
			|> Repo.get(topic_id)
			|> Repo.preload(:comments)
		\end{lstlisting}
	\end{frame}
	
	\begin{frame}[fragile]{Explanation of build\_assoc}
		\begin{lstlisting}[language=Elixir]
			{:ok, %{comments: topic.comments}, assign(socket, :topic, topic)}
			end
			
			def handle_in(name, %{"content" => content}, socket) do
			topic = socket.assigns.topic
			user_id = socket.assigns.user_id
			
			changeset = topic
			|> build_assoc(:comments, user_id: user_id)
			|> Comment.changeset(%{content: content})
		\end{lstlisting}
	\end{frame}
	
	\begin{frame}[fragile]{Explanation of build\_assoc}
		\begin{lstlisting}[language=Elixir]
			case Repo.insert(changeset) do
			{:ok, comment} ->
			broadcast!(socket, "comments:#{socket.assigns.topic.id}:new", %{comment: comment})
			{:reply, :ok, socket} 
			{:error, _reason} ->
			{:reply, {:error, %{errors: changeset}}, socket}
			end
			end
			end
		\end{lstlisting}
	\end{frame}
\end{document}

