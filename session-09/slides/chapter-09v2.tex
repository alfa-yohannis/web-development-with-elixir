\documentclass[aspectratio=169, table]{beamer}

\usepackage[utf8]{inputenc}
\usepackage{listings} 

\usetheme{Pradita}

\subtitle{IF140303-Web Application Development}

\title{\LARGE{Session-09:\\ \shortstack{Creating Forms and Routing in Phoenix}}
	\vspace{20pt}}
\date[Serial]{\scriptsize {PRU/SPMI/FR-BM-18/0222}}
\author[Pradita]{\small{\textbf{Alfa Yohannis}}}

\lstdefinelanguage{Elixir} {
	keywords={def, defmodule, do, end, for, if, else, true, false},
	basicstyle=\ttfamily\small,
	keywordstyle=\color{blue}\bfseries,
	ndkeywords={@moduledoc, iex, Enum, @doc},
	ndkeywordstyle=\color{purple}\bfseries,
	sensitive=true,
	commentstyle=\color{gray},
	stringstyle=\color{red},
	numbers=left,
	numberstyle=\tiny\color{gray},
	breaklines=true,
	frame=lines,
	backgroundcolor=\color{lightgray!10},
	tabsize=2,
	comment=[l]{\#},
	morecomment=[s]{/*}{*/},
	commentstyle=\color{gray}\ttfamily,
	stringstyle=\color{purple}\ttfamily,
	showstringspaces=false
}

\begin{document}
	
	\frame{\titlepage}
	\begin{frame}
		\frametitle{Introduction to Phoenix Models}
		\begin{itemize}
			\item In Phoenix, models represent data and are linked to a PostgreSQL database.
			\item Models validate data before it is saved to the database.
			\item To create a model, use the \texttt{use Discuss.Web, :model} macro, which includes functions necessary for model creation.
		\end{itemize}
	\end{frame}
	
	\begin{frame}
		\frametitle{\shortstack{Model Schema and \\ Changeset Implementation}}
		\begin{itemize}
			\item To build a functional model, two main components are essential:
			\begin{enumerate}
				\item Schema: Defines how the model relates to the database.
				\item Changeset: Manages data validation and transformations.
			\end{enumerate}
			\item These components ensure that the model accurately represents and validates the data.
		\end{itemize}
	\end{frame}
	
	\begin{frame}[fragile]
		\frametitle{Example: Defining a Schema}
		\begin{lstlisting}[language=Elixir]
			defmodule Discuss.Topic do
			use Discuss.Web, :model
			
			schema "topics" do
			field :title, :string
			end
			end
		\end{lstlisting}
		\begin{itemize}
			\item \texttt{Discuss.Topic}: Defines the model module.
			\item \texttt{use Discuss.Web, :model}: Adds necessary functionalities to the model.
			\item \texttt{schema "topics"}: Maps the model to the "topics" table in the database.
			\item \texttt{field :title, :string}: Defines a string field named \texttt{title}.
		\end{itemize}
	\end{frame}
	
	\begin{frame}
		\frametitle{\shortstack{Introducing Changeset \\ for Validation}}
		\begin{itemize}
			\item A changeset is used to validate and transform data before saving it.
			\item For instance, in our example, users must enter a \texttt{title} for the discussion topic.
			\item The changeset ensures that the \texttt{title} field is present and meets validation rules.
		\end{itemize}
	\end{frame}
	
	\begin{frame}
		\frametitle{Validation in OOP vs Elixir}
		\begin{itemize}
			\item In OOP:
			\begin{enumerate}
				\item Create a class and instantiate an object.
				\item Update the object's values.
				\item Validate the object's state.
				\item If valid, save to the database.
			\end{enumerate}
			\item In Elixir:
			\begin{enumerate}
				\item No classes; data is passed through multiple functions.
				\item Values are transformed and validated functionally.
				\item Default values are set using \texttt{\\} in function parameters.
			\end{enumerate}
		\end{itemize}
	\end{frame}
	
	\begin{frame}[fragile]
		\frametitle{\shortstack{Example: Implementing \\ Changeset in Elixir}}
		\begin{lstlisting}[language=Elixir]
			defmodule Discuss.Topic do
			use Discuss.Web, :model
			
			schema "topics" do
			field :title, :string
			end
		\end{lstlisting}
	\end{frame}
	
	\begin{frame}[fragile]
		\frametitle{\shortstack{Example: Implementing \\ Changeset in Elixir}}
		\begin{lstlisting}[language=Elixir]
			def changeset(struct, params \\ %{}) do
			struct
			|> cast(params, [:title])
			|> validate_required([:title])
			end
			end
		\end{lstlisting}
		\begin{itemize}
			\item \texttt{changeset/2}:
			\begin{enumerate}
				\item Takes a struct and parameters.
				\item Uses \texttt{cast} to extract the relevant fields.
				\item Validates that the \texttt{title} field is present using \texttt{validate\_required}.
			\end{enumerate}
		\end{itemize}
	\end{frame}
	
	\begin{frame}
		\frametitle{Summary}
		\begin{itemize}
			\item Phoenix models are integral for linking to databases and validating data.
			\item The model schema defines the structure of the model and its database relations.
			\item Changesets provide a functional way to validate and transform data.
		\end{itemize}
	\end{frame}
	
	\begin{frame}
		\frametitle{IEx and Understanding Changeset}
		\begin{itemize}
			\item \texttt{Discuss.Web, :model} automatically creates our struct.
			\item \texttt{iex>} allows interactive exploration of changesets.
		\end{itemize}
	\end{frame}
	
	\begin{frame}[fragile]
		\frametitle{Exploring Changesets with IEx}
		\texttt{To simulate how the program works, we will execute some iex command}
		\begin{lstlisting}[language=Elixir]
			iex> struct = %Discuss.Topic{}
			iex> params = %{title: "Great JS"}
			iex> Discuss.Topic.changeset(struct, params)
			%Discuss.Topic{
				__meta__: #Ecto.Schema.Metadata<:built, "topics">,
				id: nil,
				title: "Great JS"
			}
		\end{lstlisting}
	\end{frame}
	
	\begin{frame}[fragile]
		\frametitle{Exploring Changesets with IEx}
		\begin{lstlisting}[language=Elixir]
			iex> Discuss.Topic.changeset(struct, %{})
			{:error, changeset} = 
			%Discuss.Topic{
				__meta__: #Ecto.Schema.Metadata<:built, "topics">,
				id: nil,
				title: nil
			}
		\end{lstlisting}
		\begin{itemize}
			\item Here we create a Discuss.Topic struct and params to input the title
			\item The first command successfully validates and casts the title.
			\item The second command fails due to missing required fields.
		\end{itemize}
	\end{frame}
	
	\begin{frame}[fragile]
		\frametitle{\shortstack{Linking Model to\\TopicController}}
		\begin{itemize}
			\item \texttt{alias Discuss.Topic} allows shorthand for referencing the model.
			\item \texttt{changeset = Topic.changeset(\%Topic\{\}, \%\{\})} prepares a blank form for a new topic.
		\end{itemize}
		\begin{lstlisting}[language=Elixir]
			defmodule Discuss.TopicController do
			use Discuss.Web, :controller
			
			alias Discuss.Topic
			
			def new(conn, _params) do
			changeset = Topic.changeset(%Topic{}, %{})
			...
		\end{lstlisting}
	\end{frame}
	
	\begin{frame}[fragile]
		\frametitle{Creating TopicView and Template}
		\begin{itemize}
			\item The \texttt{TopicView} helps render the view layer.
			\item The \texttt{new.html.eex} file generates the HTML for the new topic form.
		\end{itemize}
		\begin{lstlisting}[language=Elixir]
			defmodule Discuss.TopicView do
			use Discuss.Web, :view
			end
		\end{lstlisting}
		\begin{itemize}
			\item {Now we will write the new.html.eex with:\\}
			\texttt{<h1>New Test Form</h1>}
			\item when we access it \texttt{ localhost:4000/topics/new} it will give error as render function has not been implemented
		\end{itemize}
	\end{frame}
	
	\begin{frame}[fragile]
		\frametitle{\shortstack{Adding Render\\to TopicController}}
		\begin{itemize}
			\item The \texttt{render/3} function is necessary to display the form.
			\item The \texttt{changeset} is passed to the view for rendering.
		\end{itemize}
		\begin{lstlisting}[language=Elixir]
			def new(conn, _params) do
			changeset = Topic.changeset(%Topic{}, %{})
			render conn, "new.html", changeset: changeset
			end
		\end{lstlisting}
	\end{frame}
	
	\begin{frame}[fragile]
		\frametitle{Elixir Helpers for HTML Forms}
		\begin{itemize}
			\item \texttt{form\_for/4}: Builds an HTML form using an Ecto Changeset. It takes a changeset, the form action, and a function to define form elements.
			\item \texttt{text\_input/4}: Creates a text input field within a form. It requires the form builder, field name, and optional attributes like placeholder and class.
			\item \texttt{<\%= \%>}: Inserts Elixir code into an EEx template to generate HTML output.
		\end{itemize}
	\end{frame}
	
	\begin{frame}[fragile]
		\frametitle{\shortstack{Using Elixir Helpers\\in HTML Forms}}
		\begin{lstlisting}[language=Elixir]
			<%= form_for @changeset, topic_path(@conn, :create), fn f -> %>
			<div class="form-group">
			<%= text_input f, :title, placeholder: "Title", class: "form-control" %>
			</div>
			<%= submit "Save Topic", class: "btn btn-primary" %>
			<% end %>
		\end{lstlisting}
		\begin{itemize}
			\item Generates a form and inputs, enhancing readability and maintainability.
		\end{itemize}
	\end{frame}
	
	\begin{frame}
		\frametitle{Updating Router for Post Handling}
		\begin{itemize}
			\item **Problem:** Missing post handler for topic creation.
			\item **Solution:** Update the router.ex file.
			\item **Router Configuration:**
			\begin{itemize}
				\item \texttt{scope "/", Discuss do}: Defines a scope for routes under the root path.
				\item \texttt{pipe\_through :browser}: Applies the `browser` pipeline to all routes in the scope.
				\item \texttt{get "/", PageController, :index}: Maps the root path to the `index` action of `PageController`.
				\item \texttt{get "/topics/new", TopicController, :new}: Maps the `/topics/new` path to the `new` action of `TopicController`.
				\item \texttt{post "/topics", TopicController, :create}: Maps the `/topics` path to the `create` action of `TopicController` for handling form submissions.
			\end{itemize}
		\end{itemize}
	\end{frame}
	
	\begin{frame}[fragile]
		\frametitle{\shortstack{Updating TopicController\\with Create Action}}
		\begin{itemize}
			\item \texttt{create/2} handles form submissions and extracts parameters.
			\item Pattern matching extracts \texttt{topic} from the submitted params.
		\end{itemize}
		\begin{lstlisting}[language=Elixir]
			def create(conn, %{"topic" => topic}) do
			end
		\end{lstlisting}
	\end{frame}
	
	\begin{frame}[fragile]
		\frametitle{Exploring Phoenix Routes with IEx}
		\begin{itemize}
			\item Use \texttt{iex> mix phoenix.routes} to list all routes.
			\item Provides a clear overview of available routes and their corresponding actions.
		\end{itemize}
		\begin{lstlisting}[language=Elixir]
			iex> mix phoenix.routes
			Helper            Path                     Controller
			page_path         GET   /                  PageController :index
			topic_path        GET   /topics/new        TopicController :new
			topic_path        POST  /topics            TopicController :create
		\end{lstlisting}
	\end{frame}
	
	\begin{frame}
		\frametitle{Summary}
		\begin{itemize}
			\item We explored Phoenix models, schemas, and changesets.
			\item We linked models to controllers and created views and templates.
			\item We examined routing, form creation, and handling form submissions.
			\item Understanding these concepts is crucial for building functional web applications with Phoenix.
		\end{itemize}
	\end{frame}
	
\end{document}
