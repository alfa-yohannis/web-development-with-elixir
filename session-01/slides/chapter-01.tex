\documentclass[aspectratio=169, table]{beamer}

\usepackage[utf8]{inputenc}
\usepackage{listings} 

\usetheme{Pradita}

\subtitle{IF140303-Web Application Development}

\title{\LARGE{Session-01:\\Introduction to Elixir}
	\vspace{20pt}}
\date[Serial]{\scriptsize {PRU/SPMI/FR-BM-18/0222}}
\author[Pradita]{\small{\textbf{Alfa Yohannis}}}

\lstdefinelanguage{Elixir} {
	keywords={def, defmodule, do, end, for, if, else, true, false},
	basicstyle=\ttfamily\small,
	keywordstyle=\color{blue}\bfseries,
	ndkeywords={@moduledoc, iex, Enum, @doc},
	ndkeywordstyle=\color{purple}\bfseries,
	sensitive=true,
	commentstyle=\color{gray},
	stringstyle=\color{red},
	numbers=left,
	numberstyle=\tiny\color{gray},
	breaklines=true,
	frame=lines,
	backgroundcolor=\color{lightgray!10},
	tabsize=2,
	comment=[l]{\#},
	morecomment=[s]{/*}{*/},
	commentstyle=\color{gray}\ttfamily,
	stringstyle=\color{purple}\ttfamily,
	showstringspaces=false
}

\begin{document}
	
	\frame{\titlepage}
	
	\begin{frame}
		\frametitle{What is Elixir?}
		\begin{itemize}
			\item Elixir is a dynamic, functional programming language designed for building scalable and maintainable applications.
			\item It runs on the Erlang VM (BEAM), known for its ability to handle large amounts of concurrent processes.
			\item Elixir leverages Erlang's capabilities for distributed systems, fault-tolerance, and low-latency messaging.
			\item Its syntax is inspired by Ruby, making it accessible for developers familiar with Ruby.
		\end{itemize}
	\end{frame}
	
	\begin{frame}
		\frametitle{How to Install Elixir}
		\vspace{20pt}
		\begin{itemize}
			\item \textbf{On macOS:}
			\begin{itemize}
				\item Use Homebrew: \texttt{brew install elixir}
			\end{itemize}
			\item \textbf{On Ubuntu:}
			\begin{itemize}
				\item Install dependencies: \texttt{sudo apt-get install wget gnupg}
				\item Add the Elixir repository:\\\texttt{wget -qO- \url{https://deb.nodesource.com/setup\_16.x} | sudo -E bash -}
				\item Install Elixir: \texttt{sudo apt-get install elixir}
			\end{itemize}
			\item \textbf{On Windows:}
			\begin{itemize}
				\item Download the installer from the Elixir website: \url{https://elixir-lang.org/install.html}
			\end{itemize}
			\item Verify the installation by running: \texttt{elixir --version}
		\end{itemize}
	\end{frame}
	
	\begin{frame}[fragile]
		\frametitle{Module Documentation}
		\begin{lstlisting}[language=Elixir]
			defmodule Lottery do
			@moduledoc """
			This module provides functionalities for managing a lottery system.
			It includes functions for creating, shuffling, checking for numbers, and distributing numbers within the lottery pool.
			"""
		\end{lstlisting}
		\begin{itemize}
			\item \texttt{@moduledoc} provides documentation for the \texttt{Lottery} module.
			\item Describes the module's purpose and the functionalities it offers.
			\item Includes functions for creating, shuffling, checking, and distributing numbers.
		\end{itemize}
	\end{frame}
	
	\begin{frame}[fragile]
		\frametitle{Elixir: Lottery Module}
		\begin{lstlisting}[language=Elixir]
			defmodule Lottery do
			def greet do
			"Good luck!"
			end
		\end{lstlisting}
		\begin{itemize}
			\item The \texttt{Lottery} module handles lottery operations such as creating pools, shuffling, and checking numbers.
		\end{itemize}
	\end{frame}
	
	\begin{frame}[fragile]
		\frametitle{Greeting Function}
		\begin{lstlisting}[language=Elixir]
			def greet do
			"Good luck!"
			end
		\end{lstlisting}
		\begin{itemize}
			\item The \texttt{greet/0} function returns a simple greeting message.
		\end{itemize}
	\end{frame}
	
	\begin{frame}[fragile]
		\frametitle{Generating Lottery Pool}
		\begin{lstlisting}[language=Elixir]
			@spec generate_pool() :: [<<_::24, _::_*16>>, ...]
			def generate_pool do
			numbers = ["Number 1", "Number 2", "Number 3", "Number 4", "Number 5", "Number 6"]
			pots = ["Pot 1", "Pot 2", "Pot 3", "Pot 4"]
			
			for pot <- pots, number <- numbers do
			"#{number} in #{pot}"
			end
			end
		\end{lstlisting}
		\begin{itemize}
			\item \texttt{generate\_pool/0} creates a list of lottery numbers across multiple pots.
			\item The nested \texttt{for} comprehensions combine numbers with pot labels.
		\end{itemize}
	\end{frame}
	
	\begin{frame}[fragile]
		\frametitle{Shuffling the Pool}
		\begin{lstlisting}[language=Elixir]
			def randomize(pool) do
			Enum.shuffle(pool)
			end
		\end{lstlisting}
		\begin{itemize}
			\item \texttt{randomize/1} shuffles the list of lottery numbers.
			\item Utilizes \texttt{Enum.shuffle/1} to randomize the order.
		\end{itemize}
	\end{frame}
	
	\begin{frame}[fragile]
		\frametitle{Checking Number Presence}
		\begin{lstlisting}[language=Elixir]
			@spec contains?(any(), any()) :: boolean()
			def contains?(pool, number) do
			Enum.member?(pool, number)
			end
		\end{lstlisting}
		\begin{itemize}
			\item \texttt{contains?/2} checks if a number is present in the lottery pool.
			\item Uses \texttt{Enum.member?/2} to verify presence.
		\end{itemize}
	\end{frame}
	
	\begin{frame}[fragile]
		\frametitle{Distributing the Pool}
		\begin{lstlisting}[language=Elixir]
			def distribute(pool, draw_size) do
			Enum.split(pool, draw_size)
			end
			end
		\end{lstlisting}
		\begin{itemize}
			\item \texttt{distribute/2} splits the pool into two lists based on the draw size.
			\item Returns a tuple containing two lists.
		\end{itemize}
	\end{frame}
	
	\begin{frame}[fragile]
		\frametitle{Function Documentation (1)}
		\begin{lstlisting}[language=Elixir]
			@doc """
			Splits the pool into two parts based on draw_size.
			
			## Parameters
			
			- pool: List of lottery numbers.
			- draw_size: Number of items in the first part.
			
			## Returns
			
			- A tuple with two lists: the first with draw_size items, and the second with the rest.
			
			## Example
			
			iex> Lottery.distribute(["Number 1 in Pot 1", "Number 2 in Pot 2"], 1)
			{["Number 1 in Pot 1"], ["Number 2 in Pot 2"]}
			"""
		\end{lstlisting}
	\end{frame}

	\begin{frame}[fragile]
		\frametitle{Function Documentation (2)}
		\begin{itemize}
			\item \texttt{@doc} provides documentation for the \texttt{distribute/2} function.
			\item Describes the parameters: \texttt{pool} and \texttt{draw\_size}.
			\item Details the return value: a tuple with two lists.
			\item Provides an example usage.
		\end{itemize}
	\end{frame}

	\begin{frame}
		\frametitle{Function Name/Number}
		\begin{itemize}
			\item In Elixir, \texttt{function name/number} refers to the arity of a function.
			\item The number after the slash (/) indicates the number of arguments the function takes.
			\item For example:
			\begin{itemize}
				\item \texttt{greet/0} has 0 arguments.
				\item \texttt{generate\_pool/0} has 0 arguments.
				\item \texttt{contains?/2} has 2 arguments.
				\item \texttt{distribute/2} has 2 arguments.
			\end{itemize}
			\item This notation is useful for distinguishing between different functions with the same name but different arity.
		\end{itemize}
	\end{frame}


\begin{frame}
	\frametitle{What is Functional Programming?}
	\begin{itemize}
		\item Functional programming is a paradigm that treats computation as the evaluation of mathematical functions.
		\item It avoids changing-state and mutable data.
		\item Functions are first-class citizens, meaning they can be passed as arguments, returned from other functions, and assigned to variables.
		\item It emphasizes the use of pure functions, which have no side effects and always produce the same output for the same input.
	\end{itemize}
\end{frame}

\begin{frame}
	\frametitle{Key Concepts of Functional Programming (1)}
	\begin{itemize}
		\item \textbf{Immutability:} Data cannot be modified after it is created. Instead, new data structures are created.
		\item \textbf{Pure Functions:} Functions that do not cause side effects and return the same result for the same inputs.
		\item \textbf{Higher-Order Functions:} Functions that can take other functions as arguments or return them as results.
		\item \textbf{First-Class Functions:} Functions are treated as first-class citizens, allowing them to be assigned to variables, passed as arguments, and returned from other functions.
	\end{itemize}
\end{frame}

\begin{frame}
	\frametitle{Key Concepts of Functional Programming (2)}
	\begin{itemize}
		\item \textbf{Declarative Style:} Focuses on what to compute rather than how to compute it, emphasizing expressions and declarations over statements.
		\item \textbf{Recursion:} Functional programming often uses recursion as the primary mechanism for iteration, avoiding traditional loops.
	\end{itemize}
\end{frame}

\end{document}
