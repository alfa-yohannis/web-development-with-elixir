\documentclass[aspectratio=169, table]{beamer}

\usepackage[utf8]{inputenc}
\usepackage{listings} 

\usetheme{Pradita}

\subtitle{IF140303-Web Application Development}

\title{\LARGE{Session-02:\\Pattern Matching in Elixir}
	\vspace{20pt}}
\date[Serial]{\scriptsize {PRU/SPMI/FR-BM-18/0222}}
\author[Pradita]{\small{\textbf{Alfa Yohannis}}}

\lstdefinelanguage{Elixir} {
	keywords={def, defmodule, do, end, for, if, else, true, false},
	basicstyle=\ttfamily\small,
	keywordstyle=\color{blue}\bfseries,
	ndkeywords={@moduledoc, iex, Enum, @doc},
	ndkeywordstyle=\color{purple}\bfseries,
	sensitive=true,
	commentstyle=\color{gray},
	stringstyle=\color{red},
	numbers=left,
	numberstyle=\tiny\color{gray},
	breaklines=true,
	frame=lines,
	backgroundcolor=\color{lightgray!10},
	tabsize=2,
	comment=[l]{\#},
	morecomment=[s]{/*}{*/},
	commentstyle=\color{gray}\ttfamily,
	stringstyle=\color{purple}\ttfamily,
	showstringspaces=false
}

\begin{document}
	
	\frame{\titlepage}
	
\begin{frame}[fragile]
	\frametitle{Module Definition}
	\begin{lstlisting}[language=Elixir]
		defmodule Cards do
		def hello do
		"hi there!"
		end
	\end{lstlisting}
	\begin{itemize}
		\item Defines the `Cards` module.
		\item Contains a function named `hello` that returns a simple greeting message.
	\end{itemize}
\end{frame}

\begin{frame}[fragile]
	\frametitle{Creating a Deck}
	\begin{lstlisting}[language=Elixir]
		@spec create_deck() :: [<<_::24, _::_*16>>, ...]
		def create_deck do
		values = ["Ace", "Two", "Three", "Four", "Five", "Six"]
		suits = ["Spades", "Clubs", "Hearts", "Diamond"]
		
		for suit <- suits, value <- values do
		"#{value} of #{suit}"
		end
		end
	\end{lstlisting}
	\begin{itemize}
		\item The `create\_deck` function generates a list of cards.
		\item Combines card values with suits to form a full deck.
		\item Uses a nested `for` comprehension to produce the card combinations.
	\end{itemize}
\end{frame}

\begin{frame}[fragile]
	\frametitle{Shuffling the Deck}
	\begin{lstlisting}[language=Elixir]
		def shuffle(deck) do
		Enum.shuffle(deck)
		end
	\end{lstlisting}
	\begin{itemize}
		\item The `shuffle` function randomizes the order of cards in the deck.
		\item Uses `Enum.shuffle/1` to shuffle the list.
	\end{itemize}
\end{frame}

\begin{frame}[fragile]
	\frametitle{Checking for a Card}
	\begin{lstlisting}[language=Elixir]
		@spec contains?(any(), any()) :: boolean()
		def contains?(deck, card) do
		Enum.member?(deck, card)
		end
	\end{lstlisting}
	\begin{itemize}
		\item The `contains?/2` function checks if a specific card is present in the deck.
		\item Utilizes `Enum.member?/2` to determine the presence of the card.
	\end{itemize}
\end{frame}

\begin{frame}[fragile]
	\frametitle{Dealing Cards}
	\begin{lstlisting}[language=Elixir]
		def deal(deck, hand_size) do
		Enum.split(deck, hand_size)
		end
	\end{lstlisting}
	\begin{itemize}
		\item The `deal/2` function splits the deck into two parts based on the hand size.
		\item Uses `Enum.split/2` to divide the deck.
		\item Returns a tuple with two lists: one for the hand and one for the remaining cards.
	\end{itemize}
\end{frame}

\end{document}