\documentclass[aspectratio=169, table]{beamer}

\usepackage[utf8]{inputenc}
\usepackage{listings} 

\usetheme{Pradita}

\subtitle{IF140303-Web Application Development}

\title{\LARGE{Session-02:\\ Elixir's Pattern Matching}
	\vspace{15pt}}
\date[Serial]{\scriptsize {PRU/SPMI/FR-BM-18/0222}}
\author[Pradita]{\small{\textbf{Alfa Yohannis}}}

\lstdefinelanguage{Elixir} {
	keywords={def, defmodule, do, end, for, if, else, true, false},
	basicstyle=\ttfamily\small,
	keywordstyle=\color{blue}\bfseries,
	ndkeywords={@moduledoc, iex, Enum, @doc},
	ndkeywordstyle=\color{purple}\bfseries,
	sensitive=true,
	commentstyle=\color{gray},
	stringstyle=\color{red},
	numbers=left,
	numberstyle=\tiny\color{gray},
	breaklines=true,
	frame=lines,
	backgroundcolor=\color{lightgray!10},
	tabsize=2,
	comment=[l]{\#},
	morecomment=[s]{/*}{*/},
	commentstyle=\color{gray}\ttfamily,
	stringstyle=\color{purple}\ttfamily,
	showstringspaces=false
}

\begin{document}
	
	\frame{\titlepage}
	
	\begin{frame}
		\frametitle{Recap: Function Documentation and Arity}
		\begin{itemize}
			\item In Elixir, the function name followed by a slash and number indicates the arity, or the number of arguments the function takes.
			\item \texttt{function name/number} notation helps in distinguishing between different functions with the same name but different arities.
			\item Example: \texttt{greet/0}, \texttt{generate\_pool/0}, \texttt{contains?/2}, \texttt{distribute/2}.
		\end{itemize}
	\end{frame}
	
	\begin{frame}
		\frametitle{Introduction to Advanced Concepts}
		\begin{itemize}
			\item This session covers more advanced Elixir concepts within the context of a Lottery module.
			\item We’ll explore how to save and load data, and how to create more complex operations like creating a randomized hand from the lottery pool.
		\end{itemize}
	\end{frame}
	
	\begin{frame}[fragile]
		\frametitle{Saving the Lottery Pool to a File}
		\begin{lstlisting}[language=Elixir]
			def save_pool(pool, filename) do
			binary = :erlang.term_to_binary(pool)
			File.write(filename, binary)
			end
		\end{lstlisting}
		\begin{itemize}
			\item \texttt{save\_pool/2} saves the lottery pool to a specified file.
			\item Converts the pool to binary using \texttt{:erlang.term\_to\_binary/1}.
			\item Writes the binary data to the file with \texttt{File.write/2}.
		\end{itemize}
	\end{frame}
	
	\begin{frame}[fragile]
		\frametitle{Loading the Lottery Pool from a File}
		\begin{lstlisting}[language=Elixir]
			def load_pool(filename) do
			case File.read(filename) do
			{:ok, binary} -> :erlang.binary_to_term(binary)
			{:error, _reason} -> "That file does not exist"
			end
			end
		\end{lstlisting}
		\begin{itemize}
			\item \texttt{load\_pool/1} loads the lottery pool from a specified file.
			\item Reads the binary data from the file using \texttt{File.read/1}.
			\item Converts the binary data back into the original list with \texttt{:erlang.binary\_to\_term/1}.
		\end{itemize}
	\end{frame}
	
	\begin{frame}[fragile]
		\frametitle{Creating a Randomized Hand}
		\begin{lstlisting}[language=Elixir]
			def create_hand(draw_size) do
			Lottery.generate_pool()
			|> Lottery.randomize()
			|> Lottery.distribute(draw_size)
			end
		\end{lstlisting}
		\begin{itemize}
			\item \texttt{create\_hand/1} generates a randomized hand from the lottery pool.
			\item Combines multiple operations: generating the pool, shuffling, and distributing.
			\item Utilizes the pipe operator (\texttt{|>}) for chaining functions together.
		\end{itemize}
	\end{frame}
	
	\begin{frame}
		\frametitle{The Pipe Operator (\texttt{|>})}
		\begin{itemize}
			\item The pipe operator allows for chaining function calls in a clear and readable manner.
			\item It passes the result of the expression on the left as the first argument to the function on the right.
			\item Enhances code readability, especially when working with multiple transformations or operations.
		\end{itemize}
	\end{frame}
	
	\begin{frame}
		\frametitle{Recap: Functional Programming Concepts}
		\begin{itemize}
			\item Functional programming emphasizes immutability, pure functions, and higher-order functions.
			\item Functions can be passed as arguments, returned as results, and assigned to variables.
			\item Recursion is preferred over loops for iteration.
		\end{itemize}
	\end{frame}
	
	\begin{frame}
		\frametitle{Summary}
		\begin{itemize}
			\item We explored advanced Elixir concepts by continuing the development of the \texttt{Lottery} module.
			\item We learned how to save and load data, create a randomized hand, and utilize the pipe operator.
			\item These concepts further enhance our understanding of functional programming and Elixir's capabilities.
		\end{itemize}
	\end{frame}
	
\end{document}
