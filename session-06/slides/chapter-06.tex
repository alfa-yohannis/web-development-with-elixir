\documentclass[aspectratio=169, table]{beamer}

\usepackage[utf8]{inputenc}
\usepackage{listings}

\usetheme{Pradita}

\subtitle{IF140303-Web Application Development}

\title{\LARGE{Session-06:\\ Structs and Avatar Generator}}
\date[Serial]{\scriptsize {PRU/SPMI/FR-BM-18/0222}}
\author[Pradita]{\small{\textbf{Alfa Yohannis}}}

\lstdefinelanguage{Elixir} {
	keywords={def, defmodule, do, end, for, if, else, true, false},
	basicstyle=\ttfamily\small,
	keywordstyle=\color{blue}\bfseries,
	ndkeywords={@moduledoc, iex, Enum, @doc},
	ndkeywordstyle=\color{purple}\bfseries,
	sensitive=true,
	commentstyle=\color{gray},
	stringstyle=\color{red},
	numbers=left,
	numberstyle=\tiny\color{gray},
	breaklines=true,
	frame=lines,
	backgroundcolor=\color{lightgray!10},
	tabsize=2,
	comment=[l]{\#},
	morecomment=[s]{/*}{*/},
	commentstyle=\color{gray}\ttfamily,
	stringstyle=\color{purple}\ttfamily,
	showstringspaces=false
}

\begin{document}
	
	\frame{\titlepage}
	
	\begin{frame}
		\frametitle{Introduction to Structs in Elixir}
		\begin{itemize}
			\item Structs in Elixir are special maps with a defined set of keys and default values.
			\item Unlike regular maps, structs enforce the presence of specific keys, providing more structure and clarity in your code.
			\item Example: Creating a struct in the \texttt{Avatar.Image} module.
		\end{itemize}
	\end{frame}
	
	\begin{frame}[fragile]
		\frametitle{Defining a Struct}
		\begin{lstlisting}[language=Elixir]
			defmodule Avatar.Image do
			defstruct hash: nil, color: nil, grid: nil, pixel_map: nil
			end
		\end{lstlisting}
		\begin{itemize}
			\item The \texttt{Avatar.Image} struct has predefined keys: \texttt{hash}, \texttt{color}, \texttt{grid}, and \texttt{pixel\_map}.
			\item This ensures that only these keys can be used, enhancing data consistency.
		\end{itemize}
	\end{frame}
	
	\begin{frame}[fragile]
		\frametitle{Using Structs in IEx}
		\begin{lstlisting}[language=Elixir]
			iex> %Avatar.Image{}
			%Avatar.Image{color: nil, grid: nil, hash: nil, pixel_map: nil}
		\end{lstlisting}
		\begin{itemize}
			\item When initialized, the struct fields are set to their default values.
			\item Attempting to add keys not defined in the struct will result in an error, ensuring only the expected fields are used.
		\end{itemize}
	\end{frame}
	
	\begin{frame}
		\frametitle{Overview of the Avatar Generator}
		\begin{itemize}
			\item The \texttt{AvatarGenerator} module generates a unique avatar based on an input string.
			\item This lesson will walk through the key steps in the avatar generation process, focusing on the code after hashing.
		\end{itemize}
	\end{frame}
	
	\begin{frame}[fragile]
		\frametitle{Main Function: Generating the Avatar}
		\begin{lstlisting}[language=Elixir]
			def generate(input) do
			input
			|> compute_hash
			|> select_color 
			|> create_grid
			|> remove_odd_cells
			|> generate_pixel_map
			|> render_image
			|> store_image(input)
			end
		\end{lstlisting}
		\begin{itemize}
			\item The \texttt{generate/1} function is the main entry point, taking an input string and transforming it into an avatar.
			\item The process is broken down into distinct steps, chained together using the pipe operator (\texttt{|>}).
		\end{itemize}
	\end{frame}
	
	\begin{frame}[fragile]
		\frametitle{Step 1: Selecting a Color}
		\begin{lstlisting}[language=Elixir]
			def select_color(%Avatar.Image{hash: [r, g, b | _tail]} = image) do
			%Avatar.Image{image | color: {r, g, b}}
			end
		\end{lstlisting}
		\begin{itemize}
			\item Extracts the first three elements from the hash to form an RGB color.
			\item Updates the \texttt{Avatar.Image} struct with this color.
		\end{itemize}
	\end{frame}
	
	\begin{frame}[fragile]
		\frametitle{Step 2: Creating a Grid}
		\begin{lstlisting}[language=Elixir]
			def create_grid(%Avatar.Image{hash: hash} = image) do
			grid =
			hash
			|> Enum.chunk_every(3)
			|> Enum.map(&reflect_row/1)
			|> List.flatten
			|> Enum.with_index
			
			%Avatar.Image{image | grid: grid}
			end
		\end{lstlisting}
		\begin{itemize}
			\item Converts the hash into a grid format by chunking the list and reflecting rows to ensure symmetry.
			\item The grid is then flattened and indexed, creating a structured layout for the avatar.
		\end{itemize}
	\end{frame}
	
	\begin{frame}[fragile]
		\frametitle{Step 3: Removing Odd Cells}
		\begin{lstlisting}[language=Elixir]
			def remove_odd_cells(%Avatar.Image{grid: grid} = image) do
			grid = Enum.filter grid, fn({code, _index}) ->
			rem(code, 2) == 0
			end
			
			%Avatar.Image{image | grid: grid}
			end
		\end{lstlisting}
		\begin{itemize}
			\item Filters out cells in the grid where the code is odd, keeping only even values.
			\item This step simplifies the grid, focusing on symmetrical, even-numbered cells for the final avatar.
		\end{itemize}
	\end{frame}
	
	\begin{frame}[fragile]
		\frametitle{Step 4: Generating the Pixel Map}
		\begin{lstlisting}[language=Elixir]
			def generate_pixel_map(%Avatar.Image{grid: grid} = image) do
			pixel_map = Enum.map grid, fn({_code, index}) ->
			x = rem(index, 5) * 50
			y = div(index, 5) * 50
			
			top_left = {x, y}
			bottom_right = {x + 50, y + 50}
			
			{top_left, bottom_right}
			end
			
			%Avatar.Image{image | pixel_map: pixel_map}
			end
		\end{lstlisting}
		\begin{itemize}
			\item Translates the grid into a pixel map by calculating the coordinates for each cell.
			\item Each cell is mapped to a rectangular region on the avatar image, forming the final visual representation.
		\end{itemize}
	\end{frame}
	
	\begin{frame}[fragile]
		\frametitle{Step 5: Rendering the Image}
		\begin{lstlisting}[language=Elixir]
			def render_image(%Avatar.Image{color: color, pixel_map: pixel_map}) do
			image = :egd.create(250, 250)
			fill = :egd.color(color)
			
			Enum.each pixel_map, fn({start, stop}) ->
			:egd.filledRectangle(image, start, stop, fill)
			end
			
			:egd.render(image)
			end
		\end{lstlisting}
		\begin{itemize}
			\item Creates a new image canvas and fills the pixel map regions with the selected color.
			\item The image is then rendered, producing the visual output that represents the avatar.
		\end{itemize}
	\end{frame}
	
	\begin{frame}[fragile]
		\frametitle{Step 6: Storing the Image}
		\begin{lstlisting}[language=Elixir]
			def store_image(image, input) do
			File.write("#{input}_avatar.png", image)
			end
		\end{lstlisting}
		\begin{itemize}
			\item The final image is saved as a PNG file with a name based on the input string.
			\item This concludes the avatar generation process, producing a unique, personalized avatar for each input.
		\end{itemize}
	\end{frame}
	
\end{document}
