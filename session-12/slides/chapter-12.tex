\documentclass[aspectratio=169, table]{beamer}

\usepackage[utf8]{inputenc}
\usepackage{listings} 

\usetheme{Pradita}

\subtitle{IF140303-Web Application Development}

\title{Session-12:\\\LARGE{Applying a Plug in Elixir\\and Phoenix}}
\date[Serial]{\scriptsize {PRU/SPMI/FR-BM-18/0222}}
\author[Pradita]{\small{\textbf{Alfa Yohannis}}}

\lstdefinelanguage{Elixir} {
	keywords={def, defmodule, do, end, for, if, else, true, false},
	basicstyle=\ttfamily\small,
	keywordstyle=\color{blue}\bfseries,
	ndkeywords={@moduledoc, iex, Enum, @doc},
	ndkeywordstyle=\color{purple}\bfseries,
	sensitive=true,
	commentstyle=\color{gray},
	stringstyle=\color{red},
	numbers=left,
	numberstyle=\tiny\color{gray},
	breaklines=true,
	frame=lines,
	backgroundcolor=\color{lightgray!10},
	tabsize=2,
	comment=[l]{\#},
	morecomment=[s]{/*}{*/},
	commentstyle=\color{gray}\ttfamily,
	stringstyle=\color{purple}\ttfamily,
	showstringspaces=false
}

\begin{document}
	
	\frame{\titlepage}
	
	\begin{frame}
		\frametitle{\shortstack{Introduction to \\ Plugs in Elixir}}
		\begin{itemize}
			\item A plug is a specification for composing web applications.
			\item Plugs are the building blocks of Elixir web applications like Phoenix.
			\item There are two types of plugs: 
			\begin{itemize}
				\item \textbf{Function Plug}: A simple plug, defined as a function.
				\item \textbf{Module Plug}: A more complex plug, defined as a module.
			\end{itemize}
		\end{itemize}
	\end{frame}
	
	\begin{frame}
		\frametitle{\shortstack{Creating a Module Plug \\ for Authentication}}
		\begin{itemize}
			\item We will create a module plug to check if a user is logged in.
			\item This plug checks if a user ID is assigned in the connection object.
			\item If the user ID exists, it fetches the user from the database and assigns it to the connection object.
			\item The plug consists of two main functions:
			\begin{itemize}
				\item \texttt{init/1}: Used for setup, called once.
				\item \texttt{call/2}: Called with the connection and returns a connection.
			\end{itemize}
			\item The \texttt{assign/3} function is used to store a value in the connection's assigns.
		\end{itemize}
	\end{frame}
	
	\begin{frame}[fragile]
		\frametitle{Creating the Plug Module}
		\begin{lstlisting}[language=Elixir]
			defmodule Discuss.Plugs.SetUser do
			import Plug.Conn
			import Phoenix.Controller
			
			alias Discuss.Repo
			alias Discuss.User
			alias Discuss.Router.Helpers
			
			def init(\_params) do
			end
		\end{lstlisting}
	\end{frame}
	
	\begin{frame}[fragile]
		\frametitle{Creating the Plug Module}
		\begin{lstlisting}[language=Elixir]
			def call(conn, \_params) do
			user\_id = get\_session(conn, :user\_id)
			cond do
			user = user\_id \&\& Repo.get(User, user\_id) ->
			assign(conn, :user, user)
			true ->
			assign(conn, :user, nil)
			end
			end
			end
		\end{lstlisting}
		\begin{itemize}
			\item \texttt{init/1}: Placeholder function for initialization.
			\item \texttt{call/2}: Checks session for \texttt{user\_id}, assigns user if found.
			\item \texttt{assign/3} stores user data in \texttt{conn.assigns} for later use.
		\end{itemize}
	\end{frame}
	
	\begin{frame}[fragile]
		\frametitle{\shortstack{Using the Plug \\ in the Router}}
		\begin{itemize}
			\item The created plug should be added to the router's pipeline.
			\item Modify \texttt{router.ex} to include the plug in the browser pipeline.
		\end{itemize}
		\begin{lstlisting}[language=Elixir]
			pipeline :browser do
			plug Discuss.Plugs.SetUser
			end
		\end{lstlisting}
	\end{frame}
	
	\begin{frame}[fragile]
		\frametitle{\shortstack{Adding a Login Button \\ to the Header}}
		\begin{lstlisting}[language=Elixir]
			<body>
			<nav class="light-blue">
			<div class="nav-wrapper container">
			<a href="/" class="brand-logo">Logo</a>
			<ul class="right">
			<\%= if @conn.assigns[:user] do \%>
			<li>
			<\%= link "Logout", to: session_path(@conn, :signout) \%>
			</li>
		\end{lstlisting}
	\end{frame}
	
	\begin{frame}[fragile]
		\frametitle{\shortstack{Adding a Login Button \\ to the Header}}
		\begin{itemize}
			\item We add a login button to the application header.
			\item The button checks if a user is logged in and displays the appropriate option.
			\item If the user is logged in, a logout button is shown; otherwise, a login button appears.
		\end{itemize}
	\end{frame}
	
	\begin{frame}[fragile]
		\frametitle{\shortstack{Adding a Login Button \\ to the Header}}
		\begin{lstlisting}[language=Elixir]
			<\% else \%>
			<li>
			<\%= link "Sign in with Github", to: session_path(@conn, :request, "github") \%>
			</li>
			<\% end \%>
			</ul>
			</div>
			</nav>
		\end{lstlisting}
		\begin{itemize}
			\item \texttt{@conn.assigns[:user]} is checked to see if a user is logged in.
			\item The \texttt{link/2} function creates the login/logout link.
		\end{itemize}
	\end{frame}
	
	\begin{frame}[fragile]
		\frametitle{\shortstack{Updating the Router \\ for Signout}}
		\begin{itemize}
			\item We add a signout route to the router.
			\item This allows users to log out of the application.
		\end{itemize}
		\begin{lstlisting}[language=Elixir]
			scope "/auth", Discuss do
			pipe\_through :browser
			get "/signout", AuthController, :signout
			get "/:provider", AuthController, :request
			get "/:provider/callback", AuthController, :request
			end
		\end{lstlisting}
	\end{frame}
	
	\begin{frame}[fragile]
		\frametitle{\shortstack{Updating the AuthController \\ for Signout}}
		\begin{itemize}
			\item We add a \texttt{signout} function to the \texttt{AuthController}.
			\item This function clears the session and redirects the user to the home page.
		\end{itemize}
		\begin{lstlisting}[language=Elixir]
			...
			def signout(conn, \_params) do
			conn
			|> configure\_session(drop: true)
			|> redirect(to: topic\_path(conn, :index))
			end
			...
		\end{lstlisting}
	\end{frame}
	
	\begin{frame}[fragile]
		\frametitle{\shortstack{Adding a Logout Button \\ to the Header}}
		\begin{itemize}
			\item The logout button is dynamically displayed in the header.
			\item When clicked, it triggers the signout route and logs the user out.
		\end{itemize}
		\begin{lstlisting}[language=Elixir]
			<body>
			<nav class="light-blue">
			<div class="nav-wrapper container">
			<a href="/" class="brand-logo">Logo</a>
			<ul class="right">
			<\%= if @conn.assigns[:user] do \%>
			<li>
			<\%= link "Logout", to: session_path(@conn, :signout) \%>
			</li>
		\end{lstlisting}
	\end{frame}
	
	\begin{frame}[fragile]
		\frametitle{\shortstack{Adding a Logout Button \\ to the Header}}
		\begin{lstlisting}[language=Elixir]
			<\% else \%>
			<li>
			<\%= link "Sign in with Github", to: session_path(@conn, :request, "github") \%>
			</li>
			<\% end \%>
			</ul>
			</div>
			</nav>
		\end{lstlisting}
		\begin{itemize}
			\item \texttt{link/2} creates a hyperlink to the logout route.
			\item The button is displayed based on the user's login status.
		\end{itemize}
	\end{frame}
	
	\begin{frame}{\shortstack{Adding Authorization: Restricting \\Actions Based on User Authentication}}
		\begin{itemize}
			\item We need to ensure only signed-in users can post, edit, or delete topics.
			\item To enforce this, we'll create a new plug in `Web > controllers > plug > require\_auth.ex`.
		\end{itemize}
	\end{frame}
	
	\begin{frame}[fragile]{Creating the `RequireAuth` Plug}
		\begin{lstlisting}[language=Elixir]
			defmodule Discuss.Plugs.RequireAuth do
			use Plug.Conn
			use Phoenix.Controller
			
			alias Discuss.Router.Helpers
			
			def init(_params) do
			end
			
			def call(conn, _params) do
			if conn.assigns[:user] do
			conn
		\end{lstlisting}
	\end{frame}
	
	\begin{frame}[fragile]{Creating the `RequireAuth` Plug}
		\begin{lstlisting}[language=Elixir]
			else
			conn
			|> put_flash(:error, "You must be logged in.")
			|> redirect(to: Helpers.topic_path(conn, :index))
			|> halt()
			end
			end
			end
		\end{lstlisting}
		\begin{itemize}
			\item \texttt{halt/0} stops the connection processing immediately.
			\item This ensures that unauthorized users are redirected before further actions.
		\end{itemize}
	\end{frame}
	
	\begin{frame}
		\frametitle{\shortstack{Updating `TopicController` \\ with `RequireAuth` Plug}}
		\begin{itemize}
			\item We will update the `TopicController` to restrict specific actions to signed-in users.
			\item The plug is applied conditionally using `when` with the `action` keyword.
		\end{itemize}
	\end{frame}
	
	\begin{frame}[fragile]
		\frametitle{\shortstack{Updated `TopicController` \\ with `RequireAuth` Plug}}
		\begin{lstlisting}[language=Elixir]
			defmodule Discuss.TopicController do
			use Discuss.Web, :controller
			
			alias Discuss.Topic
			
			plug Discuss.Plugs.RequireAuth when action in [:new, :create, :edit, :update, :delete]
			...
		\end{lstlisting}
		\begin{itemize}
			\item Only the specified actions (new, create, edit, update, delete) will trigger the `RequireAuth` plug.
			\item This ensures non-logged-in users cannot access these actions.
		\end{itemize}
	\end{frame}
	
	\begin{frame}{Associating Users with Topics}
		\begin{itemize}
			\item We'll associate users with topics using a foreign key relationship.
			\item This involves adding a `user\_id` column to the `topics` table.
			\item Migration will alter the existing table, leaving the `user\_id` empty for previously created topics.
		\end{itemize}
	\end{frame}
	
	\begin{frame}[fragile]
		\frametitle{\shortstack{Migration: Adding `user\_id` \\to Topics}}
		\begin{lstlisting}[language=Elixir]
			defmodule Discuss.Repo.Migrations.AddUserIdToTopics do
			use Ecto.Migration
			
			def change do
			alter table(:topics) do
			add :user_id, references(:users)
			end
			end
			end
		\end{lstlisting}
		\begin{itemize}
			\item \texttt{references(:users)} establishes the foreign key relationship.
			\item Migrate the database with `mix ecto.migrate`.
		\end{itemize}
	\end{frame}
	
	\begin{frame}{Updating Models for Association}
		\begin{itemize}
			\item Now, we need to associate the `User` model with the `Topic` model.
			\item This is done by adding the appropriate associations in both models.
		\end{itemize}
	\end{frame}
	
	\begin{frame}[fragile]
		\frametitle{\shortstack{User Model: Adding\\  `has\_many` Association}}
		\begin{lstlisting}[language=Elixir]
			defmodule Discuss.User do
			use Discuss.Web, :model
			
			schema "users" do
			field :email, :string
			field :provider, :string
			field :token, :string
			has_many :topics, Discuss.Topic
			timestamps()
			end
		\end{lstlisting}
	\end{frame}
	
	\begin{frame}[fragile]
		\frametitle{\shortstack{User Model: Adding\\  `has\_many` Association}}
		\begin{lstlisting}[language=Elixir]
			def changeset(struct, params \\ %{}) do
			struct
			|> cast(params, [:email, :provider, :token])
			|> validate_required([:email, :provider, :token])
			end
			end
		\end{lstlisting}
		\begin{itemize}
			\item \texttt{has\_many :topics} sets up the one-to-many relationship.
			\item A user can have multiple topics.
		\end{itemize}
	\end{frame}
	
	\begin{frame}[fragile]
		\frametitle{\shortstack{Topic Model: Adding \\`belongs\_to` Association}}
		\begin{lstlisting}[language=Elixir]
			defmodule Discuss.Topic do
			use Discuss.Web, :model
			
			schema "topics" do
			field :title, :string
			belongs_to :user, Discuss.User
			end
		\end{lstlisting}
	\end{frame}
	
	\begin{frame}[fragile]
		\frametitle{\shortstack{Topic Model: Adding \\`belongs\_to` Association}}
		\begin{lstlisting}[language=Elixir]
			def changeset(struct, params \\ %{}) do
			struct
			|> cast(params, [:title])
			|> validate_required([:title])
			end
			end
		\end{lstlisting}
		\begin{itemize}
			\item \texttt{belongs\_to :user} establishes the reverse relationship.
			\item Each topic is associated with a specific user.
		\end{itemize}
	\end{frame}
	
	\begin{frame}
		\frametitle{\shortstack{Updating `TopicController` to \\Associate Topics with Users}}
		\begin{itemize}
			\item We need to update the `create` function to associate a new topic with the current user.
			\item This involves using `build\_assoc/2`.
		\end{itemize}
	\end{frame}
	
	\begin{frame}[fragile]
		\frametitle{Updating the `create` Function}
		\begin{lstlisting}[language=Elixir]
			def create(conn, %{"topic" => topic}) do
			changeset = conn.assigns.user
			|> build_assoc(:topics)
			|> Topic.changeset(topic)
			
			case Repo.insert(changeset) do
			{:ok, _topic} -> 
			conn
			|> put_flash(:info, "Topic Created")
			|> redirect(to: topic_path(conn, :index))
		\end{lstlisting}
	\end{frame}
	
	\begin{frame}[fragile]
		\frametitle{Updating the `create` Function}
		\begin{lstlisting}[language=Elixir]
			{:error, changeset} -> 
			render conn, "new.html", changeset: changeset
			end
			end
		\end{lstlisting}
		\begin{itemize}
			\item \texttt{build\_assoc/2} automatically associates the new topic with the current user.
			\item This ensures that the `user\_id` field in the `topics` table is populated.
		\end{itemize}
	\end{frame}
	
	\begin{frame}
		\frametitle{\shortstack{Restricting Edit/Delete Buttons Based on Ownership}}
		\begin{itemize}
			\item We will update the `index.html.eex` template to ensure only the topic owner can see edit/delete buttons.
			\item This is done using an `if` statement.
		\end{itemize}
	\end{frame}
	
	\begin{frame}[fragile]{Updating `index.html.eex` Template}
		\begin{lstlisting}[language=Elixir]
			<ul class = "collection">
			<%= for topic <- @topics do %>
			<li class="collection-item"> 
			<%= topic.title %>
			
			<%= if @conn.assigns.user.id == topic.user_id do %>
			<div class="right">
			<%= link "Edit", to: topic_path(@conn, :edit, topic) %>
			<%= link "Delete", to: topic_path(@conn, :delete , topic), method: :delete %>
			</div>
			<% end %>
		\end{lstlisting}
	\end{frame}
	
	\begin{frame}[fragile]{Updating `index.html.eex` Template}
		\begin{lstlisting}[language=Elixir]
			</li>
			<%end%>
			</ul>
		\end{lstlisting}
		\begin{itemize}
			\item The `if` statement ensures that only the topic owner sees the edit/delete options.
			\item This provides a basic level of authorization on the front-end.
		\end{itemize}
	\end{frame}
	
	\begin{frame}{\shortstack{Adding a Plug to Refuse \\Unauthorized Actions}}
		\begin{itemize}
			\item Even though unauthorized users cannot see the buttons, they might still access actions via URL.
			\item We'll create a plug to refuse unauthorized actions.
		\end{itemize}
	\end{frame}
	
	\begin{frame}[fragile]
		\frametitle{\shortstack{Adding a Plug to Refuse \\Unauthorized Actions}}
		\begin{lstlisting}[language=Elixir]
			def check_topic_owner(conn, _params) do
			%{params: %{"id" => topic_id}} = conn
			if Repo.get(Topic, topic_id).user_id == conn.assigns.user.id do
			conn
			else
		\end{lstlisting}
	\end{frame}
	
	\begin{frame}[fragile]
		\frametitle{\shortstack{Adding a Plug to Refuse \\Unauthorized Actions}}
		\begin{lstlisting}[language=Elixir]
			conn
			|> put_flash(:error, "You cannot edit that")
			|> redirect(to: topic_path(conn, :index))
			|> halt()
			end
			end
		\end{lstlisting}
		\begin{itemize}
			\item We compare the `user\_id` of the topic with the `id` of the logged-in user.
			\item If they don't match, we refuse the action.
		\end{itemize}
	\end{frame}
	
	\begin{frame}{Conclusion}
		\begin{itemize}
			\item We implemented authorization to restrict actions based on user authentication.
			\item Users are now associated with their topics.
			\item These steps ensure better security and user management within our Discuss application.
		\end{itemize}
	\end{frame}
	
\end{document}
