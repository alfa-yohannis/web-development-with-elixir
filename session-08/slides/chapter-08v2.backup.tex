\documentclass[aspectratio=169, table]{beamer}

\usepackage[utf8]{inputenc}
\usepackage{listings} 
\usepackage{graphicx}

\usetheme{Pradita}

\subtitle{IF140303-Web Application Development}

\title{Session-08:\\ \LARGE{Create and List Data using\\Phoenix Framework}
	\vspace{5pt}}
\date[Serial]{\scriptsize {PRU/SPMI/FR-BM-18/0222}}
\author[Pradita]{\small{\textbf{Alfa Yohannis}}}


\lstdefinelanguage{Elixir} {
	keywords={def, defmodule, do, end, for, if, else, true, false},
	basicstyle=\ttfamily\small,
	keywordstyle=\color{blue}\bfseries,
	ndkeywords={@moduledoc, iex, Enum, @doc},
	ndkeywordstyle=\color{purple}\bfseries,
	sensitive=true,
	commentstyle=\color{gray},
	stringstyle=\color{red},
	numbers=left,
	numberstyle=\tiny\color{gray},
	breaklines=true,
	frame=lines,
	backgroundcolor=\color{lightgray!10},
	tabsize=2,
	comment=[l]{\#},
	morecomment=[s]{/*}{*/},
	commentstyle=\color{gray}\ttfamily,
	stringstyle=\color{purple}\ttfamily,
	showstringspaces=false
}

\begin{document}
	
	\frame{\titlepage}
	
	\begin{frame}
		\frametitle{\shortstack{Phoenix MVC Model:\\ Model, View, Controller}}
		\begin{itemize}
			\item Model: Represents the data and business logic.
			\item View: Handles the presentation layer.
			\item Controller: Manages the flow of the application.
		\end{itemize}
	\end{frame}
	
	\begin{frame}
		\frametitle{\shortstack{Model, View,\\ Controller Workflow}}
		\begin{itemize}
			\item Request received by the router.
			\item Router directs the request to the appropriate controller.
			\item Controller interacts with the model and selects a view.
			\item View renders the final output.
		\end{itemize}
	\end{frame}
	
	\begin{frame}
		\frametitle{\shortstack{Understanding Phoenix\\ Project Structure}}
		\begin{itemize}
			\item \texttt{controllers/}: Manages request handling.
			\item \texttt{models/}: Represents database tables and business logic.
			\item \texttt{views/}: Manages how data is presented.
		\end{itemize}
	\end{frame}
	
	\begin{frame}
		\frametitle{\shortstack{Request Workflow: Router,\\ Controller, View}}
		\begin{itemize}
			\item Request hits the router.
			\item Router directs to the controller.
			\item Controller processes the request, uses model, and chooses view.
			\item View generates HTML or JSON response.
		\end{itemize}
	\end{frame}
	
	\begin{frame}
		\frametitle{\shortstack{Templates vs. Views}}
		\begin{itemize}
			\item Templates: HTML or eEx files used for rendering.
			\item Views: Elixir modules that process data and select templates.
			\item Naming conventions are crucial for Phoenix to locate files.
		\end{itemize}
	\end{frame}
	
	\begin{frame}
		\frametitle{Starting Phoenix with IEx}
		\begin{itemize}
			\item In Elixir: \texttt{iex -S mix}.
			\item In Phoenix: \texttt{iex -S mix phx.server}.
		\end{itemize}
	\end{frame}
	
	\begin{frame}
		\frametitle{Phoenix Models and Database}
		\begin{itemize}
			\item Models represent database tables.
			\item Phoenix needs to be told what data is in the database.
			\item Models define the structure of database tables.
		\end{itemize}
	\end{frame}
	
	\begin{frame}[fragile]
		\frametitle{Creating a Topics Model}
		\begin{lstlisting}[language=Elixir]
			defmodule Discuss.Topic do
			use Discuss.Web, :model
			
			schema "topics" do
			field :title, :string
			belongs_to :user, Discuss.User
			has_many :comments, Discuss.Comment
			end
			
			def changeset(struct, params \\ %{}) do
			struct
			|> cast(params, [:title])
			|> validate_required([:title])
			end
			end
		\end{lstlisting}
	\end{frame}
	
	\begin{frame}
		\frametitle{\shortstack{Understanding Migrations}}
		\begin{itemize}
			\item Migrations manage changes to your database schema.
			\item Command: \texttt{mix ecto.gen.migration add\_topics}.
			\item Files stored in \texttt{priv/repo/migrations}.
			\item Migration filenames include a timestamp.
		\end{itemize}
	\end{frame}
	
	\begin{frame}[fragile]
		\frametitle{Creating a Topics Migration}
		\begin{lstlisting}[language=Elixir]
			defmodule Discuss.Repo.Migrations.AddTopics do
			use Ecto.Migration
			
			def change do
			create table(:topics) do
			add :title, :string
			end
			end
			end
		\end{lstlisting}
	\end{frame}
	
	\begin{frame}
		\frametitle{\shortstack{Using GUI Databases\\ for PostgreSQL}}
		\begin{itemize}
			\item GUI tools like Postico or pgAdmin can simplify database management.
			\item These tools provide a user-friendly interface for interacting with PostgreSQL.
		\end{itemize}
	\end{frame}
	
	\begin{frame}
		\frametitle{\shortstack{Project Problems}}
		\begin{table}[]
			\centering
			\begin{tabular}{|l|}
				\hline
				\textbf{Problem} \\ \hline
				Need a new URL for the user to visit \\ \hline
				New routes must map up to a method in a controller \\ \hline
				Need to show a form to the user \\ \hline
				Need to translate data in the form to something that can be saved \\ \hline
				The controller and view are related to 'Page', but we are making stuff related to 'Topic' \\ \hline
			\end{tabular}
		\end{table}
	\end{frame}
	
	\begin{frame}
		\frametitle{\shortstack{Project Solutions}}
		\begin{table}[]
			\centering
			\begin{tabular}{|l|}
				\hline
				\textbf{Solution} \\ \hline
				Add a new route in router file \\ \hline
				Add a new method in a controller to handle this request \\ \hline
				Make a new template that contains the form \\ \hline
				Create a 'Topic' model that can translate raw data into something that can be saved \\ \hline
				Make a new controller and view to handle everything related to topics \\ \hline
			\end{tabular}
		\end{table}
	\end{frame}
	
	
	\begin{frame}
		\frametitle{Handling Requests in Router}
		\begin{itemize}
			\item Ensure request handling in the router.
			\item Example code:
			\item 
			\texttt{
				scope "/", Discuss do\\
				pipe\_through :browser\\
				get "/", PageController, :index\\
				get "/topics/new", TopicController, :new\\
				end
				}
		\end{itemize}
	\end{frame}
	
	\begin{frame}
		\frametitle{\shortstack{Controller Function\\ Name Conventions}}
		\begin{itemize}
			\item \texttt{new}: Render form for new resource.
			\item \texttt{create}: Handle form submission and create resource.
			\item \texttt{index}: List all resources.
			\item \texttt{delete}: Remove a resource.
			\item \texttt{edit}: Render form to edit resource.
			\item \texttt{update}: Handle form submission and update resource.
		\end{itemize}
	\end{frame}
	
	\begin{frame}[fragile]
		\frametitle{Creating a Simple TopicController}
		\begin{lstlisting}[language=Elixir]
			defmodule Discuss.TopicController do
			use Discuss.Web, :controller
			def new(conn, params) do
			end
			end
		\end{lstlisting}
	\end{frame}
	
	\begin{frame}
		\frametitle{Fixing the Init Error}
		\begin{itemize}
			\item Error: function Discuss.TopicController.init/1 is undefined or private.
			\item Solution: Add \texttt{use Discuss.Web, :controller} to include required functions.
			\item Example: \\ \texttt{
				defmodule Discuss.TopicController do\\
				use Discuss.Web, :controller\\
				def new(conn, params) do\\
				end\\
				end
			}
		\end{itemize}
	\end{frame}
	
	\begin{frame}
		\frametitle{\shortstack{Elixir Keywords: \texttt{import},\\ \texttt{alias}, \texttt{use}}}
		\begin{itemize}
			\item \texttt{import}: Bring functions or macros into scope.
			\item \texttt{alias}: Create shorter names for modules.
			\item \texttt{use}: Inject functionality into a module.
		\end{itemize}
	\end{frame}
	
	\begin{frame}
		\frametitle{\shortstack{Role of \texttt{web.ex} in Phoenix}}
		\begin{itemize}
			\item \texttt{web.ex}: Contains functionality for models, controllers, and views.
			\item \texttt{use Discuss.Web, :controller}: Imports default controller functions and setup.
		\end{itemize}
	\end{frame}
	
	\begin{frame}
		\frametitle{\shortstack{Content of controller in \texttt{web.ex}}}
		\begin{itemize}
			\item
			\texttt{
				def controller do\\
				quote do\\
				use Phoenix.Controller\\
				alias Discuss.Repo\\
				import Ecto\\
				import Ecto.Query\\
				import Discuss.Router.Helpers\\
				import Discuss.Gettext\\
				end\\
				end
			}
		\end{itemize}
	\end{frame}
	
	\begin{frame}
		\frametitle{\shortstack{Eliminating Errors with \\\texttt{use Discuss.Web, :controller}}}
		\begin{itemize}
			\item Adding \texttt{use Discuss.Web, :controller} resolves init function errors. Because the init function are included in it
			\item Example:\\\texttt{
				defmodule Discuss.TopicController do\\
				use Discuss.Web, :controller\\
				def new(conn, params) do\\
				end\\
				end
			}
		\end{itemize}
	\end{frame}
	
	\begin{frame}
		\frametitle{\shortstack{Function Signature Mismatch\\ Errors}}
		\begin{itemize}
			\item Error format: \texttt{<function\_name>/<number>}
			\item Example: \texttt{new/0} indicates the function \texttt{new} takes 0 arguments.
		\end{itemize}
	\end{frame}
	
	\begin{frame}
		\frametitle{\shortstack{Debugging with \texttt{IO.inspect},\\ \texttt{conn}, and \texttt{params}}}
		\begin{itemize}
			\item \texttt{IO.inspect}: Prints values to the console for debugging.
			\item \texttt{conn}: Contains information about the request and response.
			\item \texttt{params}: Contains query parameters and form data.
		\end{itemize}
	\end{frame}
	
	\begin{frame}[fragile]
		\frametitle{\shortstack{Modifying TopicController to Use \\\texttt{conn} and \texttt{params}}}
		\begin{verbatim}[language=Elixir]
			defmodule Discuss.TopicController do
			use Discuss.Web, :controller
			def new(conn, params) do
			IO.inspect(conn)
			IO.inspect(params)
			end
			end
		\end{verbatim}
	\end{frame}
	
	\begin{frame}
		\frametitle{\shortstack{Form Validation: Title Required}}
		\begin{itemize}
			\item Ensure that form input has a title.
			\item Show error message: "You must enter a title" if missing.
		\end{itemize}
	\end{frame}
	
\end{document}
