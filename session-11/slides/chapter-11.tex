\documentclass[aspectratio=169, table]{beamer}

\usepackage[utf8]{inputenc}
\usepackage{listings}

\usetheme{Pradita}

\subtitle{IF140303-Web Application Development}

\title{Session-11:\\\LARGE{Session and Authentication using\\Plugs in Phoenix Framework}}
\date[Serial]{\scriptsize{PRU/SPMI/FR-BM-18/0222}}
\author[Pradita]{\small{\textbf{Alfa Yohannis}}}

\lstdefinelanguage{Elixir}{
	keywords={def, defmodule, do, end, for, if, else, true, false, use, alias, plug},
	basicstyle=\ttfamily\small,
	keywordstyle=\color{blue}\bfseries,
	ndkeywords={@moduledoc, iex, Enum, @doc},
	ndkeywordstyle=\color{purple}\bfseries,
	sensitive=true,
	commentstyle=\color{gray},
	stringstyle=\color{red},
	numbers=left,
	numberstyle=\tiny\color{gray},
	breaklines=true,
	frame=lines,
	backgroundcolor=\color{lightgray!10},
	tabsize=2,
	comment=[l]{\#},
	morecomment=[s]{/*}{*/},
	commentstyle=\color{gray}\ttfamily,
	stringstyle=\color{purple}\ttfamily,
	showstringspaces=false
}

\begin{document}
	
	\frame{\titlepage}
	
	\begin{frame}
		\frametitle{Introduction to GitHub OAuth Authentication}
		\begin{itemize}
			\item We'll implement OAuth authentication using GitHub.
			\item OAuth flow:
			\begin{enumerate}
				\item User clicks "Login with GitHub" on the web page.
				\item Redirect to GitHub.com for authorization.
				\item After authorization, GitHub redirects back with an authorization code.
				\item Use the authorization code to retrieve user details.
				\item Create a new user record and log the user in.
			\end{enumerate}
		\end{itemize}
	\end{frame}
	
	\begin{frame}
		\frametitle{Using the \shortstack{Ueberauth Helper \\ Library for OAuth}}
		\begin{itemize}
			\item We'll use the \texttt{Ueberauth} library to handle the OAuth process.
			\item Create a new controller to manage OAuth flows.
			\item When the user clicks "Login with GitHub":
			\begin{enumerate}
				\item The controller triggers \texttt{Ueberauth} to start the OAuth flow.
				\item Redirect to GitHub for authentication.
				\item Handle GitHub's response with a callback function in the controller.
			\end{enumerate}
		\end{itemize}
	\end{frame}
	
	\begin{frame}[fragile]
		\frametitle{Installing Dependencies}
		\begin{itemize}
			\item Open \texttt{mix.exs} to add dependencies:
			\begin{lstlisting}[language=Elixir]
				defp deps do
				[{..., :ueberauth, "~> 0.3"},
				{:ueberauth_github, "~> 0.4"}]
				end
				def application do
				[extra_applications: [...,:ueberauth, :ueberauth_github]]
				end
			\end{lstlisting}
			\item Run \texttt{mix deps.get} to install dependencies.
		\end{itemize}
	\end{frame}
	
	\begin{frame}
		\frametitle{Generating GitHub API Key}
		\begin{itemize}
			\item Go to GitHub to generate an API key for OAuth.
			\begin{enumerate}
				\item Navigate to Profile > Settings > Developer Settings.
				\item Click on \texttt{OAuth applications} and register a new application.
				\item Fill out the form, including the Authorization Callback URL.
				\item Generate the Client ID and Client Secret.
			\end{enumerate}
		\end{itemize}
	\end{frame}
	
	\begin{frame}[fragile]
		\frametitle{\shortstack{Including the API Key in \\ Configuration Files}}
		\begin{itemize}
			\item Add the API key to \texttt{config.exs}:
			\begin{lstlisting}[language=Elixir]
				config :ueberauth, Ueberauth,
				providers: [
				github: {Ueberauth.Strategy.Github, []}]
				config :ueberauth, Ueberauth.Strategy.Github.OAuth,
				client_id: "your_generated_client_id",
				client_secret: "your_generated_secret_id"
			\end{lstlisting}
			\item This configuration links Ueberauth with GitHub OAuth.
			\item It's recommended to hide the API key if uploading to a public repository.
		\end{itemize}
	\end{frame}
	
	\begin{frame}[fragile]
		\frametitle{Creating the \shortstack{Authentication Controller}}
		\begin{itemize}
			\item Create a new controller at \texttt{web > controllers > auth\_controller.ex}:
			\begin{lstlisting}[language=Elixir]
				defmodule Discuss.AuthController do
				use Discuss.Web, :controller
				plug Ueberauth
				def callback(conn, params) do
				IO.inspect(conn.assigns)
				IO.inspect(params)
				end
				end
			\end{lstlisting}
			\item \texttt{conn.assigns} returns user data (email, username, etc.).
			\item \texttt{params} returns the authorization code and provider.
		\end{itemize}
	\end{frame}
	
	\begin{frame}[fragile]
		\frametitle{Updating the Router}
		\begin{itemize}
			\item Open \texttt{router.ex} and update it to include the auth routes:
			\begin{lstlisting}[language=Elixir]
				scope "/", Discuss do
				pipe_through :browser
				resources "/", TopicController
				end
				
				scope "/auth", Discuss do
				pipe_through :browser
				get "/:provider", AuthController, :request
				get "/:provider/callback", AuthController, :callback
				end
			\end{lstlisting}
			\item \texttt{pipe\_through} specifies the pipeline to use.
			\item The \texttt{scope "/auth"} block defines routes for OAuth-related actions.
		\end{itemize}
	\end{frame}
	
	\begin{frame}[fragile]
		\frametitle{\shortstack{Generating Migration \\ for Users Table}}
		\begin{itemize}
			\item Generate a migration file with:
			\texttt{mix ecto.gen.migration add\_users}
			\item In the migration file:
			\begin{lstlisting}[language=Elixir]
				defmodule Discuss.Repo.Migrations.AddUsers do
				use Ecto.Migration
				
				def change do
				create table(:users) do
				add :email, :string
				add :provider, :string
				add :token, :string
				timestamps()
				end
				end
				end
			\end{lstlisting}
			\item This creates a \texttt{users} table with \texttt{email}, \texttt{provider}, \texttt{token}, and timestamps.
		\end{itemize}
	\end{frame}
	
	\begin{frame}[fragile]
		\frametitle{Creating the \shortstack{User Model}}
		\begin{itemize}
			\item Create a user model in \texttt{web > models > user.ex}:
			\begin{lstlisting}[language=Elixir]
				defmodule Discuss.User do
				use Discuss.Web, :model
				
				schema "users" do
				field :email, :string
				field :provider, :string
				field :token, :string
				timestamps()
				end
				
				def changeset(struct, params \\ %{}) do
				struct
				|> cast(params, [:email, :provider, :token])
				|> validate_required([:email, :provider, :token])
				end
				end
			\end{lstlisting}
			\item This defines the structure and validations for the \texttt{users} table.
		\end{itemize}
	\end{frame}
	
	\begin{frame}[fragile]
		\frametitle{Updating the \shortstack{Auth Controller}}
		\begin{itemize}
			\item Update \texttt{auth\_controller.ex} to handle user authentication:
			\begin{lstlisting}[language=Elixir]
				defmodule Discuss.AuthController do
				use Discuss.Web, :controller
				plug Ueberauth
				
				alias Discuss.User
				
				def callback(%{assigns: %{ueberauth_auth: auth}} = conn, _params) do
				user_params = %{token: auth.credentials.token,
					email: auth.info.email,
					provider: "github"}
				changeset = User.changeset(%User{}, user_params)
				insert_or_update_user(changeset)
				end
				
				defp insert_or_update_user(changeset) do
				case Repo.get_by(User, email: changeset.changes.email) do
				nil -> Repo.insert(changeset)
				user -> {:ok, user}
				end
				end
				end
			\end{lstlisting}
			\item \texttt{callback/2} processes GitHub's OAuth response.
			\item \texttt{insert\_or\_update\_user/1} inserts a new user or updates an existing one.
		\end{itemize}
	\end{frame}
	
	\begin{frame}
		\frametitle{Short Explanation of Cookies}
		\begin{itemize}
			\item Cookies store user data on the client side, enabling session persistence.
			\item In our flow:
			\begin{enumerate}
				\item After successful OAuth, the user's ID is stored in a cookie.
				\item On subsequent requests, the server checks the cookie to retrieve the user ID.
				\item If the user exists, they're automatically signed in.
			\end{enumerate}
		\end{itemize}
	\end{frame}
	
	\begin{frame}[fragile]
		\frametitle{Adding the \shortstack{Sign-In Functionality}}
		\begin{itemize}
			\item Update \texttt{auth\_controller.ex} to include sign-in:
			\begin{lstlisting}[language=Elixir]
				defmodule Discuss.AuthController do
				use Discuss.Web, :controller
				plug Ueberauth
				
				alias Discuss.User
				
				def callback(%{assigns: %{ueberauth_auth: auth}} = conn, _params) do
				user_params = %{token: auth.credentials.token,
					email: auth.info.email,
					provider: "github"}
				changeset = User.changeset(%User{}, user_params)
				signin(conn, changeset)
				end
				
				defp signin(conn, changeset) do
				case insert_or_update_user(changeset) do
				{:ok, user} ->
				conn
				|> put_flash(:info, "Welcome back!")
				|> put_session(:user_id, user.id)
				|> redirect(to: topic_path(conn, :index))
				{:error, _reason} ->
				conn
				|> put_flash(:error, "Error signing in")
				|> redirect(to: topic_path(conn, :index))
				end
				end
				
				defp insert_or_update_user(changeset) do
				case Repo.get_by(User, email: changeset.changes.email) do
				nil -> Repo.insert(changeset)
				user -> {:ok, user}
				end
				end
				end
			\end{lstlisting}
			\item \texttt{signin/2} manages the sign-in process, storing the user's ID in a session.
			\item This ensures users are recognized upon returning to the application.
		\end{itemize}
	\end{frame}
	
\end{document}
