\documentclass[aspectratio=169, table]{beamer}

\usepackage[utf8]{inputenc}
\usepackage{listings} 

\usetheme{Pradita}

\subtitle{IF140303-Web Application Development}

\title{\LARGE{Session-04:\\Atom, Map, Tuple, List, and\\Keywords in Elixir}
	}
\date[Serial]{\scriptsize {PRU/SPMI/FR-BM-18/0222}}
\author[Pradita]{\small{\textbf{Alfa Yohannis}}}

\lstdefinelanguage{Elixir} {
	keywords={def, defmodule, do, end, for, if, else, true, false},
	basicstyle=\ttfamily\small,
	keywordstyle=\color{blue}\bfseries,
	ndkeywords={@moduledoc, iex, Enum, @doc},
	ndkeywordstyle=\color{purple}\bfseries,
	sensitive=true,
	commentstyle=\color{gray},
	stringstyle=\color{red},
	numbers=left,
	numberstyle=\tiny\color{gray},
	breaklines=true,
	frame=lines,
	backgroundcolor=\color{lightgray!10},
	tabsize=2,
	comment=[l]{\#},
	morecomment=[s]{/*}{*/},
	commentstyle=\color{gray}\ttfamily,
	stringstyle=\color{purple}\ttfamily,
	showstringspaces=false
}

\begin{document}
	
	\frame{\titlepage}
	
	\begin{frame}
		\frametitle{Overview of Maps in Elixir}
		\begin{itemize}
			\item Maps are key-value data structures in Elixir.
			\item Keys can be of any type, allowing for flexible data representation.
			\item Useful for representing structured data.
		\end{itemize}
	\end{frame}
	
	\begin{frame}[fragile]
		\frametitle{Creating and Accessing Maps}
		\begin{lstlisting}[language=Elixir]
			iex> colors = %{primary: "red"}
			iex> colors.primary
			"red"
		\end{lstlisting}
		\begin{itemize}
			\item Maps are created using the \%{} syntax.
			\item Access map values using the dot notation or via keys.
		\end{itemize}
	\end{frame}
	
	\begin{frame}[fragile]
		\frametitle{Pattern Matching with Maps}
		\begin{lstlisting}[language=Elixir]
			iex> colors = %{primary: "red", secondary: "blue"}
			iex> %{secondary: secondary_color} = colors
			iex> secondary_color
			"blue"
		\end{lstlisting}
		\begin{itemize}
			\item Maps can be pattern matched to extract values.
			\item Allows for easy decomposition of map data.
		\end{itemize}
	\end{frame}
	
	\begin{frame}[fragile]
		\frametitle{Immutability and Updating Maps}
		\begin{lstlisting}[language=Elixir]
			iex> colors = %{primary: "red"}
			iex> colors.primary = "blue"
			** (MatchError) no match of right hand side value: "blue"
		\end{lstlisting}
		\begin{itemize}
			\item Maps are immutable, meaning they cannot be directly updated.
			\item Updates require creating a new map.
		\end{itemize}
	\end{frame}
	
	\begin{frame}[fragile]
		\frametitle{Updating Maps with \texttt{put/3} and the \texttt{\%{ | }} Syntax}
		\begin{lstlisting}[language=Elixir]
			iex> colors = %{primary: "red"}
			iex> colors = Map.put(colors, :primary, "blue")
			iex> colors
			%{primary: "blue"}
			
			iex> colors = %{colors | primary: "blue"}
			iex> colors
			%{primary: "blue"}
		\end{lstlisting}
		\begin{itemize}
			\item Use \texttt{Map.put/3} or the \texttt{\%{ | }} syntax to update maps.
			\item This creates a new map with the updated values.
		\end{itemize}
	\end{frame}
	
	\begin{frame}
		\frametitle{Overview of Keyword Lists}
		\begin{itemize}
			\item Keyword lists are lists of key-value pairs.
			\item Keys must be atoms, and keys are ordered.
			\item Commonly used in Ecto and other Elixir libraries.
		\end{itemize}
	\end{frame}
	
	\begin{frame}[fragile]
		\frametitle{Creating and Accessing Keyword Lists}
		\begin{lstlisting}[language=Elixir]
			iex> colors = [{:primary, "red"}, {:secondary, "green"}]
			iex> colors[:primary]
			"red"
		\end{lstlisting}
		\begin{itemize}
			\item Keyword lists can be accessed using the key in square brackets.
			\item Useful for simple, ordered collections of key-value pairs.
		\end{itemize}
	\end{frame}
	
	\begin{frame}[fragile]
		\frametitle{Updating Keyword Lists}
		\begin{lstlisting}[language=Elixir]
			iex> colors = [primary: "red", secondary: "blue"]
			iex> colors = [primary: "red", primary: "blue"]
			iex> colors[:primary]
			"blue"
		\end{lstlisting}
		\begin{itemize}
			\item Updating a keyword list involves reassigning it with new values.
			\item Duplicate keys in keyword lists will result in the last value being used.
		\end{itemize}
	\end{frame}
	
	\begin{frame}
		\frametitle{Summary}
		\begin{itemize}
			\item We explored Maps and Keyword Lists in Elixir.
			\item We learned how to create, access, and update both data structures.
			\item Understanding these structures is key for working with Elixir data and using libraries like Ecto.
		\end{itemize}
	\end{frame}
	
\end{document}
