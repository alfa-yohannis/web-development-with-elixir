\documentclass[aspectratio=169, table]{beamer}

\usepackage[utf8]{inputenc}
\usepackage{listings} 

\usetheme{Pradita}

\subtitle{IF140303-Web Application Development}

\title{Session-03:\\
\Huge{
Elixir Documentation\\
\vspace{10pt}
and Testing
\vspace{-20pt}
}
}
\date[Serial]{\scriptsize{PRU/SPMI/FR-BM-18/0222}}
\author[Pradita]{\small{\textbf{Alfa Yohannis}}}

\lstdefinelanguage{Elixir} {
	keywords={case, def, defmodule, do, end, for, if, else, true, false},
	basicstyle=\ttfamily\small,
	keywordstyle=\color{blue}\bfseries,
	ndkeywords={@moduledoc, iex, Enum, @doc},
	ndkeywordstyle=\color{purple}\bfseries,
	sensitive=true,
	numbers=left,
	numberstyle=\tiny\color{gray},
	breaklines=true,
	frame=lines,
	backgroundcolor=\color{lightgray!10},
	tabsize=2,
	comment=[l]{\#},
	morecomment=[s]{/*}{*/},
	commentstyle=\color{gray}\ttfamily,
	showstringspaces=false,
	% string settings
	morestring=[b]",
	morestring=[b]',
	stringstyle=\color{black}\ttfamily, % default, will be overridden
	moredelim=[s][\color{blue}\ttfamily]{"}{"},   % double quotes
	moredelim=[s][\color{teal}\ttfamily]{'}{'}    % single quotes
}


\lstdefinelanguage{bash} {
	keywords={},
	basicstyle=\ttfamily\small,
	keywordstyle=\color{blue}\bfseries,
	ndkeywords={iex},
	ndkeywordstyle=\color{purple}\bfseries,
	sensitive=true,
	commentstyle=\color{gray},
	stringstyle=\color{red},
	numbers=left,
	numberstyle=\tiny\color{gray},
	breaklines=true,
	frame=lines,
	backgroundcolor=\color{lightgray!10},
	tabsize=2,
	comment=[l]{\#},
	morecomment=[s]{/*}{*/},
	commentstyle=\color{gray}\ttfamily,
	stringstyle=\color{purple}\ttfamily,
	showstringspaces=false
}

\begin{document}
	
	\frame{\titlepage}
	
		\begin{frame}[fragile]
		\frametitle{Contents}
		\vspace{20pt}
		\begin{columns}[t]
			\column{0.4\textwidth}
			\tableofcontents[sections={1-3}]
			
			\column{0.6\textwidth}
			\tableofcontents[sections={4-6}]
		\end{columns}
	\end{frame}

\section{Documentation in Elixir}

\begin{frame}[fragile]{Documentation in Elixir}
\vspace{15pt}
Elixir provides simple support for documentation in modules and functions.  
\begin{itemize}
  \item Documentation is written directly in the code using special annotations.  
  \item Use \texttt{@moduledoc} to document a module.  
  \item Use \texttt{@doc} to document functions.  
  \item Documentation can be exported to HTML using the command:
\end{itemize}

\begin{lstlisting}[language=bash]
mix docs
\end{lstlisting}

This command generates documentation in HTML format, which can be viewed in a browser.
\end{frame}


\subsection{Module-level Documentation}

\begin{frame}[fragile]{Module-level Documentation}
\vspace{15pt}
To provide a description of the purpose, usage,  
and examples for a module, use the \texttt{@moduledoc} annotation.  

\begin{lstlisting}[language=Elixir]
defmodule ExampleModule do
  @moduledoc """
  This module is an example of how to document a module in Elixir.

  It demonstrates how module documentation can be written
  and used as a reference when generating HTML documentation.
  """

  # Other functions and logic can be written here
end
\end{lstlisting}

\end{frame}


\subsection{Function-level Documentation}

\begin{frame}[fragile]{Function-level Documentation}
\vspace{15pt}

\begin{columns}

\begin{column}[t]{0.4\textwidth}

In addition to modules, each function can also have documentation  
with the \texttt{@doc} annotation. This usually describes:  
\begin{itemize}
  \item The purpose of the function.  
  \item The parameters required.  
  \item The return value.  
  \item Examples of usage in IEx.  
\end{itemize}
\end{column}

\begin{column}[t]{0.6\textwidth}

\begin{lstlisting}[language=Elixir, basicstyle=\ttfamily\scriptsize]
defmodule ExampleModule do
  @moduledoc """
  This module is an example of how to document a module in Elixir.
  """

  @doc """
  This function adds two numbers.

  ## Parameters
  - a: the first number.
  - b: the second number.

  ## Example
      iex> ExampleModule.add(2, 3)
      5
  """
  def add(a, b) do
    a + b
  end
end
\end{lstlisting}
\end{column}

\end{columns}
\end{frame}


\section{Adding \texttt{ex\_doc} for Documentation}

\subsection{Steps to Add \texttt{ex\_doc}}

\begin{frame}[fragile]{Adding \texttt{ex\_doc} (Part 1)}
\vspace{15pt}
To generate project documentation in Elixir,  
you need to add the \texttt{ex\_doc} library.  

\begin{enumerate}
  \item Open the \texttt{mix.exs} file in your project,  
        then add \texttt{ex\_doc} to the list of dependencies  
        in the \texttt{deps} function. Make sure it is included only in the \texttt{dev} environment.  

\begin{lstlisting}[language=Elixir]
defp deps do
  [
    {:ex_doc, "~> 0.34", only: :dev, runtime: false}
  ]
end
\end{lstlisting}

  \item Run the following command to download and install dependencies:  
\begin{lstlisting}[language=bash]
mix deps.get
\end{lstlisting}
\end{enumerate}
\end{frame}


\begin{frame}[fragile]{Adding \texttt{ex\_doc} (Part 2)}
\vspace{15pt}
\begin{enumerate}
  \setcounter{enumi}{2}
  \item After the dependency is installed,  
        generate the documentation with the command:  

\begin{lstlisting}[language=bash]
mix docs
\end{lstlisting}

  \item The generated documentation will be stored in the \texttt{doc/} folder.  
        Open the \texttt{index.html} file in a browser to view the HTML documentation.
\end{enumerate}

\begin{block}{Note}
\texttt{ex\_doc} allows you to generate documentation  
in multiple formats (HTML, EPUB, PDF),  
making it very useful for both open-source and internal projects.
\end{block}
\end{frame}

\section{Generating HTML Documentation}

\begin{frame}[fragile]{Generating HTML Documentation}
\vspace{15pt}
To generate documentation in HTML format,  
Elixir provides the simple command \texttt{mix docs}.  
This command collects documentation from modules and functions,  
and converts it into human-readable HTML pages.

\begin{enumerate}
  \item Make sure you are in the Elixir project directory.  
  \item Run the following command in the terminal:  
\begin{lstlisting}[language=bash]
mix docs
\end{lstlisting}
  \item The HTML documentation will be stored in the \texttt{doc/} folder.  
        Open the \texttt{index.html} file in a browser.  
\end{enumerate}
\vspace{5pt}
\textbf{Benefit:}\\
HTML documentation makes it easier for others (or yourself in the future)  
to understand the code structure, functions, and how to use the modules.  
\end{frame}

\section{Unit Testing in Elixir}

\begin{frame}{Unit Testing in Elixir}
\vspace{15pt}
Unit testing in Elixir can be done using the built-in module  
\texttt{ExUnit}.  

\begin{itemize}
  \item Tests are written in dedicated files ending with \texttt{\_test.exs}.  
  \item Test modules use \texttt{ExUnit.Case} as a template.  
  \item Supports \texttt{assert}, \texttt{refute}, and additional verification features.  
  \item Function documentation can also be tested with the \texttt{doctest} annotation.  
\end{itemize}

\vspace{5pt}
\textbf{Benefit:} Unit tests ensure that functions work as expected,  
and help prevent bugs when making code changes.  
\end{frame}


\subsection{Unit Test in a Separate File}
\begin{frame}[fragile]{Unit Test in a Separate File}
\vspace{20pt}
Example of a test file inside the \texttt{test/} folder  
with the name \texttt{example\_module\_test.exs}:  

\begin{lstlisting}[language=Elixir, basicstyle=\ttfamily\scriptsize]
defmodule ExampleModuleTest do
  use ExUnit.Case

  test "adds two numbers" do
    assert ExampleModule.add(2, 3) == 5
  end

  test "subtracts two numbers" do
    assert ExampleModule.subtract(5, 2) == 3
  end
end
\end{lstlisting}

\textbf{Running the Tests:} Execute all tests with the command:  
\begin{lstlisting}[language=bash]
mix test
\end{lstlisting}
\end{frame}

\subsection{Function Documentation Tests (Doctest)}

\begin{frame}[fragile]{Unit Tests in Function Docs (Doctest)}
\vspace{20pt}
\begin{columns}

\begin{column}[t]{0.45\textwidth}
Elixir also allows you to  
write unit tests directly in function documentation using the  
\texttt{doctest} annotation. These tests are executed automatically  
when \texttt{ExUnit} runs.  

\vspace{10pt}
\textbf{Advantages:}
\begin{itemize}
  \item Documentation and tests always stay in sync.  
  \item Code examples in \texttt{@doc} are actually verified.  
  \item Easy for readers to understand and confirm correctness.  
\end{itemize}
\end{column}

\begin{column}[t]{0.55\textwidth}
\textbf{Example Module with Doctest:}
\begin{lstlisting}[language=Elixir, basicstyle=\ttfamily\tiny]
defmodule ExampleModule do
  @moduledoc """
  Demo module with documentation and tests.
  """

  @doc """
  Adds two numbers.

  ## Example
      iex> ExampleModule.add(2, 3)
      5
  """
  def add(a, b), do: a + b

  @doc """
  Subtracts two numbers.

  ## Example
      iex> ExampleModule.subtract(5, 2)
      3
  """
  def subtract(a, b), do: a - b
end
\end{lstlisting}
\end{column}

\end{columns}
\end{frame}



\begin{frame}[fragile]{Enabling and Running Doctest}
\vspace{15pt}
To enable \texttt{doctest}, add it inside your test module:  

\begin{lstlisting}[language=Elixir, basicstyle=\ttfamily\scriptsize]
defmodule ExampleModuleTest do
  use ExUnit.Case
  doctest ExampleModule
end
\end{lstlisting}

\vspace{10pt}
\textbf{Execution:} Run the following command in the terminal:  

\begin{lstlisting}[language=bash]
mix test
\end{lstlisting}

Doctest will execute the example code written in the documentation  
and verify that the results match the expected output.
\end{frame}

\subsection{Running Tests from the Command Prompt}
\begin{frame}[fragile]{Running Tests from the Command Prompt}
\vspace{15pt}
Elixir makes it easy to run tests directly from the command prompt.  
There are several options depending on your needs:
\vspace{10pt}
\begin{columns}

\begin{column}[t]{0.48\textwidth}
\textbf{Run All Tests}
\begin{lstlisting}[language=bash]
mix test
\end{lstlisting}

This command executes all tests  
inside the \texttt{test} folder and displays  
the results in the terminal.
\end{column}

\begin{column}[t]{0.52\textwidth}
\textbf{Run Tests in a Specific File}
\begin{lstlisting}[language=bash]
mix test test/example_module_test.exs
\end{lstlisting}

This command executes tests only from a specific file,  
for example \texttt{ExampleModuleTest}.
\end{column}

\end{columns}
\end{frame}

\subsection{Running Tests for a Specific Function}
\begin{frame}[fragile]{Running Tests for a Specific Function}
\vspace{15pt}
You can run tests for a specific function in a module  
by using a \textbf{line number tag}.  
\vspace{10pt}
\begin{columns}
\begin{column}[t]{0.48\textwidth}
\textbf{Example: \texttt{add} function}  
If the test for \texttt{add} is on line 4:  
\begin{lstlisting}[language=bash]
mix test test/example_module_test.exs:4
\end{lstlisting}
\end{column}

\begin{column}[t]{0.52\textwidth}
\textbf{Example: \texttt{subtract} function}  
If the test for \texttt{subtract} is on line 8:  
\begin{lstlisting}[language=bash]
mix test test/example_module_test.exs:8
\end{lstlisting}
\end{column}
\end{columns}

\vspace{8pt}
With this approach, only the test on that line will be executed,  
making debugging and partial testing easier.  
\end{frame}

\section{Exercises}

\subsection{Writing Module Documentation}
\begin{frame}[fragile]{Exercise 1: Writing Module Documentation}
\vspace{15pt}
\begin{columns}

\begin{column}[t]{0.45\textwidth}
\textbf{Instructions:}
\begin{itemize}
  \item Create an Elixir module named \texttt{Calculator}.  
  \item Add two functions: \texttt{multiply/2} and \texttt{divide/2}.  
  \item Use the \texttt{@doc} annotation to write documentation for each function.  
  \item Include usage examples (doctest) for each function.  
\end{itemize}
\end{column}

\begin{column}[t]{0.55\textwidth}
\textbf{Code Example:}
\begin{lstlisting}[language=Elixir, basicstyle=\ttfamily\tiny]
defmodule Calculator do
  @moduledoc """
  A module for basic arithmetic operations.
  """

  @doc """
  Multiplies two numbers.

  ## Example
      iex> Calculator.multiply(4, 5)
      20
  """
  def multiply(x, y), do: x * y

  @doc """
  Divides two numbers.

  ## Example
      iex> Calculator.divide(10, 2)
      5
  """
  def divide(x, y), do: x / y
end
\end{lstlisting}
\end{column}

\end{columns}
\end{frame}

\subsection{Writing Unit Tests for the Module}
\begin{frame}[fragile]{Exercise 2: Writing Unit Tests for the Module}
\vspace{20pt}
\begin{columns}

\begin{column}[t]{0.42\textwidth}
\textbf{Instructions:}
\begin{itemize}
  \item Create a file named \texttt{calculator\_test.exs} in the \texttt{test} folder.  
  \item Use \texttt{ExUnit.Case} and \texttt{doctest}.  
  \item Write tests for:
    \begin{itemize}
      \item Multiplying two numbers.  
      \item Dividing two numbers.  
      \item Division by zero (should raise an error).  
    \end{itemize}
\end{itemize}
\end{column}

\begin{column}[t]{0.58\textwidth}
\textbf{Example File:}
\begin{lstlisting}[language=Elixir, basicstyle=\ttfamily\scriptsize]
defmodule CalculatorTest do
  use ExUnit.Case
  doctest Calculator

  test "multiplies two numbers" do
    assert Calculator.multiply(4, 5) == 20
  end

  test "divides two numbers" do
    assert Calculator.divide(10, 2) == 5
  end

  test "divides by zero raises an error" do
    assert_raise ArithmeticError,
      fn -> Calculator.divide(10, 0) end
  end
end
\end{lstlisting}
\end{column}

\end{columns}
\end{frame}

\subsection{Running Tests for the Module}
\begin{frame}[fragile]{Exercise 3: Running Tests for the Module}
\vspace{15pt}
After writing \texttt{calculator\_test.exs},  
run the unit tests for the \texttt{Calculator} module with the following command:

\begin{lstlisting}[language=bash]
mix test test/calculator_test.exs
\end{lstlisting}

\textbf{Output:} The terminal will show the test results,  
the number of successful tests, and details if any test fails.  
\end{frame}

\subsection{Running a Specific Test Function}
\begin{frame}[fragile]{Exercise 4: Running a Specific Test Function}
\vspace{15pt}
Add a new test in \texttt{calculator\_test.exs} for the \texttt{multiply/2} function,  
then run that test using its line number.  

\begin{columns}
\begin{column}[t]{0.6\textwidth}
\textbf{Example File:}
\begin{lstlisting}[language=Elixir, basicstyle=\ttfamily\tiny]
defmodule CalculatorTest do
  use ExUnit.Case
  doctest Calculator

  test "multiplies two numbers" do
    assert Calculator.multiply(4, 5) == 20
  end

  # New test added for specific function
  test "checks multiply with negative numbers" do
    assert Calculator.multiply(-4, 5) == -20
  end
end
\end{lstlisting}
\end{column}

\begin{column}[t]{0.4\textwidth}
\textbf{Running the Test:}  
If the new test was added at line 6, run the following command:  

\begin{lstlisting}[language=bash]
mix test test/calculator_test.exs:6
\end{lstlisting}

This command will only run the test on that line,  
making debugging or partial testing easier.
\end{column}
\end{columns}
\end{frame}

\section{Summary}
\begin{frame}{Summary}
\vspace{15pt}
\begin{columns}

\begin{column}[t]{0.5\textwidth}
\begin{itemize}
  \item \textbf{Documentation in Elixir}  
    \begin{itemize}
      \item Use \texttt{@moduledoc} for modules  
      \item Use \texttt{@doc} for functions  
      \item Generate HTML docs with \texttt{ex\_doc} and \texttt{mix docs}  
    \end{itemize}

  \item \textbf{Exercises}  
    \begin{itemize}
      \item Create a module with documentation  
      \item Write unit tests in a separate file  
      \item Run tests for modules \& specific functions  
    \end{itemize}
\end{itemize}
\end{column}

\begin{column}[t]{0.5\textwidth}
\begin{itemize}
  \item \textbf{Unit Testing with ExUnit}  
    \begin{itemize}
      \item Write tests in \texttt{\_test.exs} files with \texttt{ExUnit.Case}  
      \item Use \texttt{assert}, \texttt{refute}, \texttt{assert\_raise}  
      \item Use \texttt{doctest} to verify examples in documentation  
    \end{itemize}

  \item \textbf{Running Tests}  
    \begin{itemize}
      \item \texttt{mix test} to run all tests  
      \item \texttt{mix test test/file\_name.exs} for a specific file  
      \item \texttt{mix test file.exs:line} for a specific function  
    \end{itemize}
\end{itemize}
\end{column}

\end{columns}
\end{frame}



\end{document}
