\documentclass[aspectratio=169, table]{beamer}

\usepackage[utf8]{inputenc}
\usepackage{listings} 

\usetheme{Pradita}

\subtitle{IF140303-Web Application Development}

\title{\LARGE{Session-03:\\ Elixir Documentation and Testing}
	\vspace{20pt}}
\date[Serial]{\scriptsize {PRU/SPMI/FR-BM-18/0222}}
\author[Pradita]{\small{\textbf{Alfa Yohannis}}}

\lstdefinelanguage{Elixir} {
	keywords={def, defmodule, do, end, for, if, else, true, false},
	basicstyle=\ttfamily\small,
	keywordstyle=\color{blue}\bfseries,
	ndkeywords={@moduledoc, iex, Enum, @doc},
	ndkeywordstyle=\color{purple}\bfseries,
	sensitive=true,
	commentstyle=\color{gray},
	stringstyle=\color{red},
	numbers=left,
	numberstyle=\tiny\color{gray},
	breaklines=true,
	frame=lines,
	backgroundcolor=\color{lightgray!10},
	tabsize=2,
	comment=[l]{\#},
	morecomment=[s]{/*}{*/},
	commentstyle=\color{gray}\ttfamily,
	stringstyle=\color{purple}\ttfamily,
	showstringspaces=false
}

\begin{document}
	
	\frame{\titlepage}
	
	\begin{frame}
		\frametitle{Overview of Elixir Documentation}
		\begin{itemize}
			\item Elixir provides powerful tools for creating documentation directly within your code.
			\item \texttt{@moduledoc} and \texttt{@doc} are used to document modules and functions.
			\item Mix tasks like \texttt{mix docs} can generate HTML documentation from your code.
		\end{itemize}
	\end{frame}
	
	\begin{frame}[fragile]
		\frametitle{Adding Dependencies for Documentation}
		\begin{itemize}
			\item To generate documentation, add \texttt{ex\_doc} to your \texttt{mix.exs} dependencies.
		\end{itemize}
		\begin{lstlisting}[language=Elixir]
			defp deps do
			[
			{:ex_doc, "~> 0.12"}
			]
			end
		\end{lstlisting}
	\end{frame}
	
	\begin{frame}
		\frametitle{\texttt{@moduledoc}: Module Documentation}
		\begin{itemize}
			\item Use \texttt{@moduledoc} at the top of your module to provide an overview.
			\item Describes the purpose, usage, and any relevant details of the module.
		\end{itemize}
	\end{frame}
	
	\begin{frame}[fragile]
		\frametitle{Example: \texttt{@moduledoc}}
		\begin{lstlisting}[language=Elixir]
			defmodule Lottery do
			@moduledoc """
			This module provides functionalities for managing a lottery system.
			It includes functions for creating, shuffling, checking for numbers,
			saving, loading, and distributing numbers within the lottery pool.
			"""
			end
		\end{lstlisting}
	\end{frame}
	
	\begin{frame}
		\frametitle{\texttt{@doc}: Function Documentation}
		\begin{itemize}
			\item \texttt{@doc} is used to document individual functions.
			\item Include descriptions, parameters, return values, and examples.
		\end{itemize}
	\end{frame}
	
	\begin{frame}[fragile]
		\frametitle{Example: \texttt{@doc}}
		\begin{lstlisting}[language=Elixir]
			@doc """
			Generates a pool of lottery numbers with different pots.
			
			## Returns
			- A list of lottery numbers with their respective pot numbers.
			
			## Examples
			iex> Lottery.generate_pool()
			["Number 1 in Pot 1", "Number 2 in Pot 1", ...]
			"""
			def generate_pool do
			# implementation
			end
		\end{lstlisting}
	\end{frame}
	
	\begin{frame}
		\frametitle{Generating Documentation}
		\begin{itemize}
			\item Run \texttt{mix docs} to generate HTML documentation.
			\item The output will be stored in the \texttt{doc} directory.
			\item View it by opening \texttt{index.html} in your browser.
		\end{itemize}
	\end{frame}
	
	\begin{frame}
		\frametitle{Overview of Testing in Elixir}
		\begin{itemize}
			\item Elixir uses \texttt{ExUnit} for writing and running tests.
			\item \texttt{ExUnit.Case} is used to define a test module.
			\item Tests are written as functions prefixed with \texttt{test}.
		\end{itemize}
	\end{frame}
	
	\begin{frame}[fragile]
		\frametitle{Basic Example of a Test}
		\begin{lstlisting}[language=Elixir]
			defmodule LotteryTest do
			use ExUnit.Case
			
			test "generate_pool creates 20 lottery numbers" do
			assert length(Lottery.generate_pool()) == 20
			end
			end
		\end{lstlisting}
	\end{frame}
	
	\begin{frame}
		\frametitle{Doctests in Elixir}
		\begin{itemize}
			\item Doctests allow you to run the examples in your documentation as tests.
			\item Ensure the examples in your \texttt{@doc} annotations are correct.
			\item Use \texttt{doctest ModuleName} in your test module.
		\end{itemize}
	\end{frame}
	
	\begin{frame}[fragile]
		\frametitle{Adding Doctests to Your Tests}
		\begin{lstlisting}[language=Elixir]
			defmodule LotteryTest do
			use ExUnit.Case
			doctest Lottery
			end
		\end{lstlisting}
	\end{frame}
	
	\begin{frame}
		\frametitle{Running Tests}
		\begin{itemize}
			\item Run \texttt{mix test} to execute all your tests.
			\item Ensure your code is thoroughly tested and your documentation examples are correct.
		\end{itemize}
	\end{frame}
	
	\begin{frame}
		\frametitle{Summary}
		\begin{itemize}
			\item We covered how to document Elixir code using \texttt{@moduledoc} and \texttt{@doc}.
			\item We explored how to write and run tests, including using doctests.
			\item Proper documentation and testing are essential for maintainable and reliable code.
		\end{itemize}
	\end{frame}
	
\end{document}
