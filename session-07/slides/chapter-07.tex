\documentclass[aspectratio=169, table]{beamer}

\usepackage[utf8]{inputenc}
\usepackage{listings} 

\usetheme{Pradita}

\subtitle{IF140303-Web Application Development}

\title{\LARGE{Session-07:\\ Introduction to the Phoenix Framework}
	\vspace{20pt}}
\date[Serial]{\scriptsize {PRU/SPMI/FR-BM-18/0222}}
\author[Pradita]{\small{\textbf{Alfa Yohannis}}}

\lstdefinelanguage{Elixir} {
	keywords={def, defmodule, do, end, for, if, else, true, false},
	basicstyle=\ttfamily\small,
	keywordstyle=\color{blue}\bfseries,
	ndkeywords={@moduledoc, iex, Enum, @doc, mix, phoenix},
	ndkeywordstyle=\color{purple}\bfseries,
	sensitive=true,
	commentstyle=\color{gray},
	stringstyle=\color{red},
	numbers=left,
	numberstyle=\tiny\color{gray},
	breaklines=true,
	frame=lines,
	backgroundcolor=\color{lightgray!10},
	tabsize=2,
	comment=[l]{\#},
	morecomment=[s]{/*}{*/},
	commentstyle=\color{gray}\ttfamily,
	stringstyle=\color{purple}\ttfamily,
	showstringspaces=false
}

\begin{document}
	
	\frame{\titlepage}
	
	\begin{frame}
		\frametitle{\shortstack{Introduction to the Phoenix \\Framework}}
		\begin{itemize}
			\item Phoenix is a web development framework for building scalable and maintainable applications in Elixir.
			\item It is known for its performance, fault tolerance, and real-time capabilities.
			\item Phoenix requires installation of Phoenix itself, Node.js for asset management, and PostgreSQL for database management.
		\end{itemize}
	\end{frame}
	
	\begin{frame}
		\frametitle{\shortstack{Why Install Node.js and\\ PostgreSQL?}}
		\begin{itemize}
			\item \textbf{Node.js}: Handles assets like JavaScript and CSS, making it essential for the front-end part of the Phoenix application.
			\item \textbf{PostgreSQL}: A robust, open-source relational database system that Phoenix uses via Ecto for managing data persistence.
		\end{itemize}
	\end{frame}
	
	\begin{frame}
		\frametitle{Phoenix Workflow Overview}
		\begin{itemize}
			\item \textbf{Incoming Request}: Phoenix receives a request from the client.
			\item \textbf{Ensure HTML Request}: Phoenix checks if the request is for an HTML response.
			\item \textbf{Session Check}: Verifies if the request has an active session.
			\item \textbf{Security Check}: Performs necessary security validations.
			\item \textbf{HTTP Headers}: Adds appropriate headers for browser compatibility.
			\item \textbf{Access Request}: Determines what resource the request is trying to access.
			\item \textbf{Formulate Response}: Phoenix formulates the response and sends it back to the client.
		\end{itemize}
	\end{frame}
	
	\begin{frame}[fragile]
		\frametitle{Generating a Phoenix Project}
		\begin{itemize}
			\item To create a new Phoenix project, use the command:
			\begin{lstlisting}[language=Elixir]
				mix phx.new <name>
			\end{lstlisting}
			\item Navigate to the project directory and create the database:
			\begin{lstlisting}[language=Elixir]
				mix ecto.create
			\end{lstlisting}
			\item Configure the database by editing \texttt{dev.exs}:
		\end{itemize}
	\end{frame}
	
	\begin{frame}[fragile]
		\frametitle{Database Configuration in \texttt{dev.exs}}
		\begin{lstlisting}[language=Elixir]
			config :reddit, Reddit.Repo,
			adapter: Ecto.Adapters.Postgres,
			username: "postgres",
			password: "postgres",
			database: "reddit_dev",
			hostname: "localhost",
			pool_size: 10
		\end{lstlisting}
		\begin{itemize}
			\item Adjust the \texttt{username} and \texttt{password} according to your PostgreSQL setup.
			\item Start the Phoenix server with:
			\begin{lstlisting}[language=Elixir]
				mix phx.server
			\end{lstlisting}
		\end{itemize}
	\end{frame}
	
	\begin{frame}
		\frametitle{\shortstack{Project Overview: Reddit-like\\ Discussion Forum}}
		\begin{itemize}
			\item The project is a discussion forum similar to Reddit.
			\item Users can sign in with GitHub, view posts, create discussions, and comment.
			\item Users can also edit or delete their posts.
		\end{itemize}
	\end{frame}
	
	\begin{frame}
		\frametitle{\shortstack{Server-Side Templating vs\\Single Page Apps (SPA)}}
		\begin{itemize}
			\item \textbf{Server-Side Templating}: The server generates HTML dynamically and sends it to the client.
			\item \textbf{Single Page Apps}: The client dynamically updates content without reloading the page, often using frameworks like React or Angular.
			\item Phoenix supports both approaches, allowing flexibility in web application design.
		\end{itemize}
	\end{frame}
	
	\begin{frame}
		\frametitle{Phoenix Layout and MaterializeCSS}
		\begin{itemize}
			\item Phoenix uses layout files to define the overall structure of HTML pages.
			\item MaterializeCSS is a modern front-end framework similar to Bootstrap, providing ready-to-use components and a responsive grid system.
			\item Add MaterializeCSS to your project by including its link tag in your layout file.
			\item Location at: web > templates > layout > app.html.eex
		\end{itemize}
	\end{frame}
	
	\begin{frame}[fragile]
		\frametitle{\shortstack{Implementing the Header\\with MaterializeCSS}}
		
		\begin{lstlisting}[language=Elixir]
			<body>
			<nav class="light-blue">
			<div class="nav-wrapper container">
			<a href="/" class="brand-logo">Logo</a>
			<ul class="right">
			<%= if @conn.assigns[:user] do %>
			<li>
			<%= link "Logout", to: session_path(@conn, :signout) %>
			</li>
		\end{lstlisting}
	\end{frame}
	
	\begin{frame}[fragile]
		\frametitle{\shortstack{Implementing the Header\\with MaterializeCSS}}
		\begin{lstlisting}[language=Elixir]
			<% else %>
			<li>
			<%= link "Sign in with Github", to: session_path(@conn, :request, "github") %>
			</li>
			<% end %>
			</ul>
			</div>
			</nav>
		\end{lstlisting}
	\end{frame}
	
	\begin{frame}
		\frametitle{Summary}
		\begin{itemize}
			\item We introduced the Phoenix framework and discussed the necessary installations.
			\item Covered the Phoenix request-response workflow and project generation.
			\item Reviewed the project requirements, server-side templating vs. SPA, and implemented a MaterializeCSS header.
		\end{itemize}
	\end{frame}
	
\end{document}
